\documentclass[12pt,authoryear, notitlepage]{elegantpaper} 

% =========================
% CORE MATH & SYMBOLS
% =========================
\usepackage{amsmath, amsfonts, amssymb}

% =========================
% PAGE LAYOUT
% =========================
\usepackage{geometry}
\geometry{a4paper, margin=1in}
\renewcommand{\baselinestretch}{2} % line spacing

% =========================
% FIGURES, TABLES & FLOATS
% =========================
\usepackage{graphicx}
\usepackage{adjustbox}              % resize boxes/tables that exceed \textwidth
\usepackage{subcaption}             % sub-figures
\usepackage[figuresright]{rotating} % sideways figures/tables
\usepackage{float}                  % improved float placement
\usepackage{booktabs, threeparttable}
\usepackage{dcolumn}
\usepackage{multirow}
\usepackage{diagbox}

% =========================
% TIKZ & DIAGRAMS
% =========================
\usepackage{tikz}
\usetikzlibrary{arrows.meta, positioning}

% =========================
% PDF & LINKS
% =========================
\usepackage{pdfpages}
\usepackage{url}      % robust URL typesetting for bib/footnotes
\usepackage{natbib}   % load before hyperref for compatibility
\usepackage{hyperref} % keep close to the end of package list

% =========================
% TEXT, QUOTES, UTILITIES
% =========================
\usepackage{dirtytalk}  % \say{}
\usepackage{bbm}        % \mathbbm{1}
\usepackage{lipsum}     % dummy text
\usepackage{setspace}   % fine-grained spacing control


\newenvironment{acknowledgments}
  {\renewcommand{\abstractname}{Acknowledgments}\begin{abstract}}
  {\end{abstract}}

\newenvironment{dedication}
  {\cleardoublepage \thispagestyle{empty} \vspace*{\fill} \begin{center}}
  {\end{center} \vspace*{\fill} \clearpage}


% =========================
% SECTIONING & NUMBERING
% =========================
\usepackage{titlesec}
\renewcommand{\thesection}{\thechapter.\arabic{section}} % show as X.1, X.2, ...
\setcounter{secnumdepth}{3}
\makeatletter
\@addtoreset{section}{chapter} % reset section counter at each chapter
\makeatother

% =========================
% PATCHES & CUSTOM ENVIRONMENTS
% =========================
\usepackage{xpatch}
\makeatletter
% Reduce default chapter top space (if using book/report)
\newlength{\chaptertopskip}
\setlength{\chaptertopskip}{10pt}
\xpatchcmd{\@makechapterhead}{\vspace*{50\p@}}{\vspace*{\chaptertopskip}}{\typeout{Success}}{\typeout{Failure!!!}}
\makeatother

% Dialogue environment customization
\usepackage{dialogue}
\makeatletter
\renewenvironment{dialogue}{%
  \begin{list}{}{%
    \setlength\itemsep{\z@ \@plus .5ex}%
    \setlength{\parsep}{\parskip}%
    \setlength{\rightmargin}{0pt}%
    \setlength{\labelsep}{0.5em}%
    \setlength{\leftmargin}{\labelwidth}%
    \addtolength{\leftmargin}{\labelsep}%
    % NOTE: If you intended a custom macro, consider \newcommand instead of \defcommand
    \defcommand\speak[2]{\item[\hfill {##1}] {\itshape ``{##2}''}}%
    \let\makelabel\DialogueLabel
  }%
  \PreDialogue\relax
}{%
  \end{list}%
}
\makeatother

% Footnote spacing tweak
\usepackage{etoolbox}
\makeatletter
\patchcmd{\@footnotetext}{\setspace@singlespace}{0.8}{}{}
\makeatother
\footnotesep=10pt

% =========================
% LOAD LAST
% =========================
\usepackage{subfiles} % Best loaded last in the preamble

\RequirePackage{silence}
\WarningFilter{etex}{Extended allocation already in use}
\usepackage{titlesec}

% 重新定义 section 编号格式
\renewcommand{\thesection}{\arabic{section}}
\setcounter{secnumdepth}{3}
%----------------------------------------------------------------%
%----------------------------------------------------------------%


\title{Essays on Time-Varying Oligopsonistic Competition, Storage Dynamics, and Welfare in Agricultural Markets}
\author{Zhiyao (Yao) Ma}
\institute{Committee: Richard J. Sexton, Jeffrey Williams, and Stephen Boucher}
\date{\today}

%----------------------------------------------------------------%
%----------------------------------------------------------------%

\begin{document}
\maketitle
\setcounter{tocdepth}{2}

\tableofcontents

%----------------------------------------------------------------%
%----------------------------------------------------------------%
\newpage
\chapter{Oligopsony, Storage Inefficiency, and Welfare in China’s Apple Industry}

\begin{abstract}
This chapter addresses three key questions:
\begin{enumerate}
    \item What does the fresh apple industry look like in China?
    \item How does upstream buyer power influence the supply chain?
    \item Why does inefficiency persist in the storage process?
\end{enumerate}
\end{abstract}

\input{Overleaf/chap1_main}




%----------------------------------------------------------------%
%----------------------------------------------------------------%
\newpage
\chapter{Time-Varying Oligopsonistic Competition and Storage Adoption: Implications for Smallholder Farmers' Welfare}

\begin{abstract}
This chapter is the first study to reveal the dynamics of time-varying oligopsonistic competition and storage adoption, and their impact on smallholder farmers' welfare. While existing research explores storage incentives like risk preferences and trading costs, it overlooks the advantages of quality-preserving technologies for accessing time-varying competitive markets. Cold storage adoption, especially for perishable crops, helps farmers overcome trading-time limitations. In a two-period model for a storable cash crop, I find that temporal changes in trader competition alone suffice for farmers to benefit from storage. This work has broader relevance to settings with multiple input suppliers selling to a limited set of traders involved in market division and price fixing over time.


\keywords{Storage Adoption, Time-varying Oligopsony, Farmer Welfare}

\end{abstract}

%--------------------------------------------------------%

\section{Introduction}
\noindent    Buyer power arises from the immobility of certain factor inputs that, in the short run, are largely “captive” to a limited set of buyers. While spatial immobility has been extensively explored in the literature, the impact of changes in temporal immobility on bargaining dynamics remains underexamined. This gap is particularly relevant in agricultural markets, where products are typically perishable, and storage plays a critical role in the supply chain.

Smallholder farmers in developing countries often contend with the oligopsony power of middlemen and processors \footnote{In this study, the terms middlemen, field buyers, intermediaries, and traders will be used interchangeably to refer to a group of buyers who directly purchase fresh apples from farmers. These buyers play a crucial role in the supply chain by acting as the initial link between farmers and the broader market. They typically visit orchards, negotiate prices with farmers, and handle the immediate procurement of apples. Their role may also include transporting apples to wholesale markets, processing units, or exporters, depending on the supply chain dynamics. Regardless of the specific term used, all these buyers share the common function of directly sourcing apples from farmers before the produce reaches larger market players, such as wholesalers, retailers, or processors.}, leading to substantial margins for the latter \citep{rogers_rich_1994assessing}. While existing literature primarily explores the correlation between market power and "spatial" trading frictions such as transportation and price-search costs \citep{bergquist_dinerstein_2020,mitra_mookherjee_torero_visaria_2018,ranjan_2017,antras_costinot_2011}, the reasons behind small farmers frequently missing out on inter-temporal marketing opportunities remain a subject of controversy \citep{williams1991storage, wright1984welfare, ruhinduka2020smallholder, lai2003optimal}.

Facing time-sensitive post-harvest decisions, small-scale farmers must choose between immediate and delayed selling, with limited access to quality-preserving technologies like cold storage constraining their marketing window \citep{aggarwal2018grain}. Despite some developing countries offering partial storage subsidies, the inherent variability in agricultural prices and the fixed cost of initial storage construction often make storage adoption a challenging investment for smallholder farmers, particularly in cash crops.

In reality, predicting storage returns proves challenging for farmers, and the volatility in farm-gate prices can discourage them from storing crops for future sales, even with access to credit \citep{cardell2023price}. Previous literature emphasizes the pivotal role of factors such as storage costs, downstream supply and demand shocks, perishability of produce, and risk aversion in influencing farmers' post-harvest decisions and welfare consequences. However, the potential competitive advantages of storage, enabling farmers to enter more competitive procurement-market conditions, have received limited exploration in the existing literature.

To the best of our knowledge, this study is the first to unveil the interplay of farmers' (sellers') storage adoption with time-varying procurement market conditions, both theoretically and empirically. Without the adoption of storage, farmers who cultivate crops that spoil quickly face limitations in their trading options, as they can only sell their produce locally at harvest time, because smallholder farmers usually lack access to trucks which prevents them from accessing distant selling locations. But if they are equipped with advanced storage technology, they would be able to seek out a higher local price brought from more competitive market conditions among middlemen in the later periods as shown in Figure \ref{Figure: Demo}.

\begin{figure}[ht]
\centering
\includegraphics[width=1\textwidth]{figures/graphic_demo.png}
\caption{Extended Marketing Opportunities from Storage Adoption}
\label{Figure: Demo}
\end{figure}

Specifically, farmers could potentially benefit from increased competition in the oligopsonistic market through two sources. Firstly, the presence of different intermediaries in the village at various times can create fluctuating levels of competition on a monthly or even weekly basis. Traders may perceive villages with greater storage as having a lower instantaneous supply of storable goods due to a smaller fraction of farmers being forced to sell. This perception leads traders to be willing to offer higher prices in these villages to secure the available supply. This allows farmers with storage options to sell their produce at the most advantageous time. Secondly, farmers with storage can tap into additional distribution channels such as e-commerce and direct selling, which differ from the conventional middlemen-dominated system. By doing so, they can introduce external participants into the oligopsonistic market at the farm gate at different time nodes. 

I develop a conceptual framework to explore how smallholder farmers adopt cold storage to bargain against the time-varying oligopsony power of middlemen. It considers a simplified dynamic two-period scenario in a developing country, where farmers sell a specific cash crop to middlemen due to high transaction costs associated with accessing larger markets. The model incorporates a storing-decision process, wherein farmers, observing farm-gate prices at harvest, decide whether to sell or store their crops. The framework shows that, absent an opportunity to observe more "draws" of the buyer-side competitive conditions, farmers would always sell during the first post-harvest period and avoid carrying costs or quality deterioration. The empirical evidence from the post-harvest decisions by small-scale apple growers in Central China further confirms the theoretical prediction. 

The outcomes of this study hold substantial implications for policymakers and farmers alike. By exploring the dynamics of time-varying oligopsony levels, the research aims to demonstrate the potential benefits of storage facilities for both producers and consumers. In a broader context, the findings suggest that embracing storage adoption could provide farmers with an effective alternative to combat anticompetitive practices like market division and avoid more intrusive measures into markets, such as governments fixing prices.

%--------------------------------------------------------%
\section{Literature Review}
    \subfile{chap2_literature_review.tex}



%--------------------------------------------------------%
\section{Conceptual Framework}
    \subfile{chap2_CF.tex}




%--------------------------------------------------------%
%--------------------------------------------------------%
\section{Empirical Analysis}


%--------------------------------------------------------%
\section{Conclusion}
\noindent This chapter demonstrates that on-farm storage adoption can have a dynamic impact on the welfare of farmers facing oligopsony power, providing inter-temporal arbitrage opportunities and stronger bargaining power over time. Cold storage adoption, especially for storable crops, helps farmers overcome trading-time limitations and benefit even from temporal changes in trader competition alone. Examining storage from a market competition perspective offers a fresh approach to analyzing the influence of inventory on growers' and sellers' marketing strategies and their bargaining power.

My baseline model provides a theoretical basis for small-scale farmers' storage and marketing choices in developing countries. It focuses on farmers' intra-seasonal strategies and the storing-decision process, where farmers observe farm-gate prices at harvest and decide whether to immediately sell some, all, or none of their crops. When farmers have insufficient bargaining power at a single point in time, the adoption of storage at lower storage costs would give them inter-temporal arbitrage opportunities. 

The empirical evidence from post-harvest decisions by small-scale apple growers in Central China supports the theoretical predictions, highlighting the vulnerability of farmers to significant price cutting from middlemen in the absence of proper cold storage adoption. 

Therefore, in many developing countries, future government policies can replace or supplement the existing minimum purchase price and direct cash transfers by subsidizing farmers' investment in storage infrastructure. Instead of intervening directly into markets, by encouraging the use of on-farm technology in a broad sense to sell their harvest at an optimal time, the government can empower weak farmers to gain stronger bargaining power against middlemen and hence benefit both consumers and growers in agricultural industry supply chains.

This chapter's application extends beyond the fresh apple industry to various settings. Cold storage and inventory play a crucial role in transactions, commonly seen as tools to help sellers take advantage of business cycle fluctuations. However, the competitive dynamics among buyers can vary over time, even without changes in demand or supply elsewhere in the industry chain. By examining storage from a market competition perspective, this model offers a fresh approach to analyzing the influence of inventory on growers' and sellers' marketing strategies and their bargaining powers in agricultural commodity industries.












\newpage
\bibliography{reference.bib}

%--------------------------------------------------------%
\appendix

\newpage
\includepdf[pages=-]{Appendix/farmers_survey.pdf}

\end{document}