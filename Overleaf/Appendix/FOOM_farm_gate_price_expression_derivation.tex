\section{Derivation of the Farm Price under the FOOM Model Framework} 
\label{Appendix: Derivation of Farm Price under the FOOM Framework}

\noindent This appendix derives the equilibrium farm-gate price under the Flexible-Oligopoly-Oligopsony Market (FOOM) model. The analysis focuses on a local agricultural market where buyers procure a homogeneous crop from many farmers. Farm-level buyers are assumed to sell into a perfectly competitive downstream market---that is, they take the downstream (e.g., wholesale) price $P$ as given---and incur a constant marginal cost $c$ per unit for marketing, storage, transportation, or processing.

\subsection{Model Setup}
\noindent
Let $p$ denote the farm-gate price paid to farmers. The total net price received by the buying sector from selling the processed or marketed product is therefore $(P - c)$. Buyers procure aggregate quantity $Q$ from farmers, who collectively supply according to an upward-sloping supply curve $Q(p), Q'>0$. Equivalently in its inverse form, supply can be expressed as $p(Q), p'>0$.

The elasticity, $\varepsilon$, of this supply curve reflects farmers' ability to shift sales across markets or across time:
$$
\varepsilon \equiv \frac{p}{Q}Q'.
$$

Let $\theta \in [0,1]$ denote the oligopsony conduct parameter, which captures the degree of market power on the buyer side:
\begin{itemize}
    \item $\theta = 0$: perfect competition among buyers (no strategic interaction);
    \item $\theta = 1$: pure monopsony (a single buyer fully internalizes its price impact);
    \item $0 < \theta < 1$: oligopsony with partial internalization of the price effect.
\end{itemize}

\subsection{Buyer's Optimization Problem}
\noindent
We can model buyers as acting collectively subject to conduct parameter $\theta$ by choosing the procurement quantity $Q$ to maximize profit:
$$
\max_{Q}\; \Pi(Q) = (P - c) Q - p(Q) Q.
$$
Differentiating with respect to $Q$ yields:
$$
\frac{d\Pi}{dQ} = (P - c) - \Big[p(Q) + Q \frac{dp(Q)}{dQ}\Big].
$$
Under perfect competition ($\theta = 0$), buyers take $p$ as given, while under monopsony ($\theta = 1$), they fully internalize the price effect of expanding procurement. The general FOOM model optimization condition interpolates between these cases by weighting the strategic effect with $\theta$:
$$
(P - c) - p(Q) - \theta Q \frac{dp(Q)}{dQ} = 0.
$$
Rearranging gives the industry first-order condition (FOC):
$$
P - c = p + \theta \frac{dp}{dQ} Q.
$$
The left-hand side represents the constant marginal revenue from downstream sales, , i.e., the value of marginal product $VMP$ of the farm product, while the right-hand side is the buyers' perceived marginal procurement cost, which increases with $\theta$ as buyers internalize more of the effect on procurement price from expanding production.

Under a standard convex inverse supply curve $p(Q)$, the second-order condition for profit maximization is satisfied:
$$
\frac{d^2\Pi}{dQ^2}
  = -\Big[(1+\theta)\frac{dp}{dQ} + \theta Q \frac{d^2p}{dQ^2}\Big] < 0.
$$


\subsection{FOOM Model Price Equation}
\noindent
Substituting $\frac{dp}{dQ} = \frac{p}{\varepsilon Q}$, implied by the elasticity definition, into the FOC gives:

$$
P - c = p + \theta \left(\frac{p}{\varepsilon Q}\right) Q 
= p \left(1 + \frac{\theta}{\varepsilon}\right).
$$

Thus, the farm-gate price can be expressed as:

$$
p = \frac{P - c}{1 + \theta / \varepsilon}.
$$

Equivalently, the oligopsony ``markdown'' wedge satisfies:

$$
\frac{(P - c) - p}{p} = \frac{\theta}{\varepsilon}.
$$

This equation is the Lerner index, the percentage markdown in the farm price relative to the competitive benchmark $p = VMP$. It equals the ratio of the conduct parameter to the supply elasticity. A higher $\theta$ (stronger buyer oligopsony power) depresses $p$, while a higher $\varepsilon$ (more elastic farm response) mitigates the markdown.


Normalizing $P - c = 1$ for simplicity yields the compact form presented in equation (\ref{simplified-price}) in the main text:

$$
p = \frac{\varepsilon}{\varepsilon + \theta}.
$$


