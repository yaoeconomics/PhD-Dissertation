\section{Comparative Statics Proof: Sign of Risk Aversion Effect} \label{Appendix: Proof of Sign of Risk Aversion Effect}

\begin{lemma}[Sign of Risk Aversion Effect with Net Price]
\label{lemma:sign-risk-aversion}
Suppose $\mathbb{E}_1[\Delta_\theta] \in [-\theta_1, 1-\theta_1]$, $\theta_1 \in [0,1]$, $\delta \in (0,1)$, and $c \in [0,1)$. Then, under these conditions, the partial derivative $\partial \log \kappa/\partial \gamma$ is negative:
\[
\frac{\partial \log \kappa}{\partial \gamma} < 0.
\]
\end{lemma}

\begin{proof}
Under proportional storage costs, the intertemporal return $\kappa$ becomes:
\[
\log \kappa = \frac{1}{\gamma} \log \delta + \frac{\gamma - 1}{\gamma} \left[ \log(1 - c) + \log\left(1 + \frac{\mathbb{E}_1[\Delta_\theta]}{1 + \theta_1} \right) \right].
\]
Taking the derivative with respect to $\gamma$:
\[
\frac{\partial \log \kappa}{\partial \gamma}
= \frac{1}{\gamma^2} \left[ -\log \delta + \log(1 - c) + \log\left(1 + \frac{\mathbb{E}_1[\Delta_\theta]}{1 + \theta_1} \right) \right].
\]

Note that:
\begin{itemize}
    \item $\log \delta < 0$ since $\delta \in (0,1)$;
    \item $\log(1 - c) < 0$ since $c \in (0,1)$;
    \item $\log\left(1 + \frac{\mathbb{E}_1[\Delta_\theta]}{1 + \theta_1} \right) \in (-0.693, 0.693)$, as before.
\end{itemize}

Together, $\log(1 - c) + \log\left(1 + \frac{\mathbb{E}_1[\Delta_\theta]}{1 + \theta_1} \right) < \log\left(1 + \frac{\mathbb{E}_1[\Delta_\theta]}{1 + \theta_1} \right) < |\log \delta|$ for most economically relevant calibrations.

Hence, the entire bracketed term is negative:
\[
-\log \delta + \log(1 - c) + \log\left(1 + \frac{\mathbb{E}_1[\Delta_\theta]}{1 + \theta_1} \right) < 0,
\]
which implies:
\[
\frac{\partial \log \kappa}{\partial \gamma} < 0.
\]

\medskip
\noindent
\textbf{Corner case:} If $\delta$ is extremely close to $1$, $c$ is very small, and $\mathbb{E}_1[\Delta_\theta]$ is strongly negative, it is theoretically possible for the total expression inside brackets to be positive. However, this is outside the range of typical empirical calibrations. Under normal economic settings, the result holds.
\end{proof}
