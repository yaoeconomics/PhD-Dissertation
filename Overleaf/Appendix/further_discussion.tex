%------------------------------------------------------%
\newpage
\section{Further Discussions}
\subsection{Time-Varying Competition Estimation}
\noindent To quantitatively evaluate the competitiveness of the oligopsonistic market faced by farmers, the study will employ an adapted version of the Lerner Index, denoted by $H$ as follows:
\begin{equation}
    H = \frac{E_t(P^w_{t+1}) - MC_t - P_t^f - S_{t,t+1}^w}{E_t(P^w_{t+1})}
\end{equation}
where $P^w_{t+1}$ captures the wholesaling price of downstream sales in month $t+1$ reported by intermediaries, $P_t^f$ represents the farm-gate price reported by farmers after the harvest in month $t$, $S_{t,t+1}^w$ shows the trader's storage cost from period $t$ to $t+1$, and $MC_t$ depicts the sum of other marginal costs of traders. In other words, the index captures the ratio of the trader's mark-up and their downstream selling price. 

The modified Lerner index could be also regarded as the pass-through of marginal product into the farm-gate price as the primary proxy for buyer's power, like in the approaches used by \cite{bergquist_dinerstein_2020} and \cite{atkin2015s}. The closer this ratio is to zero, $H \rightarrow 0$, the more competitive the oligopsonistic markets are perceived to be.

While our modified Lerner index offers a succinct measure of monopsony power, its practical application is limited due to challenges in accurately measuring traders' costs. However, since the inter-temporal behavior of prices and costs is our main focus, the delta of the Lerner Index should be reliable when we have a consistent series of traders' operational costs and storage expenses. 

To collect traders' marginal cost data and to control for demand shocks and variability, I will conduct another round of surveys on traders. The survey will include their operational costs and trading volume, and inquire about the existence of significant market events throughout the supply chain, such as fluctuations in market demand and changes in consumer preferences, natural disasters, or export/import markets.


%------------------------------------------------------%
\subsection{Extended Marketing Opportunities}
\noindent The adoption of cold storage has greatly extended the marketing opportunities of farmers in terms of both timing and channels. The sample of 549 households has confirmed that cold storage could greatly improve the bargaining power of farmers and boost their expected income through two primary channels:
\begin{itemize}
    \item \textbf{Better Sales Timing}: Cold storage enables farmers to store their harvest and sell at higher prices during the extended selling period. Farmers who do not use cold storage sell all their apples within one month of harvesting, while farmers who use cold storage can choose the optimal time to sell until June of the following year, allowing for a complete sales cycle lasting up to eight months. The result of my surveys shows that farm-gate prices during the first one or two months after harvest tend to be lower throughout the farming year due to the significant instantaneous supply. However, it remains unclear whether the higher farm-gate prices observed during the first post-harvest period reflect a monopsony market environment, which needs further exploration.
    
    \item \textbf{More Sales Channels}: The adoption of cold storage has expanded farmers' sales channels. It allows them to sell their harvest not only to the traders who frequently visit in late October or November and offer lower prices (ranging from 4 to 9 yuan/kg), but also to consumers through E-commerce platforms like WeChat and Taobao, where they can ask for higher prices (ranging from 10 to 14 yuan/kg). Consequently, the use of cold storage empowers apple growers to decide whether to sell in bulk or retail, depending on prevailing market conditions.
\end{itemize}


%------------------------------------------------------%
\subsection{Low Storage Adoption}
\noindent The adoption of cold storage among apple growers remains low. Although most of our survey respondents have the option to either rent large cold storage or build their small-scale storage, more than 85\% of them choose not to do so for the following reasons:
\begin{itemize}
    \item \textbf{Unbalanced Scale of Cold Storage}: Most of the available collective cold storage facilities are too large (over 100 tons), resulting in a low individual harvest-to-storage-volume (HSV) ratio, which leads to high costs for cold storage adoption. The storage capacity of cold storage in many villages and towns is excessively large, often hundreds or thousands of tons. To utilize cold storage, many farmers need to store their produce collectively, a decision-making process that is difficult to scale in the village, resulting in underutilized cold storage facilities. 
    
    However, I found that small cold storage units (approximately 20 to 40 tons) built by villagers have a higher utilization rate. Farmers only need to store their own harvest, without having to cooperate with others. As soon as one household in the village benefits from selling at a higher price, many other farmers follow suit, creating a positive effect. Nevertheless, due to low output, most individual apple farmers cannot fully leverage the advantages of cold storage capacity, which leads to doubts about the feasibility and cost-effectiveness of investing in cold storage.

    \item \textbf{High Transportation Loss and Cost of Usage}: When the cold storage is far from the apple orchard, transportation in the mountainous terrain becomes inconvenient, resulting in significant transportation losses. Additionally, the cost of using cold storage is relatively high. Renting either collective cold storage or storage owned by others in the village incurs storage fees ranging from 0.20 to 0.40 yuan per kilogram, regardless of the duration of storage. Considering that the price of apples per kilogram has fluctuated between 2.5 and 4.5 in recent years, the cost of using cold storage is not insignificant. In contrast, when using small self-built cold storage (with a capacity of 20 tons), the maintenance cost ranges from 250 to 350 yuan per month.
    
    If a farmer's annual apple output is 10,000 kilograms and they need to store them for three months to secure a higher price offer, the cost of renting cold storage would range from 2,000 to 4,000 yuan, while the maintenance fee for their storage would range from 750 to 1,050 yuan. The former is two to three times higher than the latter.

    \item \textbf{High Construction Investment}: The total cost of a self-built small cold storage with a 20-ton capacity is approximately 90,000 yuan, with government subsidies ranging from 40,000 to 50,000 yuan. However, some farmers are burdened with debts and cannot afford to invest more than 40,000 yuan in construction.

    \item \textbf{Risk Aversion}: To convert harvested fruits into profits quickly, many extreme risk-averse farmers opt to sell their produce at lower prices early in the season to reduce uncertainty, rather than relying on cold storage and future market conditions. During the harvest season in October, many fruit dealers enter rural areas to make direct purchases. Among farmers, obtaining income directly during the harvest season is often considered the most straightforward approach. It's worth noting that among the surveyed farmers, those with higher levels of education tend to delay selling apples, which aligns with cold storage ownership and risk preferences.
\end{itemize}

However, at the same time, Yanchang is still experiencing a shortage in the supply of storage facilities. Cold storage facilities across the county are consistently full each year. In other words, the overall storage capacity currently falls short of demand. Consequently, rental storage fees have increased from 0.4 yuan/kg last year to 0.5 yuan/kg this year. The inventory of large merchants' cold storage is complex, with about 50\% being the owners' own apples, and the remaining apples are mostly from small fruit merchants and farmers stored in their facilities.

In summary, the impact of cold storage adoption on farmers' welfare is closely related to the storage's "relative" capacity, i.e., the harvest-to-storage-volume (HSV) ratio. If apple production is sufficiently large to fill the entire cold storage, utilization rates are high, and farmers can choose to store their harvests with confidence, anticipating higher prices at the end of the year or even during the Spring Festival. In such cases, cold storage adoption tends to benefit farmers the most.

Conversely, when apple production is low compared to the cold storage scale, cold storage is more likely to be abandoned. Farmers believe that the profits from later sales may not cover the sunk costs, communication expenses, and losses associated with using cold storage and transporting apples. Consequently, they are more inclined to sell their produce promptly during the harvest season. However, our on-site survey indicates that farmers with access to cold storage have better bargaining power than those who do not. In this scenario, access to storage serves as a deterrent rather than being used actively, as traders often possess better storage facilities than farmers. Storage is not primarily used to preserve products but rather as a signal or implicit threat to compel traders to offer more competitive prices.