\section{Alternative Conceptual Framework: Linear Mean-Variance Utility} \label{Appendix: mean-variance approach}
\noindent  We modeled farmer storage decisions using a CRRA utility specification in the main text. Here we provide an alternative formulation based on a mean-variance (MV) decision rule \citep{levy1979approximating, schoemaker1982expected}. The principal advantage of the MV framework is that it avoids assuming any specific utility function, thereby abstracting away from the functional form of farmers' risk preferences \citep{coyle1992risk}. This allows for greater transparency in tracing how storage incentives respond to changes in price levels, volatility, and market structure. The disadvantage of the MV framework, as noted in the main text, is a paucity of empirical works to calibrate its parameters relative to what is available for the CRRA framework.

MV has been adopted in many works in agricultural economics such as \cite{saitone2009optimal} and \cite{yu2018effects}. Given the inter-temporal price formation rule from the FOOM model, the expected second-period revenue is not an affine transformation of the stochastic buyer power change. Hence the MV model is not always consistent with expected utility maximization \citep{meyer1987two}, although a careful choice from a mean-variance efficient frontier will approximate expected utility maximization \citep{markowitz2014mean, chiu2016supply}.


We retain the same model structure as in the main text. Each farmer harvests one unit in period 1 and chooses a storage share $s\in[0,1]$, selling $1-s$ units at harvest at the certain price $p_{1}$, storing $s$ units until period 2, and incurring an encompassing storage cost represented by $\kappa \le 1$ that reduces net period 2 price. Future market conditions are summarized by an ex-ante random buyer-power index $\theta_2$. The second-period price retains the compact form in Equation~(\ref{Eq: price formation by buyer power}), $p_{2}=\frac{1}{1+\theta_{2}},$ where $\theta_{2}$ has mean $\mu_{\theta_2}$ and variance $\sigma_{\theta_2}^2$. Total discounted revenues are therefore $\pi(s) = (1-s)p_{1} + \kappa s\, p_{2}.$

The MV method is the quadratic approximation to any twice-differentiable utility around its certainty equivalent:
$$
\max_{s\in[0,1]}
\;\;{\cal V}(s)
\equiv
\mathbb E[\pi(s)]
-\frac{\xi}{2}\operatorname{Var}[\pi(s)],
\quad \xi>0.
$$
where $\xi$ indexes the marginal rate of substitution between expected profit and profit risk; it maps one-to-one to the constant coefficient of absolute risk aversion (CARA) in the underlying utility function.

\subsection{Moments of Future Price}
\noindent
We apply the Delta Method and use $f(x)=1/(1+x)$. Evaluating at the mean $\mu_{\theta_2}$ and retaining terms up to second order:

$$
\begin{aligned}
\mathbb E[ p_{2}]
&\approx f(\mu_{\theta_2})+\tfrac12 f''(\mu_{\theta_2})\,\sigma_{\theta_2}^{2}
= \frac{1}{1+\mu_{\theta_2}}
  +\frac{\sigma_{\theta_2}^{2}}{(1+\mu_{\theta_2})^{3}},\\[6pt]
\operatorname{Var}[ p_{2}]
&\approx\bigl[f'(\mu_{\theta_2})\bigr]^{2}\sigma_{\theta_2}^{2}
= \frac{\sigma_{\theta_2}^{2}}{(1+\mu_{\theta_2})^{4}}.
\end{aligned}
$$
Define the approximated mean $A$ and variance $B$ of $p_2$: 
$$
A \equiv \frac{1}{1+\mu_{\theta_2}} + \frac{\sigma_{\theta_2}^2}{(1+\mu_{\theta_2})^3},\qquad
B \equiv \frac{\sigma_{\theta_2}^2}{(1+\mu_{\theta_2})^4}.
$$

\subsection{Optimal Storage Choice}
\noindent
Plugging the moments into ${\cal V}(s)$,
$$
\mathbb E[\pi(s)]       =(1-s)p_{1}+\kappa sA,
\qquad
\operatorname{Var}[\pi(s)]=\kappa^{2}s^{2}B.
$$
The first-order condition for an interior optimum ($0<s_{int}<1$) is:
$$
\frac{\partial{\cal V}}{\partial s}
= -p_{1}+\kappa A - \xi\kappa^{2}sB = 0
\quad\Rightarrow\quad
s_{int} = \frac{\kappa A - p_1}{\xi \kappa^2 B}.
$$
Imposing feasibility yields the decision rule: $s^{*} = \min\left\{1,\;\max\left\{0,\,s_{int}\right\}\right\}.$


\paragraph{Interior solution.}  
If $0<s^*<1$, the first-order condition yields
$$
s^{*}
= \frac{\kappa A - p_{1}}{\xi\kappa^{2}B}
= \frac{\Delta}{\xi\kappa^{2}B}.
$$

\paragraph{Corner conditions.}  
Evaluating the derivative at the bounds,
$$
\left.\frac{\partial\mathcal V}{\partial s}\right|_{s=0}
= \Delta,
\qquad
\left.\frac{\partial\mathcal V}{\partial s}\right|_{s=1}
= \Delta - \xi\kappa^{2}B.
$$
Thus,
$$
s^{*} =
\begin{cases}
0, & \kappa A \le p_{1},\\[4pt]
\dfrac{\kappa A - p_{1}}{\xi\kappa^{2}B}, 
& p_{1} < \kappa A < p_{1} + \xi\kappa^{2}B,\\[10pt]
1, & \kappa A \ge p_{1} + \xi\kappa^{2}B.
\end{cases}
$$


The decision rule is intuitive.  
\begin{itemize}
    \item \textbf{No storage:} $\kappa\,\mathbb E[p_{2}] \le p_{1}$.  
    The expected discounted future price is below the certain harvest price.
    \item \textbf{Full storage:} 
    $\kappa\,\mathbb E[p_{2}]
    - \xi\kappa^{2}\operatorname{Var}(p_{2}) 
    \ge p_{1}$.
   
    The risk-adjusted expected future price exceeds $p_{1}$.
    \item \textbf{Partial storage:} arises when the expected price premium is positive but not large enough to dominate the risk premium.
\end{itemize}

These comparative statics for the MV mirror those under CRRA utility: $s^*$ is non-decreasing in storage efficiency $\kappa$ and higher expected period 2 price, and non-increasing in $p_1$ and greater risk aversion $\xi$. This alignment follows because the MV framework corresponds to CARA preferences, whose qualitative responses are consistent with CRRA in our setting.

