\section{Conceptual Framework: Linear Mean-Variance Approach} \label{Appendix: mean-variance approach}
\noindent In the main text, I model farmer storage decisions under market-structural uncertainty using a CRRA (constant relative risk aversion) utility specification, which is widely adopted in agricultural economics and allows for a structurally interpretable mapping between risk preferences and observed behavior. This choice facilitates empirical calibration, as the CRRA form nests intuitive behavioral responses and aligns well with micro-level data on consumption and risk. Crucially, it maintains coherence between the theoretical framework and the empirical analysis, which relies on price expectations and storage behavior inferred from actual market data. Since there is limited empirical guidance on how to calibrate abstract risk aversion coefficients across different agricultural contexts, preserving this consistency helps ensure that the model remains empirically grounded and interpretable.

Nonetheless, in the Appendix here, I provide an alternative formulation based on a mean-variance (MV) decision rule as a complement \citep{levy1979approximating, schoemaker1982expected}. The principal advantage of the MV framework is that it avoids assuming any specific utility function, thereby abstracting away from the functional form of farmers' risk preferences \citep{coyle1992risk}. This allows for greater transparency in tracing how storage incentives respond to changes in price levels, volatility, and market structure. MV has been adopted in many works in the field of agricultural economics such as \cite{saitone2009optimal} and \cite{yu2018effects}. While my inter-temporal price rule formation from the \textit{FOOM} models (the expected second-period revenue is not an affine transformation of the stochastic buyer power change), hence the MV model is not always consistent with expected utility maximization \citep{meyer1987two}, a careful choice from a mean-variance efficient frontier will approximately maximize expected utility \citep{markowitz2014mean, chiu2016supply}.




\subsection{Economic Environment}
\noindent
A representative farmer harvests a normalized quantity of one unit at the beginning of period 1 and chooses a \textit{storage share} \(s\in[0,1]\). The farmer can (i) sell \((1-s)\) units immediately at the certain price \(p_{1}\), or (ii) store \(s\) units until period 2, incurring a proportional storage-cum-discount factor \(\delta\in(0,1]\). Future market conditions are summarized by an ex-ante random \textit{buyer-power index} \(\tilde\theta_{2}\). Conditional on \(\tilde\theta_{2}\), the second-period price is generated by the structural rule:
$$
\tilde p_{2} = \frac{1}{1+\tilde\theta_{2}}, \qquad 
\tilde\theta_{2} \sim (\mu_\theta, \sigma_\theta^{2}).
$$
Total discounted revenues are therefore:
$$
\pi(s) = (1-s)p_{1} + \delta s\,\tilde p_{2}.
$$

\subsection{Preference Representation}
\noindent
Departing from expected-utility maximization, I adopt the standard \textit{mean-variance} (two-moment) method, which is the quadratic approximation to any twice-differentiable utility around its certainty equivalent:
$$
\max_{s\in[0,1]}
\;\;{\cal V}(s)
\equiv
\mathbb E[\pi(s)]
-\frac{\kappa}{2}\operatorname{Var}[\pi(s)],
\quad\kappa>0.
$$
The parameter \(\kappa\) indexes the marginal rate of substitution between expected profit and profit risk; it maps one-to-one to the coefficient of (local) absolute risk aversion in the underlying utility function.

\subsection{Moments of Future Price (Delta Method)}
\noindent
Let $f(x)=1/(1+x)$. Evaluating at the mean $\mu_\theta$ and retaining terms up to second order:

$$
\begin{aligned}
\mathbb E[\tilde p_{2}]
&\approx f(\mu_\theta)+\tfrac12 f''(\mu_\theta)\,\sigma_\theta^{2}
= \frac{1}{1+\mu_\theta}
  +\frac{\sigma_\theta^{2}}{(1+\mu_\theta)^{3}},\\[6pt]
\operatorname{Var}[\tilde p_{2}]
&\approx\bigl[f'(\mu_\theta)\bigr]^{2}\sigma_\theta^{2}
= \frac{\sigma_\theta^{2}}{(1+\mu_\theta)^{4}}.
\end{aligned}
$$
Define:
$$
A \equiv \frac{1}{1+\mu_\theta} + \frac{\sigma_\theta^2}{(1+\mu_\theta)^3},\qquad
B \equiv \frac{\sigma_\theta^2}{(1+\mu_\theta)^4}.
$$

\subsection{Optimal Storage Choice}
\noindent
Plugging the moments into \({\cal V}(s)\),
$$
\mathbb E[\pi(s)]       =(1-s)p_{1}+\delta sA,
\qquad
\operatorname{Var}[\pi(s)]=\delta^{2}s^{2}B.
$$
First-order condition for an interior optimum (\(0<s<1\)):
$$
\frac{\partial{\cal V}}{\partial s}
= -p_{1}+\delta A - \kappa\delta^{2}sB = 0
\quad\Rightarrow\quad
s^{\star} = \frac{\delta A - p_1}{\kappa \delta^2 B}.
$$
Imposing feasibility yields the policy rule:
$$
s^{\star}_\text{M--V} = \min\left\{1,\;\max\left\{0,\,s^{\star}\right\}\right\}.
$$

\subsection{Comparative Statics}
\noindent
I now characterize how the optimal storage share \(s^{\star}\) responds to changes in each underlying parameter, focusing on marginal effects within the interior solution region. The closed-form expression
$$
s^{\star} = \frac{\delta A - p_1}{\kappa \delta^2 B},
\qquad
A = \frac{1}{1 + \mu_\theta} + \frac{\sigma_\theta^2}{(1 + \mu_\theta)^3},
\quad
B = \frac{\sigma_\theta^2}{(1 + \mu_\theta)^4},
$$
implies that signs and magnitudes of changes depend critically on how each parameter affects the numerator (expected gain from waiting) and the denominator (penalty for price risk).

\paragraph{Current spot price \(p_1\).}
Differentiating \(s^{\star}\) with respect to \(p_1\) yields:
$$
\frac{\partial s^{\star}}{\partial p_1} = -\frac{1}{\kappa \delta^2 B} < 0.
$$
This effect is linear and unambiguous: a higher harvest-season price reduces the attractiveness of waiting by increasing the opportunity cost of storage. Since the second-period price is uncertain, selling now at a known higher price becomes more appealing. The strength of this effect is attenuated by greater risk aversion \(\kappa\), higher variance \(B\), or greater storage discounting \(\delta^2\).

\paragraph{Discount/storage factor \(\delta\).}
The derivative with respect to \(\delta\) is:
$$
\frac{\partial s^{\star}}{\partial \delta}
=
\frac{A - 2p_1/\delta}{\kappa \delta B}
=
\frac{(2p_1/\delta) - A}{\kappa \delta^2 B}.
$$
The sign of this expression depends on the relative size of \(\delta A\) and \(2p_1\). If \(p_1 < \delta A / 2\), then \(\partial s^{\star}/\partial \delta > 0\), implying that a more favorable storage environment (e.g., through lower physical loss or discounting) increases the incentive to store. Conversely, when \(p_1\) is relatively high, further increases in \(\delta\) make variance penalties more salient (via the \(\delta^2\) term in the denominator), potentially decreasing storage.

\paragraph{Risk aversion \(\kappa\).}
Storage declines monotonically with risk aversion:
$$
\frac{\partial s^{\star}}{\partial \kappa}
=
-\frac{s^{\star}}{\kappa} < 0.
$$
A more risk-averse farmer places a higher penalty on variance and therefore chooses to market a larger share of the harvest at the safe, known price \(p_1\). The strength of this effect is proportional to the initial level of storage \(s^{\star}\) and inversely proportional to \(\kappa\) itself.

\paragraph{Mean buyer power \(\mu_\theta\).}
Changes in \(\mu_\theta\) influence both the expected price (\(A\)) and its variance (\(B\)). Applying the chain rule:
$$
\frac{\partial A}{\partial \mu_\theta} = -\frac{1}{(1 + \mu_\theta)^2} - \frac{3\sigma_\theta^2}{(1 + \mu_\theta)^4} < 0,\qquad
\frac{\partial B}{\partial \mu_\theta} = -\frac{4\sigma_\theta^2}{(1 + \mu_\theta)^5} < 0.
$$
An increase in the average level of buyer power lowers both the expected future price and its variance. However, the effect on \(s^{\star}\) is dominated by the decline in \(A\), which reduces the expected gain from deferring sales. Since the price variance \(B\) also falls, the risk-adjusted reward from waiting shrinks. Therefore:
$$
\frac{\partial s^{\star}}{\partial \mu_\theta} < 0,
$$
i.e., more monopsonistic market conditions on average lead farmers to store less.

\paragraph{Variance of buyer power \(\sigma_\theta^2\).}
This parameter enters both the numerator and denominator of \(s^{\star}\), yielding a nontrivial tradeoff between \textit{convexity gains} (via Jensen's inequality) and \textit{risk penalties}. Substituting \(A = A_0 + C\sigma_\theta^2\), \(B = D\sigma_\theta^2\), with \(A_0 = 1/(1 + \mu_\theta)\), \(C = 1/(1 + \mu_\theta)^3\), and \(D = 1/(1 + \mu_\theta)^4\), the derivative becomes:
$$
\frac{\partial s^{\star}}{\partial \sigma_\theta^2}
=
\frac{\delta A_0 - p_1}{\kappa \delta^2 D \sigma_\theta^4}.
$$
Hence,
$$
\text{sign}\left( \frac{\partial s^{\star}}{\partial \sigma_\theta^2} \right)
= \text{sign}\left( \delta A_0 - p_1 \right).
$$
When the expected future price (without Jensen's correction) already exceeds the current price, increases in variance enhance expected revenues and encourage storage. Otherwise, the added risk dominates, and storage is reduced. This reflects a \textit{threshold property}: variance is beneficial only when the price gap \(\delta/(1 + \mu_\theta) - p_1\) is positive.

\paragraph{Summary Table: Signs of Marginal Effects on Interior Storage Share \(s^{\star}\)}

\begin{table}[H]
\centering
\begin{tabular}{llll}
\toprule
\textbf{Parameter} & \(\partial s^{\star}/\partial(\cdot)\) & \textbf{Direction} & \textbf{Mechanism} \\
\midrule
\(p_1\) & \(< 0\) & \(\downarrow\) & Higher current price discourages storage \\
\(\delta\) & Mixed & \(\uparrow\) if \(\delta A > 2p_1\) & Storage gain vs. variance amplification \\
\(\kappa\) & \(< 0\) & \(\downarrow\) & More risk aversion leads to earlier selling \\
\(\mu_\theta\) & \(< 0\) & \(\downarrow\) & More monopsony power depresses future price \\
\(\sigma_\theta^2\) & \(\text{sign}(\delta/(1+\mu_\theta) - p_1)\) & \(\uparrow\) or \(\downarrow\) & Jensen gain vs. added risk \\
\bottomrule
\end{tabular}
\end{table}

These results highlight that while some comparative statics are globally monotonic, others depend on \textit{threshold conditions} shaped by the relative price gap between harvest and storage seasons.
