\newpage
\section{Alternative Conceptual Framework}

Let's denote the following:
\begin{align*}
    & P_r: \text{farmer's reservation price} \\
    & P_y: \text{marginal Value Product of Middlemen} \\
    & P_f: \text{farm-gate price of the transaction} 
\end{align*}
Therefore, middlemen's surplus would be $(P_y-P_f)$ and farmers' surplus would be $(P_f-P_r)$. To capture the asymmetry in bargaining power, this model applies the Nash Bargaining model by \cite{binmore1986nash}. The product of their surplus is $(P_y-P_f)^{1-\theta}(P_f-P_r)^{\theta}$, where $\theta$ is the bargaining power of farmers. Thus, under maximization, we have
$$
P_f(\theta) = P_r(1-\theta) + P_y\theta
$$
where $\theta=1$ represents the case where traders perfectly compete and $\theta=0$ depicts monopsony. If we normalize $P_r\equiv 0$, then $P_f(\theta) = P_y \theta$.


Choice variables in the model include $w_2$ the quantity stored in the first period and sold in the second period after the harvest and $t$ the duration between two transaction periods. I write the model as follows:
\begin{equation}
    \max_{w_t,t} E[u(\pi_0|\alpha_i) + \beta_i(t) u(\pi_t|\alpha_i)]
\end{equation}
\begin{align*}
    & \pi_0= P_{(f,0)} w_0 - c(t,w_2), & \text {(net revenue from 1st period)}\\
    & \pi_t= \Tilde{P}_{(f,t)} w_t \epsilon(t), & \text {(net revenue from 2nd period)} \\
    & W = w_0 + w_t, & \text {(harvest constraint)}\\
    & P_{f,t} = P_{y,t}\theta, & \text {(NB with normalized reservation price)}
\end{align*}
where $\alpha_i$ is the coefficient of risk aversion and $\beta_i(t)$ is the discount factor for farmer $i$; $P_{f,t}$ indicates the farm-gate price of the storable good at time t; $\epsilon(t)$ denotes the deterioration rate; $w_t$ represents the quantity sold at time t and $W$ takes the amount of the harvest as given; $c(t,w_t)$ is the variable storage cost function whose form depends on the type of storage adopted.

\begin{equation}
c\left(t, w_2\right)=
                \begin{cases}
                c_1 t, & \text {when using self-built storage} \\ 
                c_2 w_t^2, & \text {when renting collective storage}
                \end{cases}
\end{equation}

The relative bargaining positions are closely correlated with the instantaneous supply, hence we assume that $\theta$ in each period is a function of the farmer's storage decision, i.e. $\theta=\theta(w_t,t)$. 