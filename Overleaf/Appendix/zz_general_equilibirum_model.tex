\newpage
\section{General Equilibrium Framework}
My modelling framework would be a scenario as follows. There are multiple villages consisting of farmers. Each farmer receives a price offer and can choose to accept it or wait for another offer at a storage cost. Some farmers are required to sell immediately in each period. Better storage refers to a situation where a smaller fraction of farmers are forced to sell immediately. Instead of random price shocks, traders now face random demand shocks. Every period, each trader visits one village and makes a universal price offer in this village. Their offers are accepted by farmers who must sell and by farmers with storage whose reservation price is lower than the offer. Traders offer higher prices to villages with better storage to prevent market failure, as immediate supply is lower and reservation prices are higher there, as long as the price increment is not more than double their transportation cost.

At the expected general equilibrium, as long as traders have better storage facilities than farmers, on-farm Storage would serve as a deterrent (like a nuclear weapon), rather than really being used. Even if a village has better storage facilities, farmers would still sell their products in the first stage after the harvest. Storage is not used to actually preserve the products but rather as a signal or implicit threat to compel traders to be more competitive and offer higher prices.