\newpage
\section{Competing Mechanism and Confounding Factors}
A potential alternative mechanism arises from the "theory of the second best," suggesting that if an economy is not functioning at its most efficient state due to market failure, addressing only one aspect of the problem (in this case, introducing on-farm storage) may not yield the anticipated outcomes. Other factors within the system might also need attention.

In my research design, it is plausible that if the majority of farmers in a village adopt on-farm storage, traders and middlemen may redirect their focus to other villages where farmers have not yet embraced this technology. This preference could stem from traders and middlemen favoring farmers with less bargaining power, as they could negotiate lower prices for the crops they purchase.

Such a scenario could lead to a decline in local demand for the crops produced by farmers in the village that has implemented on-farm storage. Consequently, this decrease in demand might result in lower prices for these farmers, potentially offsetting any benefits they derived from adopting the quality-preserving technology.

Furthermore, several confounding factors could influence the outcomes of interest, making it challenging to establish a causal relationship between the adoption of on-farm storage and farmers' welfare. To address these confounding factors, I would consider the following:
\begin{itemize}
    \item Usage Practices of on-farm storage: Merely having access to on-farm storage does not guarantee proper or efficient utilization. Farmers may lack knowledge regarding optimal storage conditions for different crop stages, leading to spoilage or reduced quality. Additionally, financial constraints or limited technical know-how could hinder investment and maintenance of on-farm storage facilities.
    
    \item Impacts of Subsidy Policies: Government policies and subsidies related to storage can alter the effects of on-farm storage on farmers' welfare. In many developing countries, the government provides substantial subsidies, encouraging farmers to establish collective storage infrastructure for the entire village or directing them to sell their crops to large distribution centers. Consequently, farmers may not experience net benefits from individual on-farm storage since they receive higher cash transfers or profits from subsidized alternatives. 

    \item Regional Differences: The benefits of on-farm storage may vary depending on the region where farmers operate. Factors such as weather patterns, local demand and supply conditions, and access to transportation infrastructure can affect the advantages of on-farm storage. To account for this, I would incorporate a village-level location fixed effect in the empirical analysis.

    \item Differences in Farming Practices: Variances in cultivation and harvesting practices among farmers can impact the outcomes of interest. For instance, farmers may employ different fertilizers or irrigation techniques, influencing crop quality. To address this, I would include quality-related questions in the survey and leverage crops with standardized commodity markets, such as futures markets, to mitigate the impact of farming practices in the regression analysis.

    \item Credit Constraints and Present Bias: Providing on-farm storage to credit-constrained farmers with high present bias may not yield significant benefits. These farmers tend to immediately exchange their produce for cash to cover immediate expenses, rather than delaying sales for potentially higher overall profits. To account for this, I would incorporate well-crafted questions in the survey to gauge the extent of present bias and control for this factor.


\end{itemize}





