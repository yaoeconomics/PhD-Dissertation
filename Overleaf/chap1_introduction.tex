

\noindent
Buyer power has long been a central concern in agricultural economics, yet much of the literature treats it as a \textit{static} phenomenon---fixed, predictable, and structurally embedded. Classical and neoclassical models, rooted in industrial organization theory, often portray buyer power as a durable feature of market structure, shaped by concentration ratios, economies of scale, or long-term strategic behavior among firms \citep{sexton2013market}. In these frameworks, power is held by entities that are large, formally organized, and largely immune to short-run shocks or spatial variability \citep{sexton2010grocery}.

However, this perspective is increasingly insufficient, especially in the context of agricultural procurement systems in developing economies. These systems are defined not by structural rigidity, but by institutional informality, infrastructural gaps, and temporal volatility \citep{barrett2008agricultural}. Smallholder farmers in these settings interact with procurement markets that are seasonal, fragmented, and frequently unstable \citep{wang2020transaction}. Trader entry, buyer behavior, and market concentration fluctuate across space and time \citep{fafchamps2005selling, sitko2014exploitative, bergquist_dinerstein_2020}. As a result, the practical distribution of market power is both diffuse and dynamic.

This dissertation proposes a fundamentally different lens: market power in agricultural procurement is not merely a function of concentration, it is also dynamic, time-varying, and deeply contingent on local and temporal conditions \citep{barrett2022agri}. Drawing on rich empirical evidence from China's fresh apple industry---the world's largest by production volume---I explore how buyer competition fluctuates intra-seasonally; how farmers respond to this volatility through strategic use of cold storage; and how storage access, in turn, can reshape the very structure of buyer power. At the heart of this inquiry is a novel argument: storage is not only a vehicle for price arbitrage \citep{burke2019sell, wright1982econ}, but also a strategic tool for intertemporal bargaining amid a dynamic environment for market competition.



\paragraph{The Fragility and Fluidity of Buyer Power.}
The common belief that more buyers equate to more competition is challenged by detailed fieldwork in Yanchang County, Shaanxi Province. In this setting, oligopsonistic behavior is widespread, yet highly unstable. Farmers encounter procurement environments that change from month to month, and even week to week, with no clear patterns of predictability. This time-variability of buyer power stems from three interrelated economic mechanisms:
\begin{enumerate}
    \item \textbf{The Fragile Nature of Collusion}: Informal buyer cartels often emerge in local markets, particularly during the narrow window of post-harvest trade. These interest groups rely on tacit collusion, facilitated by WeChat groups, repeat interactions, and informal market territories, to suppress competition and discipline prices \citep{macchiavello2015value}. However, these arrangements are inherently fragile. They can collapse under the weight of defection, local disputes, or price pressures from downstream buyers. The absence of formal enforcement mechanisms makes them vulnerable to short-run shocks and opportunistic behavior.

    \item \textbf{Capacity Constraints and Randomized Access}: Traders operate under tight logistical and capital constraints \citep{ambler2023finance}. Their truck fleets, cold storage facilities, and downstream distribution agreements are limited, and not all procurement opportunities are equally viable. As a result, trader entry into any given village to procure product is somewhat stochastic. Some villages may see a glut of buyers competing vigorously, while others may be visited by only one, creating effective monopsony conditions. From the farmer's perspective, this randomness translates into a procurement environment that is unpredictable and unbalanced in terms of competitive conditions.

    \item \textbf{Territoriality and the Threat of Outside Entry}: Many local trader networks establish de facto procurement territories through informal agreements to avoid internal competition. However, these agreements are fragile. Non-local traders, who often arrive with fewer relational constraints and larger operational scales, pose a constant threat of disruption. Their entry can undermine local collusion, drive price competition, and redistribute bargaining power \citep{BARTKUS2022}. Yet their presence is also highly seasonal and uneven, adding another layer of temporal volatility to the procurement structure.
\end{enumerate}

\noindent
Together, these mechanisms generate a procurement landscape where market power is not a fixed feature of the supply chain, but a dynamic phenomenon. Villages may move into and out of competitive and collusive phases. Buyers switch roles between colluding partners and price competitors. Farmers, in turn, must navigate this evolving terrain with limited information and weak institutional support.

This dissertation is the first to formally model and empirically document this dynamic, and to consider its implications not just for price formation, but for the intertemporal strategies that farmers equipped with storage can use to benefit in the face of uncertainty.


\paragraph{Storage as a Strategic Bridge Across Competitive Periods.}
In agricultural markets for storable or semi-storable commodities, the temporal variability of buyer competition creates a distinctive incentive structure. Unlike perishable crops, which must be sold within days of harvest, storable crops like apples enable farmers who can access storage to choose when to sell. This choice mediated by storage opens up new strategic pathways.

Farmers who store in essence obtain multiple ``draws'' from the distribution of competitive market conditions over time. If the immediate post-harvest period is characterized by collusion or monopsony, storage allows farmers to delay sales in expectation of obtaining a more favorable buyer environment in the future.

In this sense, storage is not merely an inventory management tool \citep{goyal2010information}, it is a mechanism for intertemporal bargaining. It creates a form of market power for the farmer, shifting the dynamics of the negotiation. Once product is in storage, the immediate imperative to sell disappears. The seller gains time, options, and leverage. Buyers, in contrast, must compete in a broader, more centralized marketplace.

Indeed, cold storage facilities often function as de facto terminal markets. They aggregate supply from multiple growers, attract buyers from broader regions, and facilitate more transparent and competitive pricing. Traders converge on storage hubs rather than individual villages, and in doing so, must compete more directly with one another. This reshapes the power dynamics of the supply chain and creates a temporary re-balancing in favor of producers, especially those with access to high-quality, low-cost storage. The very act of storing thus transforms not only the timing of transactions, but also the structure of the procurement market itself.


\paragraph{Intertemporal Marketing and the New Storage Economy.}
By jointly analyzing storage decisions and buyer competition, this dissertation challenges conventional wisdom about farmer behavior. The prevailing narrative holds that farmers store to capture expected price increases arising from seasonal demand shifts. While this is true in part, the field data reveal an alternative rationale: farmers may store to escape weak bargaining conditions, hedge against collusion, and access more competitive markets.

This reframing of the rationale for storage has several important implications:

\begin{itemize}
    \item \textbf{First}, it expands the welfare narrative around storage. Storage does not only enable price smoothing or loss reduction, it serves as a strategic tool to avoid monopsony pricing and access different competitive environments over time. 

    \item \textbf{Second}, it explains why inefficient or high-cost storage is widely used. Even when the financial returns from arbitrage are marginal or negative, the non-monetary value of avoiding exploitative procurement periods may still justify the decision. In other words, storage has a strategic payoff, even in the absence of direct profits \citep{suri2011selection}.

    \item \textbf{Third}, it underscores the institutional frictions that limit storage access---especially for poor or liquidity-constrained farmers. Since the gains from storage accrue disproportionately to those who can afford it, unequal access to storage may exacerbate rural inequality and entrench disadvantage. Conversely, well-designed interventions, such as cold storage subsidies or cooperative storage networks, can have disproportionately positive effects on farmer welfare and market efficiency.
\end{itemize}

\paragraph{A New Lens on Agricultural Market Power.}
This dissertation makes three core contributions to the study of agricultural market dynamics:

\begin{enumerate}
    \item \textbf{Conceptual}: It develops a two-period dynamic model of time-varying oligopsony in agricultural procurement, integrating temporal shifts in buyer competition and endogenous farmer storage decisions. The model departs from traditional static IO frameworks by incorporating time-vary oligopsonistic power across periods.

    \item \textbf{Empirical}: It draws on original field data, including over 500 interviews and surveys with farmers, traders, and storage operators, to document the temporal variability of market structure, the strategic logic of storage, and the welfare impacts of intertemporal marketing choices.

    \item \textbf{Policy Relevance}: It offers a novel economic justification for investments in cold chain infrastructure and market design. By reframing storage as a lever for redistributing bargaining power, the findings suggest that targeted storage subsidies, improvements in farmer access to commercial storage, and reductions in transaction frictions could serve as powerful tools for enhancing both efficiency and equity in agricultural markets.
\end{enumerate}


\paragraph{Structure of the Dissertation.}
The chapters that follow elaborate on these themes using both theoretical and empirical approaches.

\begin{itemize}
    \item \textbf{Chapter 2} provides an institutional and empirical overview of China's apple industry, with a focus on storage practices, upstream buyer structure, and the field-level evidence on collusion and entry.

    \item \textbf{Chapter 3} reviews the literature on storage economics, market power in agriculture, and smallholder marketing under uncertainty, identifying critical gaps this dissertation seeks to fill.

    \item \textbf{Chapter 4} presents a dynamic two-period model of time-varying oligopsony and storage adoption, analyzing both individual farmer decisions and aggregate welfare implications under different scenarios of buyer behavior and risk preferences.

    \item \textbf{Chapter 5} empirically tests key predictions using survey data and interviews from Yanchang County. It explores the determinants of storage adoption, the relationship between buyer competition and timing of sales, and the welfare impacts of storage under constrained competitive environments.

    \item The appendices contain mathematical derivations, additional robustness checks, and the survey instruments used in fieldwork.
\end{itemize}
