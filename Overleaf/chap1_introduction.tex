\noindent
Buyer power has long been a central concern in agricultural economics, yet much of the literature treats it as a \textit{static} phenomenon---fixed, predictable, and structurally embedded. Classical and neoclassical models, rooted in industrial organization theory, often portray buyer power as a durable feature of market structure, shaped by concentration ratios, economies of scale, or long-term strategic behavior among firms \citep{sexton2013market}. In these frameworks, power is held by entities that are large, formally organized, and largely immune to short-run shocks or spatial variability \citep{sexton2010grocery}.

This perspective is increasingly inadequate, especially in the context of agricultural procurement systems in developing economies. These systems are defined not by structural rigidity, but by institutional informality, infrastructural gaps, and temporal volatility \citep{barrett2008agricultural}. Smallholder farmers in these settings interact with procurement markets that are seasonal, fragmented, and frequently unstable \citep{wang2020transaction}. Trader entry, buyer behavior, and market concentration fluctuate across space and time \citep{fafchamps2005selling, sitko2014exploitative, bergquist_dinerstein_2020}, which is even true within a marketing season for some commodities. As a result, the practical distribution of market power is both diffuse and dynamic.


This dissertation develops and extends this alternative perspective on market dynamics to shed new light on farmers' storage behavior. It argues that buyer power in agricultural procurement is not fixed by concentration, but fluctuates with local and temporal market conditions \citep{barrett2022agri}. Using evidence from China's fresh-apple industry---the world's largest by output---I show how shifting buyer competition shapes farmers' marketing timing, how storage enables strategic responses to volatility, and how storage access feeds back to alter the distribution of buyer power. At its core, the dissertation frames storage not merely as intertemporal arbitrage \citep{burke2019sell, wright1982econ}, but as a strategic tool of bargaining within evolving procurement environments.



\paragraph{The Fragility and Fluidity of Buyer Power.}
The conventional belief that ``more buyers imply greater competition'' is challenged by detailed fieldwork in the upstream fresh-apple sector of Yanchang County, Shaanxi Province. According to official statistics, the apple industry constitutes the backbone of Yanchang's agriculture: apples account for over 55\% of the county's total agricultural output \citep{cfa_2025_yanchang_apple_futures}, and apple farmers comprise roughly 65\% of its agricultural population \citep{agri_2023_yanchang_apple}. My field evidence further indicates that for most farmers, more than 90\% of annual income derives from fresh apples, when the traders (buyers) they transact with engage almost exclusively in apple trading aside from limited off-farm employment.

A defining feature of the apple market is the ability to extend the marketing window---from roughly one month post-harvest to over six months---with the use of convention cold storage.\footnote{Cold and Controlled Atmosphere storage can further reduce the perishability of fresh apples by slowing respiration and ethylene production, thereby extending marketable shelf life to nearly a year under optimal conditions.} This technology transforms a single-period sale decision into a dynamic, intra-seasonal bargaining process that unfolds from October through the following July. Within this setting, oligopsonistic behavior is both pervasive and unstable. Farmers face procurement environments that shift month by month, and sometimes even week by week, without a predictable pattern. This temporal variability of buyer power arises from three interrelated economic mechanisms:
\begin{enumerate}
    \item \textbf{The Fragile Nature of Collusion}: I show in this study that informal buyer cartels often emerge in local markets, particularly during the narrow window of post-harvest trade. These interest groups rely on tacit collusion, facilitated by ``WeChat'' groups, repeat interactions, and informal market territories, to suppress competition and discipline prices. However, these arrangements are inherently fragile. They can collapse under the weight of defection, local disputes, or price pressures from downstream buyers. The absence of formal enforcement mechanisms makes them vulnerable to short-run shocks and opportunistic behavior.

    \item \textbf{Capacity Constraints and Randomized Access}: As \cite{ambler2023finance} argue that traders operate under tight logistical and capital constraints, my fieldwork indicates that buyers' truck fleets, cold storage facilities, and downstream distribution agreements are limited, and not all procurement opportunities are equally viable. As a result, trader entry into any given village to procure product is somewhat stochastic. Some villages may see a glut of buyers competing vigorously, while others may be visited by only one, creating effective monopsony conditions. From the farmer's perspective, this randomness translates into a procurement environment that is unpredictable and unbalanced in terms of competitive conditions.

    \item \textbf{Territoriality and the Threat of Outside Entry}: Many local trader networks establish de facto procurement territories through informal agreements to avoid internal competition in Yanchang county. However, these agreements are fragile. Non-local traders, who often arrive with fewer relational constraints and larger operational scales, pose a constant threat of disruption. Their entry can undermine local collusion, drive price competition, and redistribute bargaining power \citep{BARTKUS2022}. Yet their presence is also highly seasonal and uneven, adding another layer of temporal volatility to the procurement structure.
\end{enumerate}

\noindent
Together, these mechanisms generate a procurement landscape where market power is not a fixed feature of the supply chain, but a dynamic phenomenon. Villages may move into and out of competitive and collusive phases. Buyers switch roles between colluding partners and price competitors. Farmers, in turn, must navigate this evolving terrain with limited information and weak institutional support.

This dissertation is the first to formally model and empirically document this dynamic, and to consider its implications not just for price formation, but for the intertemporal strategies that farmers equipped with storage can use to benefit in the face of uncertainty.


\paragraph{Storage as a Strategic Bridge Across Competitive Periods.}
In agricultural markets for storable or semi-storable commodities, the temporal variability of buyer competition creates a distinctive incentive structure. Unlike perishable crops, which must be sold within days of harvest, storable crops like apples enable farmers who can access storage to choose when to sell. This choice mediated by storage opens up new strategic pathways.

Apple farmers who store in essence obtain multiple ``draws'' from the distribution of competitive market conditions over time. If the immediate post-harvest period is characterized by collusion or monopsony, storage allows farmers to delay sales in expectation of obtaining a more favorable buyer environment in the future.

In this sense, storage is not merely an inventory management tool \citep{goyal2010information}, it can become a mechanism for intertemporal bargaining. It creates a form of market power for the farmer, shifting the dynamics of the negotiation. Once product is in storage, the farmers' immediate imperative to sell disappears. They gains time, options, and leverage. 

Buyers, in contrast, face a broader and more competitive marketplace, since cold storage facilities often function as de facto terminal markets. In Yanchang County, for instance, post-harvest cold storage centers aggregate supply from multiple apple growers, attract traders from wider regions, and enable more transparent price formation. As buyers converge on these centralized storage hubs---rather than visiting individual villages---they are compelled to compete more directly with one another. This spatial and temporal concentration of trade reshapes the power dynamics of the upstream supply chain, temporarily re-balancing bargaining power in favor of producers, particularly those with access to efficient, low-cost storage. The act of storing thus alters not only when sales occur but also how the procurement market itself is structured.


\paragraph{Intertemporal Marketing and the New Storage Economy.}
This dissertation integrates farmers' storage choices with time-varying buyer competition in procurement markets. Classic cost-of-carry and competitive-storage models explain intertemporal price spreads through both demand and supply fundamentals and shocks, storage costs and convenience yield \citep{working1949theory,wright1982econ,williams1991storage}, and stock-flow dynamics under uncertainty \citep{gustafson1958implications,deaton1992behaviour}. Building on this foundation, I show that storage also functions as an \emph{intertemporal strategic instrument}: when buyer competition fluctuates intra-seasonally, the option to wait enables farmers to avoid weak trading conditions and access more competitive conditions as markets thicken. This perspective formalizes a bargaining-based motive for storage, complementing conventional arbitrage explanations and linking dynamic oligopsony to farmer marketing behavior.


Reframing storage as a strategic response to time-varying buyer power yields three implications:
\begin{enumerate}
    \item \textbf{Welfare channel beyond price smoothing.} My work expands the welfare narrative around storage. Storage does not only enable price smoothing or loss reduction, it serves as a strategic tool to avoid monopsony pricing and access different competitive environments over time. 

    \item \textbf{Defensive storage and market distortions.} The widespread farmers' use of on-farm and commercial storage in China's apple industry, despite high costs, reflects procurement-market distortions rather than behavioral inefficiency. When buyer power is concentrated and competition weak over time, farmers employ storage defensively to avoid exploitative pricing. Even with limited or negative arbitrage margins, the welfare gain from avoiding low-competition periods can be substantial.  

    \item \textbf{Access and distributional effects.} Limited access to affordable storage reflects deeper institutional and financial frictions faced by poor or liquidity-constrained farmers. Because the welfare gains from storage accrue mainly to those with sufficient capital or credit, unequal access can widen rural income gaps and reinforce existing disadvantages. Expanding existing storage facilities, rising storage efficiency, and improving credit access can enhance both equity and overall market efficiency.
\end{enumerate}




\paragraph{A New Lens on Agricultural Market Power.}
This dissertation makes three core contributions to the study of agricultural market dynamics and storage adoption:

\begin{enumerate}
    \item \textbf{Conceptual}: It develops a two-period dynamic model of time-varying oligopsony in agricultural procurement, integrating temporal shifts in buyer competition and endogenous farmer storage decisions. The model departs from traditional static IO frameworks by incorporating time-vary oligopsonistic power across periods.

    \item \textbf{Empirical}: It draws on original field data, including over 500 interviews and surveys with farmers, traders, and storage operators, to document the temporal variability of market structure, the strategic logic of storage, and the welfare impacts of intertemporal marketing choices.

    \item \textbf{Policy Relevance}: It offers a novel economic justification for investments in cold chain infrastructure and market design. By reframing storage as a lever for redistributing bargaining power, the findings suggest that targeted storage subsidies, improvements in farmer access to commercial storage, and reductions in transaction frictions could serve as powerful tools for enhancing both efficiency and equity in agricultural markets.
\end{enumerate}


\paragraph{Structure of the Dissertation.}
The chapters that follow elaborate on these themes using both theoretical and empirical approaches.

\begin{itemize}
    \item \textbf{Chapter 2} provides an institutional and empirical overview of China's apple industry, with a focus on storage practices, upstream buyer structure, and the field-level evidence on collusion and entry.

    \item \textbf{Chapter 3} reviews the literature on storage economics, market power in agriculture, and smallholder marketing under uncertainty, identifying critical gaps this dissertation seeks to fill.

    \item \textbf{Chapter 4} presents a dynamic two-period model of time-varying oligopsony and storage adoption, analyzing both individual farmer decisions and aggregate welfare implications under different scenarios of buyer behavior and risk preferences.

    \item \textbf{Chapter 5} empirically tests key predictions using survey data and interviews from Yanchang County. It explores the determinants of storage adoption, the relationship between buyer competition and timing of sales, and the welfare impacts of storage under constrained competitive environments.

    \item The appendices contain mathematical derivations, additional robustness checks, and the survey instruments used in fieldwork.
\end{itemize}
