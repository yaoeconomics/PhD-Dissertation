% LR框架Outline

% 第一段:The economic role and welfare implications of commodity storage have been extensively discussed in the agricultural and development economics literature. 其中 arbitrage是主要的动机。大多数发现仓储对收入有正向影响。举例说明。 
\noindent The motivation for engaging in commodity storage has been extensively explored in the agricultural and development economics literature, with price arbitrage identified as the primary driving factor \citep{helmberger1977welfare, wright1984welfare, deaton1992behaviour, miranda1996, wright1982econ,minten2014new}. In the context of agricultural commodities, farmers often store crops after harvest, anticipating a possible increase in price that offsets their carrying charges. Most studies have found positive welfare impacts of storage adoption. For example, \cite{ruhinduka2020smallholder} investigate storage and processing decisions, which can increase income by more than 50\%, but also bring risk and time delays. \cite{aggarwal2018grain} conducted an experimental study showing that storage interventions significantly motivate farmers to store and sell their produce at later stages and higher prices. Similarly, \cite{priya2020post} examined the decision of smallholder farmers to adopt storage as a strategic tool to increase their agricultural income by leveraging price rises during non-harvest months. However, in reality, farmers may find it challenging to forecast storage returns, and the volatility in farm-gate prices can discourage them from storing crops for future sales, even when they have access to credit \citep{cardell2023price}. 


% 第二段:但是影响farm-gate price movements的因素很多且unpredictable。Traditional reasons for agricultural storage are driven by technological features (因此是跨农年的) 和 self-consumption(聚焦在主粮研究).  举例说明。。。。。
Farm-gate price movements come from various sources, such as production smoothing, downstream demand variability, consumption smoothing, and natural-disaster shocks \citep{tomek2001risk, channa2022overcoming}, resulting in the complex nature of the storage decision process. Therefore, previous literature predominantly examines storage impacts across crop years, tries to incorporate all these technological features, and focuses on staple crops such as wheat and potato, as they are consumables necessary for survival. For instance, \cite{saha1994household} present an agricultural household model capturing staple-crop consumption, storage, savings, and labor decisions. In parallel, \cite{park2006risk} develops a dynamic model demonstrating that grain's consumption role makes it an attractive form of precautionary storage. 


% 第三段:市场结构和竞争也会造成farm-gate价格波动,
Nevertheless, the often-overlooked changes in local procurement market conditions, such as competitiveness or the level of market integration, can result in input price fluctuations, thereby creating incentives for storage \citep{dries2009farmers,kopp2021farmers}. As suggested by \cite{zimmerman2003asset}, time presents both opportunities and vulnerabilities, with the latter often taking precedence for the poor. It is important to acknowledge that small-scale farmers in developing countries are often poor and face significant disadvantages in the battle against unpredictable price movements due to market structure change. A recent work by \cite{rubens2023market} shows that ownership consolidation in the Chinese tobacco industry resulted in an important rise in input price markdowns, redistributing income away from rural households. Also, \cite{chatterjee2023market} confirms that increasing competition between intermediaries generated by the law causes prices received by farmers to increase a lot. 
%% 需要添加文件!!!!!%%



% 第四段:但文献忽略了time-varying market competitive conditions的影响。举例说明。。。。。。   其实很多情况下,即使其他变量恒定,市场竞争条件的变化依然会让农户仓储变得有效。
Despite the importance of this channel, little attention has been paid to the potential impacts of time-varying market competitive conditions on farmers’ storage decisions. As \cite{sudhir2005time} suggest, competition can vary over time in a local market and might be a function of demand and cost conditions. Other sources of time-varying competition include inter-temporal market division where firms alternate to be active \citep{herings2005intertemporal} and potential competition deterrents \citep{gilbert1989role, stiglitz1981potential}. In fact, inter-temporal changes in market competition conditions alone are sufficient for farmers to benefit from storage in many cases. 


% 第五段:虽然有一些Industrial Organization和Operation Management领域的文献已经触及了市场竞争和仓储/库存的之间的interplay,但多是聚焦在供应链中下游,并且多数研究的是卖方之间的竞争,而不是买方。
Although literature in industrial organization and operations management has delved into the interplay between storage inventory and market competition, the emphasis has largely centered on the perspective of those involved in storage as sellers \citep{leng2005game}, leaving the role of input buyers and their oligopsony power relatively unexplored. For example, \cite{li1992role} demonstrates that storage adoption could promote seller's delivery-time competition and hence increase consumers' welfare; \cite{rotemberg1989cyclical} present a model in which a duopoly uses storage to deter deviations from an implicitly collusive agreement; \cite{anand2008strategic} capture the existence of strategic inventory where buyers employ inventories to prompt sellers to reduce future prices, influencing vertical competition in a dynamic monopoly-to-monopsony model; Building upon this framework, \cite{hu2021strategic} and \cite{cai2021supply} extend the analysis by incorporating various forms of horizontal competition among sellers. While most of these works developed dynamic storage-related models to either capture inventory deterioration or allow demand variability, none has introduced the possibility of time-varying market structure change, especially on the side of buyers. 



% 第七段:我的文章的贡献。
My work here aims to address these gaps by examining a dynamic storable crop market with stochastic time-varying market competitive conditions where oligopsonistic middlemen face a continuum of competitive farmers capable of storing the crop in anticipation of higher future prices due to higher buyer-side competition levels. Aligning with the work of \cite{porteous2019high} and \cite{ruhinduka2020smallholder}, this study abstracts away from production decisions and focuses on post-harvest storage strategies within a crop year. To isolate the time-varying oligposony power, this study further controls demand variability and other confounding factors. In essence, this work lies in the marriage of industrial organization and agricultural development economics \citep{bellemare2022agricultural} and offers an innovative approach to assisting small and low-income suppliers in developing nations to make better post-harvest decisions to combat oligopsony power downstream.






