\documentclass[12pt]{article}
\usepackage{hyperref}
\usepackage{authblk}
% Document Layout
\usepackage{natbib}
\bibliographystyle{aer}

\usepackage{geometry} % Customize document dimensions, margins, and page size.
\usepackage{fancyhdr} % Extensive control of page headers and footers.
\usepackage{titlesec} % Control over section and chapter headings.

% Font and Text
\usepackage[utf8]{inputenc} % Allows input of international characters.
\usepackage[T1]{fontenc} % Font encoding.
\usepackage[english]{babel} % Multilingual support.
\usepackage{amsmath, amsfonts, amssymb} % American Mathematical Society packages for advanced math typesetting.
\usepackage{mathptmx} % Times font
\usepackage{helvet} % Helvetica font
\usepackage{courier} % Courier font

% Graphics and Tables
\usepackage{graphicx} % Enhanced support for graphics.
\usepackage{subfigure} % or use \usepackage{subcaption} for handling sub-figures within a single figure environment.
\usepackage{float} % Improved interface for floating objects (tables, figures).
\usepackage{wrapfig} % Allows figures or tables to have text wrapped around them.
\usepackage{pgf, tikz} % Creating high-quality diagrams and figures.
\usepackage{xcolor} % Easy driver-independent access to several kinds of color tints, shades, tones, and mixes of arbitrary colors.
\usepackage{color} 
\usepackage{tabularx} % Enhanced tables.
\usepackage{booktabs} % Publication quality tables in LaTeX.

\usepackage{threeparttable}
\usepackage{longtable}
\usepackage{pdflscape}

% Set the page size and margins
\geometry{letterpaper, portrait, margin=1in}


\title{Sketch for the Conceptual Framework}
\author[1]{Zhiyao (Yao) Ma}
\affil[1]{UC Davis}
\date{\today}


\begin{document}
\maketitle
\tableofcontents
\newpage

% ------------------------------------------ %
% ------------------------------------------ %
% ------------------------------------------ %
\section{Base Model: A Two-Period Storage Decision Model}

I study a farmer's intertemporal selling decision over two trading periods under uncertainty and risk-sensitive preferences. The objective is to characterize how much of the harvest should be stored for the second period, and how this decision depends on price expectations, time preferences, and the farmer's risk aversion.

\subsection{Setup}

The farmer harvests one unit of output and chooses the share \( s \in [0,1] \) to store for the second period. The rest, \( 1 - s \), is sold in the first period. I normalize the first-period price to 1 without loss of generality, so the second-period price \( P_2 \) is measured relative to this benchmark.

Let \( \delta \in (0,1) \) denote the discount factor reflecting the cost of storage or the farmer’s time preference. The farmer's preferences are described by a constant relative risk aversion (CRRA) utility function:

\[
U(x) = 
\begin{cases}
\frac{x^{1 - \gamma}}{1 - \gamma}, & \gamma \neq 1 \\
\ln x, & \gamma = 1
\end{cases}
\]
The parameter \( \gamma \in \mathbb{R} \) governs the curvature of utility here. The farmer maximizes expected utility from revenue across two periods hence:

\[
\max_{s \in [0,1]}  U(1 - s) + \delta \cdot \mathbb{E} \left[ U(s \cdot P_2) \right] 
\]

\subsection{Case 1: Risk-Averse}
Under risk aversion, \( U(x) \) is strictly concave. The problem becomes:

\[
\max_{s \in [0,1]}  \frac{(1 - s)^{1 - \gamma}}{1 - \gamma} + \delta \cdot \mathbb{E} \left[ \frac{(s \cdot P_2)^{1 - \gamma}}{1 - \gamma} \right] 
\]
Factoring out constants, the problem simplifies to:

\[
\max_{s \in [0,1]}  (1 - s)^{1 - \gamma} + \delta \cdot s^{1 - \gamma} \cdot \mathbb{E}[P_2^{1 - \gamma}] 
\]
Define \( \mu \equiv \mathbb{E}[P_2^{1 - \gamma}] \). Then the objective function becomes:

\[
f(s) = (1 - s)^{1 - \gamma} + \delta \cdot \mu \cdot s^{1 - \gamma}
\]
This is a concave function over \( s \in [0,1] \).

\paragraph{Kuhn–Tucker Conditions:}
Let \( \lambda_1 \geq 0 \) and \( \lambda_2 \geq 0 \) be the multipliers on the constraints \( s \geq 0 \) and \( s \leq 1 \), respectively. The Lagrangian is:

\[
\mathcal{L}(s, \lambda_1, \lambda_2) = (1 - s)^{1 - \gamma} + \delta \cdot \mu \cdot s^{1 - \gamma} + \lambda_1 s + \lambda_2 (1 - s)
\]
The Kuhn–Tucker conditions are:

\begin{align*}
\text{(i)} \quad & f'(s) + \lambda_1 - \lambda_2 = 0 \\
\text{(ii)} \quad & \lambda_1 \geq 0,\quad \lambda_1 \cdot s = 0 \\
\text{(iii)} \quad & \lambda_2 \geq 0,\quad \lambda_2 \cdot (1 - s) = 0 \\
\text{(iv)} \quad & 0 \leq s \leq 1
\end{align*}
The derivative of the objective is:

\[
f'(s) = - (1 - \gamma)(1 - s)^{-\gamma} + \delta \cdot \mu \cdot (1 - \gamma) s^{-\gamma}
\]
Setting \( f'(s) = 0 \) and solving for \( s \) gives the interior solution:

\[
\left( \frac{s}{1 - s} \right)^\gamma = \delta \cdot \mu
\quad \Longrightarrow \quad
s^* = \frac{(\delta \cdot \mu)^{1/\gamma}}{1 + (\delta \cdot \mu)^{1/\gamma}}
\]
This solution lies in \( (0,1) \) if \( \delta \cdot \mu \in (0, \infty) \). When \( \delta \cdot \mu \to 0 \), \( s^* \to 0 \); when \( \delta \cdot \mu \to \infty \), \( s^* \to 1 \). Hence, corner solutions are possible even under risk aversion.

\subsection{Case 2: Risk-Neutral}

When the utility function is linear, the problem becomes:

\[
\max_{s \in [0,1]} \; (1 - s) + \delta \cdot s \cdot \mathbb{E}[P_2]
\]
This is a linear function in \( s \). The slope is:
\[
f'(s) = -1 + \delta \cdot \mathbb{E}[P_2]
\]
The optimal decision is:
\[
s^* =
\begin{cases}
0 & \text{if } \delta \cdot \mathbb{E}[P_2] < 1 \\
1 & \text{if } \delta \cdot \mathbb{E}[P_2] > 1 \\
\text{any } s \in [0,1] & \text{if } \delta \cdot \mathbb{E}[P_2] = 1
\end{cases}
\]

\subsection{Case 3: Risk-Loving}

With convex utility, the objective function is also convex. Any interior point that satisfies the first-order condition is a minimum, not a maximum. Therefore, the optimum lies at one of the boundaries.

Let's compare the utility at \( s = 0 \) and \( s = 1 \):

\[
U(1) \quad \text{vs.} \quad \delta \cdot \mathbb{E}[U(P_2)]
\quad \Rightarrow \quad
1^{1 - \gamma} \quad \text{vs.} \quad \delta \cdot \mathbb{E}[P_2^{1 - \gamma}]
\]
This yields:

\[
s^* =
\begin{cases}
0 & \text{if } \delta \cdot \mathbb{E}[P_2^{1 - \gamma}] < 1 \\
1 & \text{if } \delta \cdot \mathbb{E}[P_2^{1 - \gamma}] > 1 \\
\text{any } s \in [0,1] & \text{if } \delta \cdot \mathbb{E}[P_2^{1 - \gamma}] = 1
\end{cases}
\]

\subsection{Summary}

Let \( \mu = \mathbb{E}[P_2^{1 - \gamma}] \). The optimal storage share is:

\[
s^* =
\begin{cases}
\displaystyle \frac{(\delta \cdot \mu)^{1/\gamma}}{1 + (\delta \cdot \mu)^{1/\gamma}}, & \gamma > 0 \\
\text{any } s \in [0,1], & \gamma = 0 \text{ and } \delta \cdot \mathbb{E}[P_2] = 1 \\
0, & \gamma < 0 \text{ and } \delta \cdot \mu < 1 \\
1, & \gamma < 0 \text{ and } \delta \cdot \mu > 1
\end{cases}
\]
This expression captures both interior and corner solutions depending on the values of the risk aversion coefficient, the discount factor, and the expected second-period price.




\section{Buyer Competitiveness and Price Formation}

Let's allow the second-period price to depend on a market structure parameter \( \theta \in [0,1] \), which captures the degree of buyer-side competitiveness. A higher value of \( \theta \) indicates more competition among buyers, resulting in higher prices received by the farmer. A lower value of \( \theta \) reflects greater market power among buyers, potentially reducing the price through oligopsonistic behavior.

\subsection{Price Formation under Buyer Competition}

Let the second-period price be given by:
\[
P_2(\theta) = \bar{p}_2 \cdot \phi(\theta),
\]
where \( \bar{p}_2 > 0 \) is a base or frictionless price, and \( \phi(\theta) \in (0,1] \) is an increasing function of \( \theta \) capturing the effect of buyer competition on the realized price. For example, I may assume \( \phi(\theta) = \theta^\eta \) for some \( \eta \in (0,1] \), reflecting diminishing marginal gains from competition. 

As \( \theta \to 0 \), the market is highly concentrated (e.g., monopsony), and prices are suppressed. As \( \theta \to 1 \), prices approach the perfectly competitive benchmark \( \bar{p}_2 \).

\subsection{Modified Objective Function}

The farmer’s problem now becomes:
\[
\max_{s \in [0,1]} \left\{ U(1 - s) + \delta \cdot \mathbb{E} \left[ U\left( s \cdot \bar{p}_2 \cdot \phi(\theta) \right) \right] \right\}
\]
Define:

\[
\mu(\theta) \equiv \mathbb{E} \left[ \left( \bar{p}_2 \cdot \phi(\theta) \right)^{1 - \gamma} \right] = \bar{p}_2^{1 - \gamma} \cdot \mathbb{E} \left[ \phi(\theta)^{1 - \gamma} \right]
\]
Then the problem reduces to the earlier form, but now with a \( \theta \)-dependent moment:

\[
\max_{s \in [0,1]} \left\{ (1 - s)^{1 - \gamma} + \delta \cdot s^{1 - \gamma} \cdot \mu(\theta) \right\}
\]
The optimal storage share is given by:

\[
s^*(\theta) = 
\begin{cases}
\displaystyle \frac{\left( \delta \cdot \mu(\theta) \right)^{1/\gamma}}{1 + \left( \delta \cdot \mu(\theta) \right)^{1/\gamma}}, & \gamma > 0 \\
\text{corner solution based on } \delta \cdot \bar{p}_2 \cdot \phi(\theta), & \gamma \leq 0
\end{cases}
\]
The function \( \mu(\theta) \) is increasing in \( \theta \) for \( \gamma > 0 \), since \( \phi(\theta)^{1 - \gamma} \) is increasing in \( \theta \). As buyer competition intensifies, the expected price improves and the farmer finds it more attractive to store the harvest. 

Thus, \( \frac{\partial s^*}{\partial \theta} > 0 \), holding all else constant. This implies that policies or market changes that reduce buyer concentration—such as easing entry restrictions or enhancing transportation infrastructure—can increase the farmer’s incentive to store and sell in later periods.












% ------------------------------------------ %
% ------------------------------------------ %
% ------------------------------------------ %
\newpage
\section{Traders' Competition: Oligopsonistic Maximization Problem}

To capture the element of time-varying buyer-side competition and incorporate it into the price parameters in the model above, I am considering the Cournot market structure for each trading period from the traders' perspective as follows. 

\subsection{Objective and Profit Function}
In each trading period, each buyer \( i \) aims to maximize their profit \( \pi_i \). The profit for buyer \( i \) is given by:
\begin{equation}
\pi_i = r_i(q_i) - p(Q) q_i
\end{equation}
where:
\begin{itemize}
  \item \( r_i(q_i) \) is the revenue obtained from selling the purchased quantity \( q_i \).
  \item \( p(Q) \) is the inverse supply function representing the price per unit when the total quantity \( Q \) is purchased.
  \item \( q_i \) is the quantity purchased by buyer \( i \).
  \item \( Q \) is the total quantity purchased by all buyers, \( Q = \sum_{j} q_j \).
\end{itemize}

\subsubsection{First-Order Condition (FOC)}
To maximize profit, the buyer will choose \( q_i \) such that the marginal revenue equals the marginal expenditure. The first-order condition (FOC) for maximization is:
\begin{equation}
\frac{\partial \pi_i}{\partial q_i} = \frac{\partial r_i(q_i)}{\partial q_i} - \frac{\partial [p(Q) q_i]}{\partial q_i} = 0
\end{equation}

\subsubsection{Additional Assumptions}
\begin{itemize}
  \item Homogeneous buyers: all buyers are identical.
  \item Perfect Competition in the output market: \( r_i'(q_i) = MR \) (marginal revenue is a constant, MR, for all buyers). Reasoning behind: Although traders may exercise buyer power in localized procurement markets, they then sell into broader output markets and face competition from traders who operate in different regions.
  \item The derivative of \( Q \) with respect to \( q_i \) is given by:
    \begin{equation}
    \frac{\partial Q}{\partial q_i} = 1 + \lambda
    \end{equation}
  \item The quantity purchased by each buyer is:
    \begin{equation}
    q_i = \frac{Q}{n}
    \end{equation}
    where \( n \) is the number of buyers.
  \item The elasticity of supply is:
    \begin{equation}
    \mu = \frac{\partial Q}{\partial p} \cdot \frac{p}{Q}
    \end{equation}
\end{itemize}

\subsubsection{First-Order Condition again}
\begin{equation}
MR = p + \frac{Q}{n} p'(Q) (1 + \lambda)
\end{equation}
Given the elasticity of supply \(\mu\):
\begin{equation}
\mu = \frac{\partial Q}{\partial p} \cdot \frac{p}{Q}
\end{equation}
Inverting to express \(\frac{\partial p}{\partial Q}\):
\begin{equation}
\frac{\partial p}{\partial Q} = \frac{1}{\frac{\partial Q}{\partial p}} = \frac{p}{\mu Q}
\end{equation}
Substitute \( p'(Q) \) into FOC:
\begin{equation}
MR = p + \frac{Q}{n} \cdot \frac{p}{\mu Q} \cdot (1 + \lambda)
\end{equation}
Simplifying:
\begin{equation}
MR = p + \frac{p}{n \mu} \cdot (1 + \lambda)
\end{equation}
\begin{equation}
MR = p \left(1 + \frac{1 + \lambda}{n \mu}\right)
\end{equation}

\subsection{Lerner Index}
\begin{equation}
L = \frac{MR - p}{p} = \frac{1 + \lambda}{n \mu}
\end{equation}

\subsection{Optimal Procurement Price}
Starting from:
\begin{equation}
MR = p \left(1 + \frac{1 + \lambda}{n \mu}\right)
\end{equation}
Solving for \( p \):
\begin{equation}
p = \frac{MR}{1 + \frac{1 + \lambda}{n \mu}}
\end{equation}
Simplify to avoid fraction within a fraction:
\begin{equation}
\textcolor{blue}{p = \frac{MR \cdot n \mu}{n \mu + 1 + \lambda}}
\label{Eq: optimal procure price, general case}
\end{equation}



\subsection{Important Notes}
\begin{enumerate}
    \item \textbf{Cartel Solution:} Cartel solution is NOT an NE for any static game, but it may be NE behavior in any single play of an infinite-horizon repeated game. (by Folk Theorem)
    
    \item \textbf{Why Cournot, instead of Bertrand?} Analogy to \cite{kreps1983quantity}, I can infer that when traders simultaneously and independently receive "downstream orders" for subsequent distribution, and I assume that the capacity level (the quantity of the downstream orders) are public information and traders compete in Bertrand-like price competition, with the supply allocated in Bertrand fashion where the provision that one cannot satisfy more supply than one's order from downstream in the first stage: \\
    "Capacity introduced + Bertrand" $\Longleftrightarrow$ "Cournot".

    \item \textbf{Supply Elasticity:} the village supply to this model should be elastic in both periods, but differ in two periods because farmers have the "outside" selling options differently. 
\end{enumerate}


\subsection{Case 1: Linear Supply Function}

\subsubsection*{Problem Setting}

Each buyer \( i \) aims to maximize their profit \( \pi_i \):
\[
\pi_i = r_i(q_i) - (a + bQ)q_i
\]
where \( p = a + bQ \) is the linear supply function.

\subsubsection*{First-Order Condition}

To maximize profit, the first-order condition (FOC) is:
\[
MR = \frac{\partial r_i(q_i)}{\partial q_i} = a + bQ + b \frac{Q}{n}
\]
Since \( q_i = \frac{Q}{n} \), the total quantity \( Q \) is:
\[
Q = nq_i
\]
Thus, the FOC simplifies to:
\[
MR = a + bQ \left(1 + \frac{1}{n}\right) = a + bQ \left(\frac{n+1}{n}\right)
\]
Rearranging to solve for \( Q \):
\[
MR = a + bQ \left(\frac{n+1}{n}\right)
\]
\[
MR - a = bQ \left(\frac{n+1}{n}\right)
\]
\[
Q = \frac{n(MR - a)}{b(n+1)}
\]

\subsubsection*{Cournot-competition Price}

Substitute \( Q \) back into the supply function:
\[
p = a + bQ
\]
\[
p = a + b \left( \frac{n(MR - a)}{b(n+1)} \right)
\]
\[
p = a + \frac{n(MR - a)}{n+1}
\]
\begin{equation}
    \textcolor{blue}{p = \frac{a + nR}{n+1}}
    \label{Eq: optimal procure price, linear supply case}
\end{equation}
Thus, the farm-gate price that these middlemen offer to farmers becomes an increasing function of the number of traders showing up ($n$), their marginal value product ($MR$), and the farmers' reservation price ($a$).


\section{Challenges Needed to Solve}
Therefore, I may combine the two settings above together. However, there exists a severe issue of inconsistency: in the derivation of the procurement price from the Cournot oligopsonistic setting, I implicitly assume an inverse supply function in each trading period. However, the farmers' maximization problem would actually alter the quantity of local supply hence change the farm-gate price I derive from the traders' perspective. 




% ------------------------------------------ %
% ------------------------------------------ %
% ------------------------------------------ %
\newpage
\appendix
\section{Previous Try}
\subsection{Numerical Illustration}
To illustrate our model clearly, let's assume prices are fixed at $p_1=6$ and expected future price $p_2=8$. I first analyze how the optimal share stored ($s^*$) varies with the risk aversion parameter $\gamma$ when $\delta=0.95$ (small storage cost), and then show how $s^*$ changes as the discount factor $\delta$ varies for different risk aversion levels.

\paragraph{Impact of Risk Aversion on Storage Decision}
Figure depicts how the optimal storage share ($s^*$) depends on the risk aversion parameter ($\gamma$) when $0<\gamma\leq1$. When $\gamma$ increases, farmers become more risk-averse. Given our parameters, more risk-averse farmers tend to store less to secure immediate certain income, avoiding exposure to future price uncertainty.


\paragraph{Impact of Storage Cost on Storage Decision}
Keeping the prices constant at the same values, Figure explores the impact of varying the discount factor (\( \delta \)) on the optimal storage decision. This graph features three curves, each representing different levels of risk aversion: low ($\gamma=0.1$), moderate ($\gamma=0.3$), and high ($\gamma=0.8$).

As the discount factor ($\delta$) increases, storage becomes less costly (or farmers become more patient), raising the incentive to store for future sales. Consequently, the optimal stored share ($s^*$) consistently increases. 

When the storage cost is reasonably small ($\delta \ge 0.75$), comparing different risk aversion levels, the lowest risk aversion curve ($\gamma=0.1$, yellow) lies above the more risk-averse ones. Less risk-averse farmers store more since they are more comfortable bearing future price uncertainty. Conversely, highly risk-averse farmers ($\gamma=0.8$, red) prefer immediate and certain income.


\paragraph{Role of Buyer Competition}
In our model, the parameter $\theta$ represents buyer competition intensity in period two. Increasing $\theta$ shifts the expected future price upward, enhancing incentives to store. In terms of Figure, higher second-period buyer competition would shift all curves upward, making storage more attractive at all discount factor levels. Similarly, higher buyer competition would raise the curve in Figure, leading farmers to store a larger share at every level of risk aversion.




\subsection{Linking the Model to Empirical Evidence}
The adoption of a constant relative risk aversion (CRRA) utility form here is motivated by two primary considerations. First, empirical evidence consistently shows that individuals' relative risk aversion remains constant across varying levels of wealth \citep{chiappori2011relative}, and the CRRA coefficient, being unitless, facilitates meaningful international comparisons of risk preferences \citep{hardaker2000some}. Second, recent empirical studies conducted in regions closely aligned with my own field sites reinforce this theoretical choice. For example, \cite{jin2024losses} find that apple growers in their surveyed area—which substantially overlaps with my fieldwork locations—exhibit consistent risk-averse behaviors, estimating their CRRA coefficients within the interval of approximately [0.437, 0.575).

My own empirical data, gathered from apple growers in Yanchang County, Central China, indicate an average CRRA coefficient of about 0.3 among the sampled farmers. Referring to the numerical analysis presented above, the optimal storage decisions of these farmers align closely with the orange curve illustrated in Figure.

Additionally, within my full sample of 549 apple growers, 200 households (approximately 36.4\%) chose to store all of their produce. Assuming that these farmers are identical and my theoretical model above accurately represents their behavioral tendencies, the observed storage rate of roughly 36.4\% corresponds to an inferred discount factor ($\delta$) of approximately 0.65. Consequently, this suggests that the combined storage costs—including storage fees, transportation, and deterioration—average around 0.35 per unit of stored produce.


\newpage
\bibliography{reference}




\end{document}