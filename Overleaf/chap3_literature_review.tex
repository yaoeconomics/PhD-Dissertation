% The economic role and welfare implications of commodity storage have been extensively discussed in the agricultural and development economics literature.  
\noindent The motivation for engaging in commodity storage has been extensively explored in the agricultural and development economics literature, with price arbitrage identified as the primary driving factor \citep{helmberger1977welfare, wright1984welfare, deaton1992behaviour, miranda1996, wright1982econ, minten2014new, nindi2024incentive}. In the context of agricultural commodities, farmers often store crops after harvest, anticipating a possible increase in price that offsets their carrying charges. Most studies have found positive welfare impacts of storage adoption. For example, \cite{ruhinduka2020smallholder} investigate storage and processing decisions, which can increase income by more than 50\%, but also bring risk and time delays. \cite{aggarwal2018grain} conducted an experimental study showing that storage interventions significantly motivate farmers to store and sell their produce at later stages and higher prices. Similarly, \cite{priya2020post} examined the decision of smallholder farmers to adopt storage as a strategic tool to increase their agricultural income by leveraging price rises during non-harvest months.

Beyond explicit price uncertainty, storage decisions are also shaped by financial considerations. For instance, \cite{janzen2024commodity} show that the opportunity cost of capital is an important determinant of inventory holdings among Illinois grain farms, with some farms behaving in line with profit-maximizing theory while others hold inframarginal inventories for non-market reasons. However, in reality, predicting storage returns proves challenging for farmers, and the volatility in farm-gate prices can discourage them from storing crops for future sales, even when they have access to credit \citep{cardell2023price}. In addition, limited access to those quality-preserving technologies further narrows farmers' marketing window, and the fixed cost of initial storage construction often makes storage adoption prohibitive for smallholder farmers, particularly in cash crops \citep{aggarwal2018grain}. Although some developing countries provide partial subsidies, uptake remains limited due to high uncertainty and capital constraints.

Traditional explanations for agricultural storage emphasize \textit{technological} features, such as cross-year production smoothing and self-consumption needs of staple crops. For instance, \cite{saha1994household} present a household model that integrates staple-crop consumption, storage, savings, and labor decisions. Likewise, \cite{park2006risk} develops a dynamic model showing that grain's consumption role makes it attractive for precautionary storage. In this strand of the literature, storage primarily serves food security and consumption smoothing, and is therefore concentrated in staple crop studies (wheat, potato, rice).



At the same time, smallholder farmers in developing countries often contend with the oligopsony power of middlemen and processors, leading to substantial margins for the latter \citep{rogers_rich_1994assessing}. While existing literature primarily explores the correlation between market power and \textit{spatial} trading frictions such as transportation and price-search costs \citep{bergquist_dinerstein_2020, mitra_mookherjee_torero_visaria_2018, ranjan_2017, antras_costinot_2011, jung2022structural}, the reasons behind small farmers frequently missing out on \textit{inter-temporal} marketing opportunities remain less well understood \citep{williams1991storage, wright1984welfare, ruhinduka2020smallholder, lai2003optimal}. Farm-gate price movements arise from multiple sources, including production smoothing, downstream demand variability, consumption smoothing, and natural-disaster shocks \citep{tomek2001risk, channa2022overcoming}, making the storage decision process highly complex.



Nevertheless, the often-overlooked changes in local procurement market conditions, such as competitiveness or the level of market integration, can result in significant input price fluctuations, thereby creating new incentives for storage \citep{dries2009farmers, kopp2021farmers}. As suggested by \cite{zimmerman2003asset}, time presents both opportunities and vulnerabilities, with the latter often taking precedence for the poor. In practice, the competitive advantages of storage, whereby farmers delay sales to face more competitive procurement conditions, have received very limited exploration in the literature. Recent evidence highlights this gap: \cite{rubens2023market} show that ownership consolidation in the Chinese tobacco industry raised buyer power and increased input price markdowns, reducing rural household incomes, while \cite{chatterjee2023market} confirm that increased competition among intermediaries leads to substantially higher prices for farmers.



Despite its importance, little attention has been paid to the potential impacts of \textit{time-varying market competitive conditions} on farmers' storage decisions. As \cite{sudhir2005time} suggest, competition in local markets can vary over time as a function of demand and cost conditions. Other sources of time-varying competition include inter-temporal market division where firms alternate activity \citep{herings2005intertemporal}, or potential competition deterrents \citep{gilbert1989role, stiglitz1981potential}. Inter-temporal changes in buyer competition alone may be sufficient to make storage profitable, even when other factors remain constant.



Although literature in industrial organization and operations management has delved into the interplay between storage inventory and market competition, the emphasis has largely centered on the perspective of those involved in storage as sellers \citep{leng2005game}, leaving the role of input buyers and their oligopsony power relatively unexplored. For example, \cite{li1992role} demonstrates that storage adoption could promote seller's delivery-time competition and hence increase consumers' welfare; \cite{rotemberg1989cyclical} present a model in which a duopoly uses storage to deter deviations from an implicitly collusive agreement; \cite{anand2008strategic} capture the existence of strategic inventory where buyers employ inventories to prompt sellers to reduce future prices, influencing vertical competition in a dynamic monopoly-to-monopsony model; Building upon this framework, \cite{hu2021strategic} and \cite{cai2021supply} extend the analysis by incorporating various forms of horizontal competition among sellers. While most of these works developed dynamic storage-related models to either capture inventory deterioration or allow demand variability, none has introduced the possibility of time-varying market structure change, especially on the side of buyers. 


My work aims to address these gaps by examining a dynamic storable crop market with stochastic time-varying market competitive conditions where oligopsonistic middlemen face a continuum of competitive farmers capable of storing crops in anticipation of higher future prices. Aligning with \cite{porteous2019high} and \cite{ruhinduka2020smallholder}, my theoretical analysis in Chapter 4 abstracts away from production decisions to focus on post-harvest storage strategies within a crop year. To isolate the role of buyer power, I further control for demand variability and other confounders. In essence, this work contributes at the intersection of industrial organization and agricultural development economics \citep{bellemare2022agricultural}, offering an innovative approach to assist small and low-income suppliers in developing nations to make more effective post-harvest decisions under oligopsony power.
