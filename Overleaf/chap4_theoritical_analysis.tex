\noindent
Buyer power is most often framed as a static structural characteristic of agricultural-product procurement markets, but especially in developing-country settings, is a dynamic phenomenon based on trader activity and behavior in local markets. This paper reframes oligopsony power in farm-product procurement markets as a time-varying rather than fixed phenomenon. Farmers with capacity to store their production relatively efficiently may benefit in these settings from accessing more competitive market conditions than prevail at harvest. Commodity storage in these settings is more than a vehicle for arbitrage of predictable demand and supply shifts. It is also a means to potentially countervail harvest-period buyer power. To study this behavior, I build a two-period model in which buyer power is time-varying and farmers with storage access can possibly attain more competitive selling windows into the future. The framework illuminates how expectations about future competitive conditions contribute to determining farmer storage choices and realized selling conditions. I show that temporal changes in buyer power alone can suffice for farmers to benefit from storage. Data-driven simulations map farmers' storage efficiency, risk attitudes, knowledge of harvest-period market conditions, and expectations for future competitive conditions to optimal storage behavior and welfare outcomes. The analysis quantifies conditions when waiting and storage pay and shows how risk aversion and inefficient storage suppress storage and potential welfare gains. Policies that increase farmers' access to efficient storage or reduce effective risk aversion can work to increase farmer incomes.

%--------------------------------------------------------%
\section{Introduction}
\noindent    
Buyer power arises from the immobility of certain factor inputs that, in the short run, are largely ``captive'' to a limited set of buyers. While spatial immobility has been extensively explored in the literature, the impact of changes in temporal immobility on bargaining dynamics remains underexamined. This gap is particularly relevant in agricultural markets, where products are typically somewhat perishable, and proper storage plays a critical role in the supply chain.


To the best of our knowledge, this study is the first to unveil the interplay of farmers' (sellers') storage adoption in the presence of time-varying competition in the procurement market. Without the adoption of storage, smallholder farmers who cultivate crops that spoil quickly face limitations in their trading options, as they can only sell their produce locally at harvest time because they typically lack access to trucks or other transportation equipment, preventing them from accessing distant selling locations. But if they were equipped with advanced storage technology, they would be able to seek out a higher local price brought from more competitive market conditions among middlemen in the later periods, as shown in Figure \ref{Figure: Demo}.

\begin{figure}[ht!]
\centering
\includegraphics[width=1\textwidth]{figures/graphic_demo.png}
\caption{Extended Marketing Opportunities from Storage Adoption}
\label{Figure: Demo}
\end{figure}

Specifically, farmers could potentially benefit from increased competition in the oligopsonistic market through two sources. Firstly, the presence of different intermediaries in the village at various times can create fluctuating levels of competition on a monthly or even weekly basis.\footnote{In this chapter, the terms middlemen, field buyers, intermediaries, and traders will be used interchangeably to refer to a group of buyers who directly purchase fresh apples from farmers. These buyers play a crucial role in the supply chain by acting as the initial link between farmers and the downstream market. They typically visit orchards, negotiate prices with farmers, and handle the immediate procurement of apples. Their role may also include transporting apples to wholesale markets, processing facilities, or exporters, depending on the supply chain dynamics. Regardless of the specific term used, all these buyers share the common function of directly sourcing apples from farmers and selling them downstream to wholesalers, retailers, and processors.}  Secondly, farmers with storage can tap into additional distribution channels such as e-commerce and direct selling, which differ from the conventional middlemen-dominated system. By doing so, they can introduce external participants into the oligopsonistic market at the farm gate at different time nodes. 

I develop a conceptual framework to explore how smallholder farmers can adopt cold storage to potentially exploit the time-varying buyer power of middlemen. It considers a simplified two-period scenario in a developing country, where farmers sell a specific cash crop to middlemen who visit local villages to procure the farm product. The model incorporates a storage-decision process, wherein farmers, observing farm-gate prices at harvest, decide whether to sell or store their crops.

The outcomes of this study hold substantial implications for policymakers and farmers alike. By exploring the dynamics of time-varying oligopsony levels, the research aims to demonstrate the potential benefits of storage facilities for smallholder producers. In a broader context, the findings suggest that embracing storage adoption could provide farmers with an effective alternative to combat anti-competitive practices like buyer collusion and market allocations and avoid more intrusive measures in markets, such as direct government intervention.





%----------------------------------------------------------------%
%----------------------------------------------------------------%
\section{Model}
\noindent
We consider a two-period framework to analyze how smallholder farmers with access to storage may use it strategically in response to time-varying buyer power within a marketing season. Output at harvest time is exogenous and is normalized to one unit of homogeneous quality for each farmer. Farmers exhibit risk preferences ranging from risk neutrality to significant risk aversion. Risk preferences are represented by a von Neumann-Morgenstern utility function,\footnote{An alternative mean-variance framework, presented in Appendix \ref{Appendix: mean-variance approach}, yields similar insights. However, due to limited empirical guidance on calibrating the risk aversion coefficient within agricultural supply chains for the mean-variance framework, our primary analysis relies on the expected utility framework and, specifically, on the CRRA utility function.} $U_i(\pi_i)$, with $U_i' > 0$ and $U_i'' \leq 0$, where $\pi_i$ is the farmer's net income from crop sales. Farmers aim to maximize the expected utility of income across two trading periods.

A farmer's decision at harvest is determining the proportion of the crop to store, denoted by $s \in [0,1]$, where by default $s=0$ for farmers who lack access to storage. Farmers with storage access observe farm-gate price offers from traders and infer the current buyer power, $\theta_1$. They form expectations about future market conditions, specifically the extent of buyer power, $\theta_2$. To maintain focus on the role of time-varying competition conditions, we assume that future market demand and supply conditions are predictable. In this setting, absent changes in market competition over time, farm price and downstream prices would rise between periods 1 and 2 to exactly compensate the costs of efficient period 1 to 2 storage. However, small-scale, on-farm storage or even storage organized at the village level may not be efficient relative to commercial storage facilities or facilities operated by downstream agents.\footnote{The extent, if any, to which farmers discount future income at a greater rate than other supply-chain actors also causes $\kappa <1$} With efficient storage denoted by $\kappa = 1$, decrements in farmer storage efficiency are indicated by $\kappa$ values less than 1.0.

Formally, each farmer seeks to maximize expected utility over two trading periods by selecting storage share $s$. Given harvest quantity $q = 1$, the optimization problem is:
\begin{equation}
\label{eq:starting objective}
\max_{s \in [0,1]} \mathbb{E} \left(U\left[ (1 - s) p_1 + s \cdot \kappa p_{2} \right]\right)
\end{equation}
where $p_1$ is the observed farm-gate price at harvest and $p_{2}$ is expected price in period 2, which is discounted to the extent farmer storage is inefficient, $\kappa < 1$.


\subsection{Trader Market Structure and Farm-Gate Price Formation} \label{Section: Middlemen Market Structure and Farm-Gate Price Formation}
\noindent
Following the framework of the FOOM model \citep{karp1996dynamic,sexton2001assessment,saitone2009flexible,Perekhozhuk_2017, hamilton2021joint}, farm-gate price, $p_t$, in any period is determined by the product's net value of marginal product in that period ($VMP_t$), but $p_t$ is reduced below $VMP_t$ by the exercise of buyer oligopsony power, $\theta_t$. The price decrement due to buyer power is also influenced by the price elasticity of the farm product supply, $\epsilon_t$, with markdowns due to buyer power made more extreme the less elastic is the farm product supply. 

We assume traders act as perfect competitors in their downstream sales and incur constant per-unit marginal costs, $c_t$, in acquiring farm product and selling to downstream buyers. Let $P_t$ denote the downstream price, in which case the farm product's net value of marginal product is $VMP_t = P_t-c_t$. The farmgate price, accounting for buyer power is then:\footnote{The derivation of Equation~(\ref{farm_price-t}) follows directly from the FOOM model for settings where intermediaries exercise only buyer power. Details are provided in Appendix \ref{Appendix: Derivation of Farm Price under the FOOM Framework}. }

\begin{equation}
\label{farm_price-t}
p_{t} = \frac{VMP_t}{1 + \frac{\theta_t}{\varepsilon_t}}.
\end{equation}

Given the assumptions, $VMP_1 = VMP_2$ because $P_2$ must rise in equilibrium to exactly compensate higher transaction costs in period 2 relative to period 1 ($c_2 > c_1$) due to efficient commodity storage. Without loss of generality, we normalize $VMP_t = 1, t = 1, 2$. Given this normalization, equation~(\ref{farm_price-t}) can be re-expressed as:
\begin{equation}
\label{simplified-price}
p_{t} = \frac{\varepsilon_t}{\varepsilon_t + \theta_t}.
\end{equation}
Although total harvest is fixed, market supply at harvest is elastic, $\varepsilon_1 > 0$, based upon the opportunity to store the product or divert it to alternative markets, e.g., fresh versus processing or domestic versus export. Storage is no longer an option at period 2, but elasticity in supply is still retained through the ability to sell outside the primary market, e.g., to the juice processing sector in the case of Fuji apples studied by \citet{ma2025timevarying}.

Although the two-period model developed here suffices to parsimoniously establish our key points regarding farmer incentives to store commodity to access time-varying market competition, it creates a technical issue in the form of a ``final-period problem'' in that farmers lose the option for further storage at period 2. However, in reality farmers who have committed a product to storage gain access to multiple market windows, where a final-period problem is not an issue. 

To capture the effect of supply elasticity parsimoniously and avoid an artificial final-period problem, we assume a baseline market supply elasticity of $\varepsilon_t = 1$ in both periods.\footnote{This normalization preserves a transparent mapping from buyer power to prices and avoids mechanical intertemporal asymmetries. Allowing $\varepsilon_t$ to vary does not alter the core mechanisms of the model. It simply rescales the markdowns implied by buyer power: higher elasticity compresses the markdown and lifts the corresponding price schedule, whereas lower elasticity expands the markdown and depresses prices.}
We relax this assumption by allowing $\varepsilon_t$ to vary either across periods in Appendix~\ref{Appendix: Sensitivity to relative Supply Elasticity}, or jointly over time in Appendix~\ref{Appendix: Sensitivity to common Supply Elasticity change}. Under this further simplification, farm-gate prices across the two periods simplify to
\begin{equation}
p_{t} = \frac{1}{1+\theta_t}, \qquad \theta_t \in [0,1], \; t \in \{1,2\}.
\label{Eq: price formation by buyer power}
\end{equation}
Accounting for storage costs, the expected net second-period price is
\begin{equation}
    \kappa\, p_2(\theta_2)
= \frac{\kappa}{1+E[\theta_2]}.
\end{equation}

Farmers' intertemporal storage decisions depend critically on their expectations about future buyer power, $E[\theta_2]$, which directly shape their expectations of the net second-period price, $E[p_{2}] = E\!\left[\frac{\kappa}{1+\theta_2}\right]$. Farmers form beliefs about the distribution of $\theta_2$ potentially informed by first-period buyer conduct, $\theta_1$, and observations of the evolution of market conditions from prior experience. These expectations determine the perceived price differential between immediate sale and deferred sale through storage, thereby linking market-structure uncertainty to individual storage decisions.



\begin{table}[H]
\centering
\caption{Economic Environment at Harvest}
\label{tab:baseline model parameter table}
\begin{tabular}{lll}
\toprule
\textbf{Item} & \textbf{Symbol (units)} & \textbf{Status at Harvest} \\
\midrule
Harvest quantity & $q$ & Known, fixed (normalized to 1) \\
First-period price & $p_1$ & Observed \\
First-period buyer power & $\theta_1 = \frac{1 - p_1}{p_1}$ & Inferred from $p_1$ \\
Second-period buyer power & $\theta_2$ & Stochastic \\
Second-period price & $p_2 = \frac{1}{1 + \theta_2}$ & Derived \\
Storage efficiency factor & $\kappa \le 1$ & Observed \\
Second-period net price & $\kappa p_2 = \frac{\kappa}{1 + \theta_2}$ & Derived \\
CRRA coefficient & $\gamma \in [0,10]$ & Observed \\
\bottomrule
\end{tabular}
\end{table}



\subsection{Farmer Optimization}

\noindent The economic environment for a farmer at harvest is summarized in Table~\ref{tab:baseline model parameter table}. A farmer allocates a share $s \in [0,1]$ of output to the second-period sale and retains the remainder for immediate sale at price $p_1$. Substituting the expressions for $p_1$ and $p_{2}$, a farmer's maximization problem becomes:
\begin{equation}
\label{eq:final objective}
\max_{s \in [0,1]} \mathbb{E} \left\{U\left[\underbrace{\frac{1-s}{1+\theta_1}}_{\text{First-period income}} + \underbrace{s \cdot \frac{\kappa}{1+\theta_2}}_{\text{Adjusted second-period income}} \right]\right\}.
\end{equation}
We assume farmer risk preferences follow CRRA:
\begin{equation}
    U(\pi)=\left\{
    \begin{array}{ll}
    \pi & \text { if } \gamma = 0 \\
    \frac{\left(\pi^{1-\gamma}-1\right)}{(1-\gamma)} & \text { if } \gamma > 0 \text { and } \gamma \neq 1 \\
    \ln (\pi) & \text { if } \gamma=1
    \end{array}\right.
\label{eq: CRRA}
\end{equation}
where $\gamma$ denotes the CRRA coefficient and $\pi$ represents the net income realized over two periods. 


CRRA implies a constant degree of relative risk aversion across income levels, consistent with standard models of choice under uncertainty. Empirically, household portfolio studies show that the share of risky assets remains stable across wealth levels despite income fluctuations, supporting CRRA as a reasonable approximation \citep{chiappori2011relative, Berger2020Characterizing, zavala2024unfair}. The unitless risk-aversion coefficient $\gamma$ further allows meaningful comparisons of risk preferences across contexts \citep{Szpiro1986Relative, hardaker2000some}. A recent meta-analysis by \citet{Irsova2025Relative} estimates average relative risk aversion near 1 in general applications and between 2 and 7 in financial settings, values consistent with field observations for the apple-producing regions in China \citep{jin2024losses}. We allow $\gamma$ to vary over $[0,10]$, spanning risk-neutral to highly risk-averse preferences to capture a full picture of how risk attitudes shape storage decisions.

\subsection{Risk Neutrality: Closed-Form Solution}
\noindent Under risk neutrality, $U(\pi) = \pi$, the objective simplifies to:
\begin{equation}
\max_{s \in [0,1]} \; 
(1 - s) \cdot \frac{1}{1 + \theta_1} 
+ 
s \cdot \kappa \cdot \mathbb{E} \left[ \frac{1}{1 + \theta_2} \right].
\end{equation}
The solution, $s^*$, depends solely on a comparison of marginal returns:
\begin{equation}
s^*( \gamma = 0) =
\begin{cases}
1 & \text{if } \kappa \cdot \mathbb{E} \left[ \frac{1}{1 + \theta_2} \right] > \frac{1}{1 + \theta_1}, \\
0 & \text{if } \kappa \cdot \mathbb{E} \left[ \frac{1}{1 + \theta_2} \right] \leq \frac{1}{1 + \theta_1}.
\end{cases}
\label{Eq: risk-neutrality solution}
\end{equation}

\noindent The risk-neutral farmer evaluates only the expected monetary payoff, and the storage decision reduces to a binary choice: if the net expected second-period price exceeds the first-period price, full storage is optimal; otherwise, full sale at harvest is optimal.
%Indifference arises only when the two expected returns are equal.


\subsection{Risk Aversion: Numerical Approach Needed}
\noindent In the presence of risk aversion, the CRRA utility function is strictly concave, and a closed-form solution to the farmer's maximization problem is generally not available. The difficulty arises because the random second-period buyer–power parameter $\theta_2$ enters inside a nonlinear utility transformation. With strictly concave $U(\cdot)$, the first-order condition (FOC) for interior solution is
\begin{equation}
    \mathbb{E}\!\left\{
U'\!\left[\frac{1-s}{1+\theta_1}
+ s\frac{\kappa}{1+\theta_2}\right]
\left[-\frac{1}{1+\theta_1}+\frac{\kappa}{1+\theta_2}\right]
\right\}=0.
\label{Eq: Optimization FOC}
\end{equation}

For standard concave preferences (e.g., CRRA) and a nondegenerate distribution for $\theta_2$, the first-order condition does not simplify to an algebraic expression that can be solved explicitly for $s$. Even when the distribution of $\theta_2$ is simple, analytic integration of the first-order condition is generally not available. The second-period net return $\kappa/(1+\theta_2)$ is hyperbolic, and its composition with a nonlinear marginal utility function $U'$ does not yield an expectation that simplifies to an algebraic expression in $s$. As a result, no closed-form expression for an interior optimum exists for the baseline parametric preferences used in this paper.

Risk aversion also implies that the optimality condition depends jointly on the mean and higher-order moments of $\theta_2$. The effects of variance, skewness, or higher moments on $s^{*}$ enter through higher derivatives of $U$, which remain embedded inside the expectation operator in Equation~(\ref{Eq: Optimization FOC}) and cannot be separated into moment-specific terms. This interaction between utility function curvature and the full distribution of $\theta_2$ further limits tractability and reinforces the need for numerical methods. We therefore approximate the solution to Equation~(\ref{Eq: Optimization FOC}) using Monte Carlo draws of $\theta_2$ and solve for the optimal storage share $s^{*}$, while respecting the corner constraints $s \in [0,1]$. Appendix \ref{Section: Special Cases: Explicit Competition Forms} recasts the model using standard forms of noncooperative oligopsony behavior---price-setting (Bertrand-Nash) competition and quantity-setting (Cournot-Nash) competition.






\section{Numerical Analysis} \label{Section: Base Model Numerical Analysis} 
\noindent This section describes the simulation design and computational procedure employed to approximate the farmer's decision problem under CRRA utility. We set first-period buyer power to a representative value of $\theta_1 = 0.5$, the midpoint of its support. This choice is made with little loss of generality because results depend primarily on how the mean and variance of future buyer power $\theta_2$ shape storage incentives for a given value of $\theta_1$, and the qualitative patterns are unchanged for other values of $\theta_1$. What matters is how expectations about $\theta_2$ evolve relative to harvest-time conditions, not the absolute level of $\theta_1$.

Second-period buyer power $\theta_2$ is modeled as a Beta-distributed random variable on $[0,1]$, providing flexible control over its mean, variance, and skewness and maintaining supports on the unit interval consistent with the FOOM model. We examined eight Beta distributions formed by a full factorial of four mean values for $\theta_2$, 
$\mu_{\theta_2} \in \{0.2, 0.4, 0.5, 0.8\}$, and two variances, 
$\sigma^2_{\theta_2} \in \{0.02, 0.05\}$, corresponding to low and high uncertainty regarding future buyer power.\footnote{For each $(\mu_{\theta_2}, \sigma^2_{\theta_2})$ pair, the corresponding unique pair of Beta distribution shape parameters $(\alpha, \beta)$ are computed using the standard moment-matching formulas: $\alpha = \mu_{\theta_2} \left( \frac{\mu_{\theta_2}(1 - \mu_{\theta_2})}{\sigma^2_{\theta_2}} - 1 \right), \quad \beta = (1 - \mu_{\theta_2}) \left( \frac{\mu_{\theta_2}(1 - \mu_{\theta_2})}{\sigma^2_{\theta_2}} - 1 \right).$}$^,$\footnote{Each admissible $(\mu,\sigma^2)$ implies unique parameters $(\alpha,\beta)$ for the Beta distribution.} We simulate the optimal storage share $s^*$ by drawing 5{,}000 realizations of $\theta_2$ from the Beta distribution for each of the eight $(\mu_{\theta_2}, \sigma^2_{\theta_2})$ scenarios and searching a 100-point grid over $s \in [0,1]$ to identify the value, $s^*$, that maximizes expected CRRA utility of two-period income for every $(\gamma,\kappa)$ pair that defines a specific farmer.

We set the range for the farmer storage-efficiency parameter $\kappa \in [0.6,1.0]$, capturing quality loss, storage costs, and discounting between periods relative to the efficient storage benchmark, $\kappa = 1$. Although evidence concerning efficiency decrements of small-scale, on-farm storage relative to commercial storage is not readily available, evidence from developing-country settings shows substantial farmer discounting of future income streams: \citet{jin2024losses} report present values for delayed payments ranging from 22--96\% among apple growers and \citet{ding2021time} estimate an annual value around 85\% for grape growers in China. Even U.S. farmers exhibit sizable discounting, with annual present values of roughly 70--78\% \citep{duquette2012farmer}.\footnote{For comparability across studies, we convert standard discount rates $r$ into present-value factors using $\alpha = 1/(1+r)$, where $r = \left(\frac{\text{Future Value}}{\text{Present Value}}\right)^{1/t} - 1$. For instance, $r = 0.17$ implies $\alpha = 1/1.17 \approx 0.85$.} Given that $\kappa$ incorporates both time discounting and physical or financial storage frictions, allowing values as low as 0.6 is empirically well grounded.


\subsection{3D Visualization and Interpretation}
\noindent To present results, we use a 4-by-4 panel (Figure~\ref{Figure:3D_formulation}). The top and bottom rows show the PDFs of the eight Beta distributions for $\theta_2$, each labeled with its mean--variance pair $(\mu_{\theta_2}, \sigma^2_{\theta_2})$ and implied shape parameters $(\alpha, \beta)$. The middle two rows plot 3D surfaces of the optimal storage share $s^*$ over $(\gamma, \kappa)$, separately for low-variance (second row) and high-variance (third row) cases. The columns from left to right denote increasing values for $\mu_{\theta_2}$. All 3D plots share a common viewing angle, with the lowest $\gamma$ and $\kappa$ values ``closest'' to the observer. These surfaces summarize how intertemporal storage responds to uncertainty in local buyer power.

\begin{figure}[ht!]
\centering
\includegraphics[width=\textwidth]{model_figures/3D_formulation.png}
\caption{Optimal Storage Share and PDF Visualizations under Eight Beta Distributions of $\theta_2$}
\label{Figure:3D_formulation}
\end{figure}

In general, when the farmer expects buyer power to be stronger in the future than at harvest, that is, when $\mathbb{E}[\theta_2] > \theta_1$, a farmer has no incentive to store the harvest. Thus, farmers will not store when $\mu > 0.5$. As illustrated in the far-right column of the panel ($\mu_{\theta_2}=0.8$), the optimal strategy across all risk and efficiency levels is to sell immediately ($s^* = 0$).

When $\mathbb{E}[\theta_2] = \theta_1$ (center-right column), a farmer with high storage efficiency and little risk aversion may still find storage worthwhile. This arises because the expected second-period price $\mathbb{E}[p_2] = \mathbb{E}\!\left[\frac{1}{1+\theta_2}\right]$ slightly exceeds the deterministic benchmark $p_1= \frac{1}{1+\theta_1}$, owing to the convexity of the price function in $\theta_2$. Consequently, even when the mean level of buyer power is unchanged across periods, the convexity of payoffs with respect to price risk may induce positive storage adoption, as illustrated in the center-right column in Figure~\ref{Figure:3D_formulation}.

 When expectations about future buyer power are favorable ($\mathbb{E}[\theta_2] < \theta_1$), interior solutions ($0 < s^* < 1$) emerge over a narrow range of risk preferences and storage efficiencies. Moreover, as the expected buyer power $\mathbb{E}[\theta_2]$ declines, both the scope for interior solutions and the frequency of full storage ($s^* = 1$) expand.

Within each simulated surface, $s^*$ is non-increasing in $\gamma$. This pattern is especially pronounced when future buyer power is moderate-that is, when expected second-period prices are only marginally higher than those in the first period. Risk-averse farmers value certainty over potentially higher but uncertain future payoffs. As $\gamma$ rises, the disutility associated with price risk tends to outweigh the option value of storage, especially when those gains are modest due to expectations of relatively higher buyer power in period 2. This point is illustrated by the center left column where $\mu_{\theta_2}$ = 0.4. Although a more competitive market is forecast in period 2, storage inefficiency and risk aversion cause $s^*=0$ for many farmers.


To no surprise, greater storage efficiency supports a positive storage decision. For a given level of risk aversion, increases in $\kappa$ make storage more attractive by increasing the expected net second-period price. The responsiveness of $s^*$ to $\kappa$ weakens at higher values of $\gamma$, suggesting that even access to highly efficient storage cannot fully offset the behavioral costs of risk exposure when farmers are strongly averse to it. Even in the settings most conducive to profitable storage, such as $\mu_{\theta_2} = 0.2, \sigma^2= 0.02$, inefficient storage ($\kappa < 0.78$) makes storage suboptimal for farmers.


Cross-panel comparisons further illustrate how changes in the distribution of second-period buyer power affect storage incentives. As the mean future buyer power $\mu_{\theta_2}$ increases (moving left to right across the panels), expected second-period prices decline, rendering storage less attractive. Accordingly, the optimal storage share $s^*$ shifts downward across the entire surface. This effect is especially pronounced for risk-neutral or mildly risk-averse farmers, who condition decisions mostly or exclusively on expected returns. The lower the expected future price, the more immediate sale becomes the dominant strategy.


Variance in farmers' expectations of future buyer power exerts a subtle yet systematic influence. Comparing the second and third rows of the panel, low versus high variance cases, shows that greater dispersion in expectations of future buyer power ($\sigma^2$) discourages storage in general, especially among the most risk-averse farmers. Higher variance increases downside risk in second-period returns, and with concave utility this uncertainty weighs more heavily for those with stronger risk aversion. While highly risk-averse farmers are already reluctant to store, an increase in variance further reinforces their preference for immediate sale. By contrast, at low or moderate levels of $\gamma$, the marginal effect of variance on $s^*$ is smaller, as these farmers are less sensitive to additional risk.



\subsection{Rarity of Interior Solutions}
Interior solutions, where the farmer chooses to allocate positive fractions of the harvest to both the first and second periods, are rare in this framework. The logic is straightforward: when the intertemporal trade-off is starkly tilted in favor of one period (either due to realization of $\theta_1$ relative to expected second-period buyer power $\mu$, extreme risk preferences, or inefficient storage), the farmer's best response is to sell all in one period. This is consistent with fieldwork evidence in the fresh apple industry in Central China: most apple growers tend to make corner decisions, either storing nearly everything when they expect future prices to be favorable, or selling everything immediately when uncertainty or storage cost dampens the incentive to wait. Within the model, such corner solutions arise from the nonlinearity of the CRRA utility function combined with the hyperbolic dependence of price on buyer power, which amplifies small differences in expectations or attitudes into decisive action.

To better understand the conditions under which interior solutions might arise, I extract a few representative cases from the simulation results shown in Figure~\ref{Figure:3D_formulation}. These examples are selected from scenarios with low variance in second-period buyer power and a low mean, specifically $\mu = 0.2$, where the distribution of future prices is skewed toward favorable outcomes. I restrict attention to moderate levels of risk aversion and relatively high storage efficiency, where neither risk nor cost fully dominates the farmer's decisions.

In the first case, I consider a farmer with a risk aversion coefficient of $\gamma = 2$ and a storage efficiency of $\kappa = 0.80$, with an expected second-period buyer power of $\mu = 0.2$. Here, this farmer's optimal choice is to store approximately 80\% of the harvest. The low expected buyer power implies a high expected price in the second period, while the relatively efficient storage technology ensures that much of this value is preserved. The farmer, though risk averse, finds it worthwhile to defer income but hedges by selling a small portion immediately. This behavior aligns well with farmers who have moderate patience and risk tolerance and access to good post-harvest storage infrastructure.

In the second case, I increase risk aversion to $\gamma = 2.75$, holding $\kappa = 0.80$ and $\mu = 0.2$ constant. The optimal storage share drops to about 62.5\%. The higher risk aversion amplifies the farmer's sensitivity to downside risk in second-period prices, even though the mean is favorable. The farmer still stores the majority, but the share sold in the first period increases, locking in a greater share of income from the crop. This behavior exemplifies more cautious farmers who hedge against worst-case outcomes even when market fundamentals appear strong.

In the third case, I keep risk aversion fixed at $\gamma = 2.75$ but reduce storage efficiency to $\kappa = 0.68$. The optimal storage share declines further to about 46\%. The decline in $\kappa$ reduces the effective return to deferred sales, making immediate selling relatively more attractive. This farmer, facing the same preferences and expectations as in the previous case, responds not to belief or attitude but to technical constraint. It reflects real-world circumstances where storage facilities are substandard or subject to losses from spoilage, pests, or theft-conditions that naturally discourage holding output even among patient, forward-looking producers.

Together, these three cases illustrate the sharp responsiveness of the storage decision to both preferences and storage costs. As risk aversion increases, storage declines, all else equal. And for a fixed level of risk aversion, lower storage efficiency can easily shift the farmer from a moderately storage-heavy strategy to one that leans toward immediate sale. Crucially, these interior solutions emerge only under a narrow configuration of parameters: relatively favorable and stable expectations, moderate caution, and at least tolerable storage conditions. Once any of these dimensions shifts toward uncertainty, risk aversion, or storage economic inefficiency, the solution quickly reverts to a corner.

In summary, interior solutions occur only when multiple forces---expected price gains, risk preferences, and storage efficiency---are delicately balanced. These cases are not typical but represent transitional zones in the farmer's decision space. Their rarity in both theory and field data affirms the robustness of the model's prediction: for most farmers, especially under poor storage conditions, decisive corner choices are the norm.




\subsection{Sensitivity: Expected Future Buyer Power and Risk Aversion}
\noindent To further examine how a farmer's optimal storage share $s^*$ responds to varying expectations about future buyer power, I conducted a simulation with the same parameter setting as in the 3D graphics above by holding first-period buyer power fixed at $\theta_1 = 0.5$, with a storage efficiency of $\kappa = 0.9$. The second-period buyer power $\theta_2$ was modeled as a Beta-distributed random variable with support on $[0, 1]$, maintaining a fixed variance of 0.02 while allowing the mean to vary from 0.05 to 0.95. For each mean value, I derived the associated Beta shape parameters and simulated 5,000 draws of $\theta_2$. The optimal storage share was then computed by maximizing expected utility under CRRA preferences, evaluated across five levels of risk aversion: $\gamma \in \{0, 0.5, 2, 4, 7\}$. As shown in Figure~\ref{Figure: sensitivity to second-period buyer power}, each resulting function $s^*(\mathbb{E}[\theta_2])$ was plotted to visualize how forward-looking uncertainty and risk preferences jointly shape storage behavior. Line color was varied monotonically with $\gamma$, increasing in darkness as risk aversion intensified.

\begin{figure}[ht!]
\centering
\includegraphics[width=\textwidth]{model_figures/sensitivity_to_theta_2.png}
\caption{Sensitivity to Expected Second-Period Buyer Power}
\label{Figure: sensitivity to second-period buyer power}
    \begin{tablenotes}[flushleft]
    \footnotesize
    \item \textit{Note:} This figure illustrates how the optimal storage share $s^*$ varies with the expected second-period buyer power $\mathbb{E}[\theta_2]=\mu$ under different levels of risk aversion. 
    Simulation parameters: $\theta_1=0.5$, $\kappa=0.9$, $\mathrm{Var}(\theta_2)=0.02$, number of draws $=5000$, and storage grid of 25 points.
    \end{tablenotes}
\end{figure}


The results reveal a distinct shift in behavior across risk preference levels. Under risk neutrality ($\gamma = 0$), the decision rule is binary: the farmer stores all output if the expected net-storage-cost second-period price exceeds the current one, and none otherwise. As $\gamma$ increases to positive values, the decision becomes more nuanced. The sharp threshold gradually turns into a smooth, decreasing function of $\mathbb{E}[\theta_2]$, with interior solutions appearing when the expected future buyer power is only marginally lower than the first-period one.

Even when the expected second-period price exceeds today's price, the farmer often chooses to sell immediately due to the downside risk embedded in the distribution of $\theta_2$. The turning point at which full storage becomes less attractive shifts leftward as the more risk averse farmers are.

\begin{figure}[ht!]
\centering
\includegraphics[width=\textwidth]{model_figures/expected_price_gain_vs_mu.png}
\caption{Expected Price Gains versus Mean Second-Period Buyer Power}
\label{Figure: expected price gain vs mu}
\begin{tablenotes}[flushleft]
\footnotesize
\item \textit{Notes:} This figure plots expected price gains, $\kappa\,\mathbb{E}[p_2]-p_1$, as a function of mean buyer power $\mu = \mathbb{E}[\theta_2]$ for $\theta_1 \in \{0.25, 0.5, 0.75\}$, highlighting the $\theta_1=0.5$ case. 
Simulation setup: $\kappa=0.9$, $\mathrm{Var}(\theta_2)=0.02$, and $5000$ Beta draws.
\end{tablenotes}
\end{figure}



\begin{figure}[ht!]
\centering
\includegraphics[width=\textwidth]{model_figures/sensitivity_to_gamma.png}
\caption{Sensitivity to Risk Aversion}
\label{Figure: sensitivity to risk aversion}
    \begin{tablenotes}[flushleft]
    \footnotesize
    \item \textit{Notes:} This figure compares optimal storage behavior across risk aversion coefficients $\gamma \in \{0.5, 2, 4, 7\}$. 
    Other parameters: $\theta_1=0.5$, $\kappa=0.9$, $\mathrm{Var}(\theta_2)=0.02$, and $5000$ Beta-distributed draws for $\theta_2$ in each $\mu$ grid point.
\end{tablenotes}
\end{figure}


In addition, I constructed another visualization in Figure~\ref{Figure: sensitivity to risk aversion} that examines how a farmer's optimal storage share responds to changes in risk aversion, under varying expectations about future buyer power. With the same parameter setting as in the 3D graphics above, I simulate buyer power beliefs using Beta distributions with means ranging from $\mu = 0.2$ to $\mu = 0.4$ in increments of 0.05. For each mean, I compute the corresponding shape parameters $(\alpha, \beta)$, draw 5,000 samples of $\theta_2$, and evaluate the expected utility under a CRRA utility function across 100 grid points of $\gamma \in [0, 10]$. The optimal storage share $s^*$ is computed by maximizing the expected utility over a grid of candidate values $s \in [0, 1]$, reflecting the proportion of harvest stored for second-period sale. Also, Figure~\ref{fig:unified_gap_plot}, discussed later in policy implications, quantifies the income consequences of risk aversion, showing both the absolute and proportional losses relative to the risk-neutral benchmark across the range of expected buyer power.



The resulting graph in Figure~\ref{Figure: sensitivity to risk aversion} plots $s^*$ against $\gamma$ for each value of $\mu$. Several patterns emerge clearly. First, storage shares generally decline with increasing risk aversion, but the relationship is not strictly monotonic. Over some ranges of $\gamma$, the curves flatten, indicating regions where changes in risk aversion have little effect on storage behavior. In these cases, either the expected price advantage or the storage cost dominates the decision, making farmers' optimal storage choices relatively insensitive to marginal changes in risk preferences. 

Second, higher expected buyer power in the second period (i.e., higher $\mu$) uniformly depresses the storage share. When future market power is expected to be strong (i.e., lower $p_2$), forward sales become less attractive, reinforcing the incentive to sell early regardless of risk preferences.

Interestingly, the sensitivity of $s^*$ to risk aversion ($\gamma$) becomes more pronounced as the expected future buyer power ($\mu$) approaches the current level $\theta_1$. When $\mu$ is low (e.g., $0.2$), storage remains attractive even for highly risk-averse farmers, as the expected second-period price is sufficiently high to compensate for risk. As $\mu$ increases (e.g., to $0.35$), the storage share declines sharply at relatively low levels of $\gamma$, indicating that even moderately risk-averse farmers prefer to sell immediately when future price expectations weaken. Consistent with Figure~\ref{Figure: sensitivity to second-period buyer power}, when $\mu = 0.4$, no farmer stores at all, since the expected price advantage is fully eroded by the effective storage costs.


This graphical structure complements prior results showing $s^*$ as a function of $\mu$ at fixed $\gamma$. Together, they offer a two-dimensional understanding of farmer behavior: beliefs about future buyer power shift the entire $s^*(\gamma)$ curve, while changes in risk aversion tilt its shape. These patterns are particularly relevant for understanding heterogeneous storage behavior across farmers who face similar market fundamentals but differ in attitudes toward risk or expectations about downstream market structure.



\subsection{Sensitivity Analysis: Impacts of Supply Elasticity on Optimal Storage}
\noindent Results in the main text assumed a constant farm supply elasticity, $\varepsilon_1 = \varepsilon_2 = 1$ across both periods. In this appendix, we first consider how results are impacted when the farm supply elasticity changes across periods, considering possibilities that period 2 supply is less elastic, $\varepsilon_1 > \varepsilon_2$ or more elastic, $\varepsilon_1 < \varepsilon_2$, than harvest period elasticity. We then consider impacts when $\varepsilon_1 = \varepsilon_2$ but the common supply elasticity is greater or less than the unitary elasticity assumed in the main text.



\subsubsection{Optimal Storage When Supply Elasticity Changes across Periods} 
\label{Appendix: Sensitivity to relative Supply Elasticity}
\noindent
We consider how farmers' optimal storage responds when farm supply is either more or less elastic in the second period. Either condition is possible. $\varepsilon_2 < \varepsilon_1$ is plausible because availability of the storage option in period 1 but not period 2 can make period 1 supply more elastic than in period 2. $\varepsilon_2 > \varepsilon_1$ is also plausible because, as noted, moving product into central storage facilities may expand selling options.



To isolate the impacts of changing price elasticity, we fix the distribution of second-period buyer power at $\theta_2 \sim \text{Beta}(\mu,\sigma^2)$ with $\mu = 0.2$ and $\sigma^2 = 0.05$ and vary only the second-period supply elasticity $\varepsilon_2 \in \{0.75,1.00,1.25\}$. The first-period elasticity is set at $\varepsilon_1 = 1$, so that any change in intertemporal incentives 
is driven entirely by how $\varepsilon_2$ reshapes the mapping from buyer power to price in period 2.

\begin{figure}[ht!]
    \centering
    \includegraphics[width=\linewidth]{model_figures/sensitivity_to_gamma_eps2_with_p2_panel.png}
    \caption{Comparative Sensitivity of Storage and Price to Second-Period Supply Elasticity}
    \label{fig: sensitivity_to_supply_elasticity}
\end{figure}

Figure~\ref{fig: sensitivity_to_supply_elasticity} summarizes the results. 
Panel~(a) shows that $s^*(\gamma)$ is non-decreasing with the second-period supply elasticity: the schedule for $\varepsilon_2 = 1.25$ lies above 
that for $\varepsilon_2 = 1.00$, which lies above that for $\varepsilon_2 = 0.75$. Greater elasticity dampens the price markdown associated with period 2 buyer power, raising $p_2(\theta_2,\varepsilon_2)$ for any given $\theta_2$. This increases the expected payoff from waiting, lowers the risk-aversion threshold at which storage becomes profitable, and yields higher optimal storage shares throughout the range of~$\gamma$.

Panel~(b) makes this mechanism explicit by plotting the implied expected second-period price,
$$
    \mathbb{E}[p_2]
    \;=\;
    \mathbb{E}\!\left[\frac{\varepsilon_2}{\varepsilon_2 + \theta_2}\right],
$$
for each elasticity value. Because $p_2(\theta_2,\varepsilon_2)$ is strictly increasing in $\varepsilon_2$ for all $\theta_2 \geq 0$, the distribution of second-period prices shifts upward as supply becomes more elastic. In our calibration, moving from $\varepsilon_2 = 0.75$ to $\varepsilon_2 = 1.25$ raises $\mathbb{E}[p_2]$ by a nontrivial amount, and this upward shift in expected revenue accounts for the higher storage shares in Panel~(a).

\begin{figure}[ht!]
    \centering
    \includegraphics[width=\linewidth]{model_figures/3D_formulation_supply_elasticity.png}
    \caption{Optimal Storage Share for Different Second-period Mean Buyer Power and Supply Elasticity}
    \label{fig: 3D_supply_elasticity}
\end{figure}


In addition to the one-dimensional storage schedules, Figure~\ref{fig: 3D_supply_elasticity} provides a fuller visualization of the interaction between risk aversion, storage efficiency, the mean of second-period buyer power, and the supply elasticity. Each row holds fixed a value of $\varepsilon_2 \in \{0.75,1.00,1.25\}$, while each column varies the mean of future buyer power $\mu \in \{0.2,0.4,0.5,0.8\}$. 

Comparing rows reveals the key elasticity mechanism: increasing $\varepsilon_2$ uniformly lifts the entire $s^*(\gamma,\kappa)$ surface. Because higher second-period supply elasticity compresses the markdown generated by any realization of $\theta_2$, the second-period price schedule shifts upward, and so even moderate levels of risk aversion now justify delaying sales for a broader set of $(\gamma,\kappa)$ combinations.


Overall, the sensitivity analysis confirms that the model's qualitative comparative statics are robust to reasonable variation in supply elasticity. Allowing $\varepsilon_2 \neq 1$ simply rescales the expected second-period price path: more elastic supply under buyer power (higher $\varepsilon_2$) strengthens the incentive to store by raising expected future prices, whereas less elastic supply (lower $\varepsilon_2$) dampens storage incentives by lowering those prices. The underlying mechanism---storage as a response to time-varying buyer power---remains unchanged.



\begin{figure}[ht!]
    \centering
    \includegraphics[width=\linewidth]{model_figures/grid_sensitivity_gamma_eps_common_by_mu.png}
    \caption{Storage Share and Prices under Different Second-period Mean Buyer Power and Supply Elasticity}
    \label{fig: GRID_common_supply_elasticity_across_periods}
\end{figure}


\subsubsection{Sensitivity of Optimal Storage to Common Supply Elasticity Changes Across Periods}
\label{Appendix: Sensitivity to common Supply Elasticity change}
\noindent
Figure~\ref{fig: GRID_common_supply_elasticity_across_periods} summarizes how farmers' optimal storage behavior responds to common changes in farmers' supply elasticity across periods. Each row corresponds to a different expected level of future buyer power, $E[\theta_2]\in\{0.20,0.30,0.35\}$, holding period 1 buyer power constant at $\theta_1 = 0.5$ and $\kappa = 0.9$. Each subplot traces the optimal storage share $s^*(\gamma)$ for three values of the common elasticity $\varepsilon\in\{0.75,1.00,1.25\}$. A change in $\varepsilon$ simultaneously alters the harvest-time price $p_1$ and the entire distribution of future prices $p_2(\theta_2)$.


A feature of Figure~\ref{fig: GRID_common_supply_elasticity_across_periods} is that the ordering of the three $s^*(\gamma)$ curves by $\varepsilon$ is not stable across rows. When expected future buyer power is low ($E[\theta_2]=0.20$), the curve for the highest elasticity ($\varepsilon=1.25$) lies slightly above the curves for lower elasticities, indicating that a more elastic supply environment encourages farmers to store more. When $E[\theta_2]$ increases to 0.30 or 0.35, however, the ranking reverses: the highest-elasticity curve shifts below the others, implying a lower storage propensity as $\varepsilon$ increases. This reversal is systematic and reflects how a joint change in elasticity affects the \emph{relative} movements of $p_1$ and $\mathbb{E}[p_2]$ across periods.


The underlying economic mechanism is that increasing $\varepsilon$ raises both $p_1$ and $p_2$, but not at the same rate. When future buyer power is low (small $\theta_2$), $p_2$ is already relatively high, and a marginal increase in $\varepsilon$ proportionally boosts $p_2$ more than $p_1$. The intertemporal price ratio,
$\frac{\kappa\,\mathbb{E}[p_2(\varepsilon,\theta_2)]}{p_1(\varepsilon,\theta_1)},$
rises, strengthening the return to waiting. In this region, higher elasticity naturally leads to higher storage, and the $\varepsilon=1.25$ curve lies above the others.


As expected future buyer power rises and approaches the harvest-time level ($\theta_1=0.5$), the two periods become more similar. In this environment, increases in $\varepsilon$ tend to raise $p_1$ more rapidly than $\mathbb{E}[p_2]$. Intuitively, when both periods face comparable buyer power, a high-elasticity environment disproportionately improves the return to selling immediately at harvest. The intertemporal price ratio then declines, reducing the value of waiting and lowering optimal storage. Consequently, the high-elasticity curve falls below the others in the lower rows of Figure~\ref{fig: GRID_common_supply_elasticity_across_periods}.


The grid thus illustrates that supply elasticity does not exert a monotonic or uniform influence on intertemporal marketing decisions. Its effect depends critically on the expected path of buyer power. When future competitive conditions are favorable, higher elasticity amplifies the gains from waiting; when future and current conditions converge, higher elasticity primarily raises the attractiveness of immediate sale. This non-monotonicity arises naturally once both $p_1$ and $p_2$ co-move with the same elasticity parameter, underscoring that storage incentives hinge on \emph{relative}, rather than absolute, price movements across periods.







\section{Policy Implications}
\noindent
Whether in developing or advanced economies, policymakers care deeply about farm incomes. A common response has been direct interventions, such as minimum support prices or income transfers, that boost incomes but often at the cost of market distortions, fiscal burden, and misallocation of resources. The model here suggests an alternative emphasis: improve the environment in which farmers make intertemporal selling decisions. Three margins emerge from the analysis, competition, storage efficiency, and risk aversion, each capable of lifting expected incomes without direct price manipulation.



\subsection{Enhancing Competition among Buyers}

\textbf{Mechanism.} When competition among buyers is weak at harvest, farmers face low farm-gate prices. In the model this is captured by a high $\theta_1$, which depresses the immediate return to selling. Storage provides farmers with the option value of waiting, and its payoff rises if competition is expected to be stronger in the future, i.e.\ when the mean of $\theta_2$ falls and the second-period price distribution shifts upward. In this way, stronger future competition amplifies the benefits of storage by improving the relative return to waiting rather than selling at harvest.

\textbf{Quantitative illustration ($\kappa=0.9$, $\text{Var}(\theta_2)=0.02$):}
\begin{itemize}
  \item Suppose $\theta_1=0.5$ (harvest price $p_1=0.667$) and beliefs about buyer power shift from $\mu=0.37$ (weaker future competition) to $\mu=0.32$ (stronger future competition). Risk-neutral and risk-averse ($\gamma=2$) farmers both switch from \emph{no storage} to \emph{full storage}. Expected income rises from $0.667$ to $0.690$, a gain of $+\;0.023$.
\end{itemize}

\textbf{Policy interpretation.} When policies succeed in enhancing buyer competition at harvest, the immediate farm-gate price $p_1$ rises, thereby improving farmer welfare unconditionally. In practice, however, such interventions may not always produce an immediate effect at the start of the marketing season. The model indicates that as long as buyer competition improves at any point during the marketing period, farmers can still possibly benefit through storage. Relevant policy instruments include stricter antitrust enforcement to deter collusion, lowering entry barriers for new traders, and investments in digital trading platforms or logistics infrastructure that expand farmers' access to alternative buyers throughout the season.






\subsection{Improving Storage Efficiency} \label{sec:storage-efficiency}

\textbf{Mechanism.} 
An increase in storage efficiency, represented by a higher $\kappa$, raises the expected return from delaying sales. Farmers effectively preserve a larger share of the crop's value, making intertemporal arbitrage more attractive. The effect is strongest when market conditions are expected to be better: even a modest gain in $\kappa$ can induce a shift from immediate sale to full storage. 

The welfare implications, however, depend on which producers adjust behavior. For farmers who would have stored fully regardless, a higher $\kappa$ merely transfers income. The principal welfare gains arise among marginal storers, those who switch from zero to positive storage or from partial to full storage. Only for these producers does the policy alter real storage decisions.


\begin{figure}[ht!]
    \centering
    \includegraphics[width=\textwidth]{model_figures/individual_storage_efficiency_policy_grid.png}
    \caption{Welfare and Behavior Change from 10-Percent Storage Efficiency Improvement: by Initial Efficiency Level}
    \label{fig: individual storage efficiency improvement}
    \begin{tablenotes}[flushleft]
    \footnotesize
    \item \textit{Note:} 
    Each panel shows simulated changes in farmers' expected income ($\Delta E[\pi]$) and optimal storage share ($\Delta s^*$) following a 10\% increase in storage efficiency ($\kappa$), under different initial efficiency levels ($\kappa_0 \in \{0.70, 0.80, 0.90\}$). 
    Rows distinguish between two market structures: symmetric buyer power ($\theta_1 = E[\theta_2]$) and fixed current buyer power ($\theta_1 = 0.5$). 
    Lines correspond to different risk-aversion levels ($\gamma$). 
    A positive $\Delta s^*$ indicates higher storage share for delay sales, while positive $\Delta E[\pi]$ reflects welfare gains.
    \end{tablenotes}
\end{figure}



\textbf{Quantitative Illustration.} 
Consider a policy intervention that raises $\kappa$ by 10\%. Let $\operatorname{var}(\theta_2)=0.02$ and examine risk aversion at $\gamma \in \{0,\,0.5,\,2,\,4\}$. Figure~\ref{fig: individual storage efficiency improvement} reports the induced changes in expected income, $\Delta E[\pi]$, and optimal storage share, $\Delta s^*$.

\emph{Symmetric buyer power ($\theta_1 = E[\theta_2] =\mu$).} 
Under symmetric expectations, storage efficiency improvements have subtle impacts on a farmer's welfare. When initial efficiency is $\kappa_0 = 0.70$ or $0.80$, the $\Delta E[\pi]$ curves are essentially zero across all $\mu$, indicating negligible behavioral response. At $\kappa_0 = 0.90$, however, $\Delta E[\pi]$ becomes slightly positive and declines with $\mu$: farmers facing lower mean buyer power (low $\mu$) experience greater gains because higher $\kappa$ triggers a shift from no to positive storage. Random variation in $\theta_2$ generates a Jensen effect on expected second-period prices, confirmed by a modest upward movement in the $\Delta s^*$ panel. These effects are much smaller than in the asymmetric case below.


\emph{Fixed current buyer power ($\theta_1 = 0.5$).} 
When current buyer power is fixed at $0.5$, both the level and shape of the curves change. In the $\Delta E[\pi]$--$\mu$ panels, all $\kappa_0$ lines start high at low $\mu$ and converge toward zero as $\mu$ rises. These higher curves, relative to the symmetric case above, reflect the larger value of delaying sales when the current price is weak. 

The $\Delta s^*$ panels exhibit a clear pattern. For $\kappa_0 = 0.70$, farmers transition from no to partial storage within a narrow $\mu$ interval---a steep ``hill.'' As $\kappa_0$ rises to $0.80$ and $0.90$, the hill shifts rightward, indicating that more farmers are already full storers and the pool of potential switchers narrows. 

Consequently, a 10\% improvement in $\kappa$ yields positive expected-income gains over a broader range of $\mu$ as baseline efficiency rises, but the behavioral margin drifts rightward.


\textbf{Policy Interpretation.} 
The simulations suggest that gains from improved storage efficiency are largest when future buyer power is expected to be weaker than current levels. As baseline efficiency increases, welfare effects become increasingly infra-marginal, and the $\Delta s^*$ ``hill'' in Figure~\ref{fig: individual storage efficiency improvement} shifts rightward. 

Importantly, higher effective $\kappa$ need not result from direct subsidies to storage. Similar outcomes can arise if farmers discount future income less. For instance, through income safety nets that smooth consumption or credit access that relaxes liquidity constraints. Such mechanisms enhance intertemporal storage efficiency without direct fiscal expenditure. 

It is worth-noting that such mechanisms often simultaneously reduce effective risk aversion. The two effects reinforce one another, making financial deepening a particularly high-leverage intervention. They improve farmers' capacity to wait (a $\kappa$ effect) and their willingness to wait (a $\gamma$ effect).







\subsection{Reducing Effective Risk Aversion}

\textbf{Mechanism.} Building on this complementarity, I turn to the role of risk preferences themselves. While intrinsic attitudes toward risk are relatively stable, \emph{effective} risk aversion can decline as farmers gain financial security through liquidity, insurance, or wealth accumulation. A lower coefficient of relative risk aversion ($\gamma$) shifts behavior toward the risk-neutral benchmark, with the largest effects concentrated in the region where storage is only marginally profitable in expectation.


\begin{figure}[ht!]
    \centering
    \includegraphics[width=0.85\textwidth]{model_figures/income_gap_vs_mu_(unified).png}
    \caption{Absolute and Relative Expected Income Losses from Risk Aversion}
    \label{fig:unified_gap_plot}
    \begin{tablenotes}[flushleft]
    \footnotesize
    \item Note: solid lines show the absolute gap $\mathbb{E}[\pi(s^*_{RN})]-\mathbb{E}[\pi(s^*_{RA})]$ (left axis). 
    Dashed lines show the same gap as a percentage of the risk-neutral benchmark (right axis). 
    Colors indicate the degree of risk aversion ($\gamma$). 
    The vertical dotted line marks the risk-neutral switching threshold $\mu^\star$ where storage ceases to be profitable.
    \end{tablenotes}
\end{figure}


\textbf{Quantitative illustration:} The Simulation results from the baseline model in Figure~\ref{fig:unified_gap_plot} show that absolute income losses from risk aversion peak at roughly 0.02--0.03 in normalized units, while the relative gap reaches about 2 percent of expected income under strong risk aversion ($\gamma=7$) and falls below 1 percent for moderate values ($\gamma=2$ to $4$). These effects vanish when storage is clearly profitable or clearly unprofitable, underscoring that the ``cost of risk aversion'' is concentrated exactly at the margin where storage incentives are fragile.


\textbf{Policy Interpretation.} 
The income effect from lowering risk aversion is modest but targeted: it recovers the ``money left on the table'' exactly where storage incentives are fragile. Even modest financial deepening---for example, reducing effective $\gamma$ from 7 to 2---eliminates more than half of the small but systematic income gap in this calibration. 

In practice, however, many instruments that reduce effective risk aversion also raise effective storage efficiency. Liquidity provision, warehouse---receipt finance, and crop insurance do more than alter risk preferences---they reduce the implicit cost of waiting by lowering liquidity needs, thereby increasing the effective $\kappa$.

Consequently, policies that expand farmers' financial security operate on both margins of welfare improvement: they make farmers less risk averse (lower $\gamma$) and more capable of delaying sales without loss (higher $\kappa$). The model treats these channels separately for analytical clarity, but in reality they reinforce each other, amplifying welfare gains relative to interventions that shift either margin alone.











\section{Optimal Storage under Cournot and Bertrand Competition}\label{Section: Special Cases: Explicit Competition Forms}
\noindent
Whereas buyer competition was represented flexibly in the main text based on values for the parameter $\theta$ consistent with the FOOM model, in this appendix we recast the model using the baseline models of noncooperative behavior from the industrial organization literature, symmetric Cournot-Nash (quantity) competition and Bertrand-Nash (price) competition. Both models allow buyers to exercise unilateral market power, but do not consider the possibility of buyer collusive behavior.

These models of symmetric competition assume buyers are identical, i.e., they operate with the same marginal costs, $c$ and are able to sell downstream at the same price $P$. Farmers observe the number of buyers present in the village in period 1, $N_1$, and form expectations about buyer presence in the second period, $\mathbb{E}[N_2]$.

Under Cournot competition the equilibrium farm-gate price increases smoothly in $N$, with $p$ converging to the competitive benchmark asymptotically as $N \to \infty$. In contrast, in the classic Bertrand setting, as few as two symmetric buyers engaged in price competition generate a Nash equilibrium equivalent to the competitive outcome, $p = VMP= P-c=1$, a result often referred to as the ``Bertrand's paradox.''



\subsection{Quantity-Setting Competition: Cournot--Nash Equilibrium}
\noindent In the quantity-setting environment, each homogeneous and symmetric buyer simultaneously chooses a purchase quantity, taking others' purchase decisions as given. The resulting Cournot-Nash equilibrium reflects strategic quantity-setting behavior. Unlike in the Bertrand case, prices are not binary but vary smoothly with the number of active buyers.

Maintaining the linear demand structure introduced in Equation~(\ref{Eq: price formation by buyer power}), the farm-gate price in each period $t$ under quantity-setting competition is given by:
\begin{equation}
p_t = \frac{N_t}{N_t + 1}, \quad t = 1,2, \text{ and } N_t \geq 1
\end{equation}
where $N_t$ denotes the number of buyers in period $t$. The price is strictly increasing in $N_t$, approaching the competitive limit of 1 as $N_t$ becomes very large.

Thus, the farmer's maximization problem under Cournot competition becomes:
\begin{equation}
\label{eq:Cournot objective function}
\max_{s \in [0,1]} \mathbb{E} \left\{ U \left[ 
\underbrace{(1-s) \cdot p_1(N_1)}_{\text{First-period income}} 
+ \underbrace{s \cdot \kappa \cdot p_2(N_2)}_{\text{Net second-period income}} 
\right] \right\},
\end{equation}
where $N_1$ is observed at harvest and $N_2$ is stochastic. 

We model the second-period buyer count $N_2$ as a Poisson random variable truncated to $[1,9]$, with the first-period count fixed at $N_1 = 2$, so $p_1 = 0.667$, same as our baseline in the main text with $\theta_1=0.5$. Farmers have CRRA preferences.

We consider four Poisson means, $\mu \in \{8,6,3,1\}$, capturing alternative expectations about second-period market competition. For each mean, we simulate 5{,}000 draws of $N_2$ to construct the $2\times4$ panel in Figure~\ref{Fig: 3D Cournot}. The top row reports the resulting probability mass functions; the bottom row displays 3D surfaces of the optimal storage share $s^*$ over $\gamma \in [0,10]$ and $\kappa \in [0.6,1.0]$, applying a closed-form solution when $\gamma = 0$. 

Storage increases with higher efficiency and lower risk aversion. When expected buyer presence in period 2 is low (e.g., $\mu = 1$), risk-averse farmers store substantially less.

\begin{figure}[ht!]
    \centering
    \begin{subfigure}{\textwidth}
        \centering
        \includegraphics[width=\textwidth, keepaspectratio=true]{model_figures/3D_cournot.png}
        \caption{Optimal Storage Share and PDF Visualizations under Four Poisson Distributions of $N_2$}
        \label{Fig: 3D Cournot}
    \end{subfigure}
    
    \vspace{10mm} % Reduced vertical gap between subfigures
    
    \begin{subfigure}{\textwidth}
        \centering
        \includegraphics[width=0.8\textwidth, keepaspectratio=true]{model_figures/buyer_count_sensitivity_cournot.png}
        \caption{Sensitivity to Expected Second-Period Buyer Presence ($\kappa=0.9$)}
        \label{Fig: Cournot sensitivity}
    \end{subfigure}

    \caption{Numerical Analysis under Cournot Competition}
\end{figure}

To examine how expectations about second-period buyer power interact with risk aversion, we varied the expected second-period buyer count $\mathbb{E}[N_2]$ from 1 to 8 in increments of 1 using truncated Poisson simulations. For each value of $\mathbb{E}[N_2]$, we computed the optimal storage share $s^*$ under five risk-aversion levels, $\gamma \in \{0, 0.5, 2, 4, 7\}$, holding storage efficiency fixed at $\kappa = 0.9$. Figure~\ref{Fig: Cournot sensitivity} plots the resulting curves $s^*(\mathbb{E}[N_2])$, with darker lines indicating higher risk aversion. A vertical dashed line marks $\mathbb{E}[N_2] = N_1$, highlighting the benchmark of symmetric market conditions across periods. For all $\gamma$, $s^*$ rises with expected buyer presence in period~2, but more risk-averse farmers require stronger second-period prospects to justify storage.

A central finding is that even under favorable conditions---a relatively efficient storage technology ($\kappa = 0.9$) and moderate competition at harvest ($N_1 = 2$)---storage becomes attractive only once farmers expect a substantially greater competition in period 2 as manifest by $\mathbb{E}[N_2]$. In particular, $s^*>0$ emerges only when $\mathbb{E}[N_2] \ge 4$, corresponding to an expected price increase from $p_1 = 0.667$ to $p_2 = \tfrac{4}{5} = 0.800$. This price increase is just sufficient to offset the $10\%$ effective loss from storage inefficiency.

The underlying mechanism reflects the concave price function $p(N) = \tfrac{N}{N+1}$: as $N$ grows, price increments from an additional buyer diminish rapidly, so even risk-neutral farmers need relatively optimistic expectations about period-2 competition to undertake intertemporal arbitrage.

By contrast, when buyer competition at harvest is weak, incremental increases in buyer presence generate much larger price improvements. For example, moving from $N_1=1$ to $N_2=2$ raises the price from $p_1=\tfrac{1}{2}$ to $p_2=\tfrac{2}{3}$, a substantial jump relative to the much smaller increase obtained when moving from $N_1=4$ to $N_2=5$. These discrete jumps can readily incentivize storage. Thus, the strong threshold effect above is most pronounced when the harvest market already displays moderate competition.

Re-specification of the model in the Cournot-Nash framework confirms that the mechanisms in the main text are not tied to the FOOM specification. With a smooth price schedule, storage responds in the same qualitative way to $\kappa$, $\mathbb{E}[N_2]$, and $\gamma$, reinforcing the robustness of the baseline insights.



\subsection{Price-Setting Competition: Bertrand--Nash Equilibrium}
\noindent
Thin local markets are widespread in smallholder economies. For example, \citet{Barrett_WD_1997} reports that in Madagascar only 29\% of rice farmers faced more than one buyer, meaning the majority was forced to sell into a monopsony setting. Under Bertrand-Nash price competition, the market outcome hinges critically upon whether multiple competing buyers are present. Under Bertrand competition, according to Equation~(\ref{Eq: price formation by buyer power}), the farm-gate price received by farmers can be modeled as:
\begin{equation}
p_t = 
\begin{cases}
0.5, & N_t = 1 \\
1, & N_t \geq 2
\end{cases}
\label{Eq: Bertrand Price Schedule}
\end{equation}
\noindent where $0.5$ denotes the monopsony price paid when only one buyer is present. The transition from a single-buyer market to a multi-buyer market results in a discontinuous shift in price.

Here, we assume that farmers observe the first-period price at harvest, and the number of traders at a specific village in the second trading period ($N_2$) is stochastic, with the probability of one middleman being $\rho$, i.e., $Pr(N_2=1)=\rho$. So, the probability of multiple intermediaries appearing in village $j$ is $Pr(N_2 \geq 2) = 1-\rho$.

Because all farmers would sell immediately when multiple buyers are present at harvest ($p_1 = 1$), we focus on the monopsony case with $p_1 = 0.5$. The farmer's problem in Equation~(\ref{eq:final objective}) then specializes to
\begin{equation}
\label{eq:Bertrand objective}
\max_{s \in [0,1]} \psi(s)
= \rho\, U(\pi_L(s)) + (1-\rho)\, U(\pi_H(s)),
\end{equation}

where
$$
\pi_L(s) = (1-s)\cdot 0.5 + \kappa s \cdot 0.5,
\qquad
\pi_H(s) = (1-s)\cdot 0.5 + \kappa s
$$
denote net income under low and high second-period prices, respectively.

Differentiating Equation~(\ref{eq:Bertrand objective}) with respect to $s$ gives the marginal utility gain from reallocating one unit of output from immediate sale to storage:

\begin{equation}
g(\cdot) = \frac{d\psi}{ds} = \rho U^{\prime}(\pi_L(s)) \cdot 0.5(\kappa-1) + (1 - \rho) U^{\prime}(\pi_H(s)) \cdot (\kappa-0.5)
\label{Eq: bertrand FOC}
\end{equation}
Evaluated at $s = 0$, i.e., no storage-the marginal utility becomes:
$$
\left.\frac{d\psi}{ds}\right|_{s=0} = U^{\prime}(0.5) \cdot \left[\kappa(1 - 0.5\rho ) - 0.5\right]
$$
The bracketed term reflects the expected second-period price, net of storage loss. Since $U'(\cdot) > 0$, the sign of this expression determines whether storage is utility-improving. In particular, the farmer finds storage profitable ($s^* > 0$) if and only if:
\begin{equation}
\kappa(1 - 0.5\rho ) > 0.5.
\label{eq:Bertrand threshold}
\end{equation}
Equation~(\ref{eq:Bertrand threshold}) offers the same interpretation as that generated from the FOOM framework in the main text. Expected second-period gains net of storage loss must exceed the harvest period monopsony price for storage to be attractive. Here, competition enters via the probability ($1-\rho$) of avoiding monopsony procurement in period~2, rather than through a continuous buyer-power index.


Under risk aversion ($U^{\prime\prime}<0$), $\psi(s)$ is strictly concave, so Equation~(\ref{Eq: bertrand FOC}) delivers a unique optimum. The storage policy can then be summarized as follows: If $N_1 \ge 2$, so $p_1 = 1$, store nothing ($s^*=0$). If $N_1=1$ and $\kappa(1 - 0.5\rho)\leq 0.5$, marginal utility at $s=0$ is non-positive and the farmer optimally sells the entire harvest at harvest ($s^*=0$). If $\kappa(1 - 0.5\rho)$ is sufficiently large that $g(1)\geq 0$, marginal utility remains non-negative throughout $[0,1]$ and full storage is optimal ($s^*=1$). For intermediate parameter values that render $g(0)>0$ and $g(1)<0$, there exists a unique interior solution $s^*\in(0,1)$ solving $g(s^*)=0$.

Local comparative statics with respect to storage efficiency $\kappa$ and the monopsony probability $\rho$ follow directly from the implicit function theorem applied to $g(s^*)=0$. Because $\partial g/\partial s = d^2\psi/ds^2<0$, we have
$$
\frac{\partial s^*}{\partial \xi} = -\,\frac{\partial g / \partial \xi}{\partial g / \partial s},
\qquad
\xi \in \{\kappa,\rho\}.
$$
An increase in storage efficiency ($\kappa$) unambiguously raises the marginal payoff to storing, so $\partial s^*/\partial \kappa>0$, consistent with findings in the main text. By contrast, the sign of $\partial s^*/\partial \rho$ is not determinate in general, reflecting the discrete nature of price jumps from monopsony to competitive pricing in Equation~(\ref{Eq: Bertrand Price Schedule}). This again parallels the FOOM specification: storage reacts smoothly and positively to better expected competitive conditions, but the precise response to changes in the distribution of buyer competition is best studied numerically. To summarize, the Bertrand-Nash formulation of buyer competition also yields the same qualitative predictions as in the main model: storage increases with higher expected competition and better storage efficiency, and declines with risk aversion.



\section{Aggregate Welfare Impacts}\label{subsec:agg_welfare}
\noindent
We now move from studying individual farmer storage decisions to investigating aggregate welfare implications from competition-induced commodity storage. Across developing and advanced economies alike, policymakers are deeply concerned with farm incomes. Traditional tools---price supports, input subsidies, or direct transfers---can raise incomes but cause market distortions and/or carry high fiscal costs. We highlight here the income benefits from improving the environment in which farmers make intertemporal selling decisions.

A key implication of the model is that eliciting stronger buyer competition through commodity storage can deliver income and welfare gains without distorting market signals. To quantify these gains, we extend the baseline framework by allowing stochastic buyer power in both periods and simulating a representative village of 100 farmers who share identical production conditions but differ in risk aversion. This structure allows us to evaluate how access to storage, along with intertemporal changes in buyer power or storage efficiency translate into aggregate welfare changes. We compute (i) the percentage change in income and (ii) the percentage change in certainty equivalent (CE) income for each farmer given their risk preferences and then aggregate welfare effects to the village level.

\paragraph{Environments and buyer power.}
Buyer power in periods 1 and 2, $\theta_1$ and $\theta_2$, is modeled as Beta-distributed with common variance $\sigma^2$ but potentially different means $\mu_1$ and $\mu_2$. For each admissible $(\mu,\sigma^2)$ pair, we recover $(\alpha,\beta)$ using the standard mean–variance mapping. Expectations about future competitive conditions are introduced through a mean gap $g\in\{0.00,0.05,0.15\}$, such that $\mu_2=\mu_1-g$. This framework captures cases in which farmers anticipate improved competition in the second period.

We vary storage policy through efficiency levels $\kappa\in\{0.95,0.90,0.85,0.80\}$ and uncertainty through variances $\sigma^2\in\{0.02,0.05,0.10,0.15\}$. For each $(\sigma^2,g)$ pair, $\mu_1$ is swept over a $0.05$ grid, subject to feasibility of both $(\mu_1,\sigma^2)$ and $(\mu_1-g,\sigma^2)$.\footnote{Feasible mean ranges for each variance appear in Figure~\ref{fig:parameter_range_Beta_distribution}.}$^,$ \footnote{Figure~\ref{fig:beta_pdf_cdf_grid} illustrates the corresponding Beta PDFs (top) and CDFs (bottom) across four variance levels and means $\mu\in\{0.2,0.4,0.6,0.8\}$. Darker curves denote higher means.}




\paragraph{Individual storage decision.}
The simulated village consists of $N=100$ farmers with CRRA preferences who differ only in risk aversion coefficient $\gamma_i\sim \mathrm{Uniform}[0,10]$ (drawn once and held fixed).

For a given $(\theta_1,\text{dist}(\theta_2),\kappa,\gamma)$, each farmer's optimal storage share $s^*(\theta_1,\gamma,\kappa)$ solves Equation~(\ref{eq:final objective}). Expectations over $\theta_2$ are evaluated by a 12-point Gauss--Legendre quadrature mapped to the Beta distribution. For each $\kappa$ and $(\mu_2,\sigma^2)$, we precompute $s^*$ on a $30\times 30$ grid in $(\theta_1,\gamma)$ and bi-linearly interpolate for arbitrary realizations of $\theta_1$. \footnote{When $\gamma\approx 0$, we use the standard risk-neutral shortcut: $s^*\in\{0,1\}$ depending on whether $\kappa\,\mathbb{E}[p_2]>p_1$.}

\paragraph{World simulation and aggregation.}
For every triplet $(\mu_1,\sigma^2,g)$, we simulate $R=20{,}000$ worlds.\footnote{we fix random seeds for reproducibility and reuse common random numbers across $\mu_1$ within a given $(\sigma^2,g)$ to improve comparability.} In world $r$, $(\theta_{1r},\theta_{2r})$ are drawn from their Beta distributions. Each farmer $i$ observes $\theta_{1r}$, knows the distribution of $\theta_{2r}$, and determines the optimal storage share, $s^*_i(\theta_{1r},\gamma_i,\kappa)$. Realized income is:
$$
y_{ir}=(1-s^*_{ir})\,p(\theta_{1r})+s^*_{ir}\,\kappa\,p(\theta_{2r}).
$$
We aggregate to the village mean in each world, $\bar y_r=\frac{1}{N}\sum_i y_{ir}$, $N=100$. The no-storage benchmark in world $r$ is $\bar y^{\,\text{no}}_r=\frac{1}{N}\sum_i p(\theta_{1r})$.



\paragraph{Income gain metric.}  
To quantify the effect of storage on aggregate revenues, we first construct a percentage income gain measure. For a given environment $(\kappa,\sigma^2,g)$ and initial mean $\mu_1$ buyer power, the expected proportional increase in average income with storage relative to the no-storage benchmark is:
\begin{equation}
\mathbb{E}[\Delta y_r](\mu_1;\kappa,\sigma^2,g)
=100\%\times
\frac{\frac{1}{R}\sum_{r=1}^R \bar y_{r}-\frac{1}{R}\sum_{r=1}^R \bar y^{\,\text{no}}_{r}}
{\frac{1}{R}\sum_{r=1}^R \bar y^{\,\text{no}}_{r}}.
\end{equation}
%which captures the average proportional improvement in farm income arising from intertemporal arbitrage to cope with time-varying buyer power.

\paragraph{Certainty–equivalent (CE) metric.}  
Because risk preferences are central to welfare, we complement income-based comparisons with a certainty-equivalent (CE) measure derived from CRRA utility. For farmer $i$, expected utility is $\mathrm{EU}_i=\frac{1}{R}\sum_{r=1}^R U(y_{ir};\gamma_i)$, and the corresponding CE income follows from the inverse function of CRRA utility,
\begin{equation}
\mathrm{CE}_i=
\begin{cases}
\left(\frac{1}{R}\sum_{r=1}^R y_{ir}^{\,1-\gamma_i}\right)^{\!\!\tfrac{1}{1-\gamma_i}}, & \gamma_i\neq 1,\\[8pt]
\exp\!\left(\frac{1}{R}\sum_{r=1}^R \ln y_{ir}\right), & \gamma_i=1.
\end{cases}
\end{equation}
Village-level welfare is the simple average, $\overline{\mathrm{CE}}=\frac{1}{N}\sum_{i=1}^N \mathrm{CE}_i$, with an analogous benchmark $\overline{\mathrm{CE}}^{\,\text{no}}$ based on immediate sale, $y^{\text{no}}_{ir}=p(\theta_{1r})$.  
The welfare effect of storage is then:
$$
\mathrm{CE\text{-}Gain}(\mu_1;\kappa,\sigma^2,g)
=100\% \times
\frac{\overline{\mathrm{CE}}-\overline{\mathrm{CE}}^{\,\text{no}}}{\overline{\mathrm{CE}}^{\,\text{no}}}.
$$

By construction, $\mathrm{CE}_i<\mathbb{E}[y_i]$ for any risk-averse farmer. But when comparing two policies---optimal storage versus no storage---the CE difference can exceed the difference in mean income. Writing
$$
\Delta \mathrm{CE}
=\Delta \mathbb{E}[y] - \Delta \mathrm{RP},
\qquad 
\mathrm{RP}=\mathbb{E}[y]-\mathrm{CE},
$$
the simulations show $\Delta\mathrm{RP}<0$.  
The reason is insurance: the payoff $(1-s) p_1 + s\kappa p_2$ has lower variance than $p_1$ whenever $\kappa<1$ and $s\in(0,1)$, and optimal choices place more weight on the future precisely in low-price states. This truncates downside tails and dampens upside exposure, reducing the risk premium.

Thus, relative to the no-storage benchmark, storage raises expected income and simultaneously smoothens risk. The CE-Gain measure captures both effects---a mean-income improvement plus an ``insurance dividend''---and therefore provides a more complete welfare metric than income gains alone.


\begin{figure}[ht!]
    \centering
    \includegraphics[width=\linewidth]{model_figures/gainpct_grid_4x4.png}
    \caption{Percent Mean Income Gain vs. No-Storage}
    \label{fig: Income gains}
\end{figure}



\subsection{Welfare versus No-Storage Benchmark}
\noindent 
Figure~\ref{fig: Income gains} reports a $4\times4$ grid of percentage income gains from intertemporal storage relative to the no-storage benchmark. Rows correspond to storage efficiencies $\kappa$ (from $0.95$ at the top to $0.80$ at the bottom). Columns reflect buyer-power uncertainty, with the variance $\sigma^2$ rising from $0.02$ (left) to $0.15$ (right). Within each panel, the horizontal axis plots the mean first-period buyer power $\mu_1$, and the vertical axis reports the resulting average village-level percent income gain from enabling storage.

Each panel includes three curves corresponding to alternative expectations about second-period buyer power. The dark-gray line sets $\mu_2=\mu_1$ (no expected change), while the light- and dark-green lines impose $\mu_2=\mu_1-0.05$ and $\mu_2=\mu_1-0.15$, respectively, to capture expectations of weaker buyer power in period 2. The vertical distance between curves therefore reflects sensitivity of income gains from storage to intertemporal beliefs about competition. Several insights emerge from this set of panels:
\begin{enumerate}
    \item \textit{Storage efficiency.}  
    For any level of buyer-power uncertainty, higher storage efficiency ($\kappa$) yields larger income gains, shifting all curves upward. Reducing storage losses strengthens the returns to intertemporal arbitrage.
    
    \item \textit{Intertemporal buyer-power expectations.}  
    When farmers expect weaker buyer power in period 2 (higher expected $p_2$), storage becomes systematically more profitable. The vertical gap between the dark-gray and green curves quantifies this effect and widens as the anticipated deterioration in buyer power increases.

    \item \textit{Buyer-power volatility.}  
    Larger variance in buyer power raises all curves, reflecting the convexity of $p(\theta)=1/(1+\theta)$ and the option value of storage under uncertainty. Greater volatility also amplifies differences across expectation scenarios, showing how risk interacts with beliefs to shape welfare gains.

    \item \textit{Average buyer power.}  
    Holding variance fixed, higher first-period mean buyer power ($\mu_1$), and the associated $\mu_2$, generally lowers the gains from storage because of the convexity of the $p(\theta)$ price schedule. When the base value of buyer power is high, the opportunity to generate significant price improvements from storage is limited, although greater volatility in buyer power increases the potential to attain a competitive outcome that will generate a significant price increase in period 2.

    \item \textit{Non-monotonicity.}  
    With high storage efficiency and large uncertainty, the relationship between $\mu_1$ and income gains becomes non-monotonic, generating an inverted-U shape.\footnote{The economic and mathematical underpinnings of the inverted-U in Figure~\ref{fig: Income gains} are discussed in Appendix~\ref{Section: Why Inverted U shapes in some Expected-Income-Gain Curves?}.} Moderate buyer power creates the greatest scope for welfare improvements, while very low or very high buyer power dampens the benefit of storage.

    \item \textit{Benefits to storage exist even when $\mu_1=\mu_2$}. Access to commodity storage can yield income gains even when there is no expectation of improved competition in period 2. Storage access protects farmers against a bad ``draw'' in buyer competition in the harvest period. This benefit becomes increasingly important as the variability in competitive outcomes becomes greater, i.e., as one moves from left to right in the figure~\ref{fig: Income gains} panels.
\end{enumerate}


Figure~\ref{fig:CE gains} reports certainty-equivalent (CE) gains relative to the no-storage benchmark. The qualitative patterns parallel those for expected income: higher storage efficiency, stronger expectations of future buyer competition, and greater uncertainty all raise the value of storage. The key distinction lies in how CE gains respond to mean buyer power. Whereas expected-income gains can exhibit an inverted-U in $\mu_1$ when efficiency and uncertainty are high, CE gains decline monotonically in $\mu_1$ across all environments. Once risk aversion is incorporated, the non-monotonicity disappears: farmer welfare gains from storage fall steadily as first-period buyer power (and, hence, expectations for second period buyer power) strengthen.



%----------------------------------------------------%
\subsection{Welfare Effects of Storage Efficiency Improvements}
\noindent 
The preceding analysis shows that access to storage can generate sizable welfare gains when buyer power is time-varying. In many settings, however, the binding constraint is not the absence of storage but its high effective cost from low storage efficiency.

Improvements in $\kappa$ can arise through multiple channels: improved access to credit that reduces farmer discounting, better rural roads that reduce transport losses to storage facilities, warehouse-receipt or inventory-financing schemes that ease liquidity constraints, or improved storage technologies that curb spoilage. Increasing $\kappa$ generates two distinct effects: a pure income transfer to infra-marginal storers,\footnote{Direct per-unit subsidies to storage costs to raise $\kappa$ disproportionately benefit infra-marginal storers, who are likely to be among the wealthier members of rural villages.} and it shifts the \emph{marginal storage threshold}, inducing some farmers who otherwise would sell at harvest to undertake storage. These marginal storers gain access to a second-period market that may feature weaker buyer power, amplifying the welfare impact beyond the transfer component. In this sense, raising $\kappa$ produces a super-additive effect, lowering effective costs and enabling more farmers to access improved competitive conditions.

The magnitude and distribution of these gains are heterogeneous. A given improvement in $\kappa$ delivers very different benefits depending on farmers' initial efficiency levels and their risk preferences, implying that uniform efficiency-enhancing policies may lead to uneven welfare outcomes across producer groups.

\begin{figure}[ht]
    \centering
    \includegraphics[width=\linewidth]{model_figures/storage_subsidy_gain_decomposition.png}
    \caption{Decomposition of Welfare Gains from Storage Efficiency Improvement by Risk Aversion}
    \label{fig:storage_subsidy_gain_decomposition}
    \begin{tablenotes}[flushleft]
    \footnotesize
    \item \textit{Notes:} Each panel reports results for one storage-efficiency contrast. For each buyer-power mean $\mu=\mu_1=\mu_2$, we simulate $R=5000$ worlds $(\theta_1,\theta_2)$ from Beta$(\mu,\sigma^2=0.10)$. Farmers ($N=100$) have heterogeneous risk preferences, with $\gamma \sim U[0,10]$ grouped into deciles. Bars show the decomposition of welfare gains into income growth ($\Delta E[\pi]$) and insurance ($-\Delta RP$); the line shows the certainty-equivalent gain ($\Delta CE$). All values are averaged across $\mu \in [0.15,0.85]$ in steps of 0.05.
    \end{tablenotes}
\end{figure}

In Figure~\ref{fig:storage_subsidy_gain_decomposition}, each panel corresponds to an efficiency improvement,
$\kappa \in \{0.75\!\to\!0.80,\,0.80\!\to\!0.85,\,0.85\!\to\!0.90,\,0.90\!\to\!0.95\}$. For each mean buyer-power level $\mu=\mu_1=\mu_2 \in [0.15,0.85]$ (in 0.05 increments), we simulate $R=5000$ realizations of $(\theta_1,\theta_2)$ from Beta distributions with variance $\sigma^2=0.10$. $N=100$ farmers who differ in risk aversion are grouped into deciles, indexed on the horizontal axis. Within each decile, we plot: (i) the expected-income effect $\Delta E[\pi]$, from commodity storage, (ii) the insurance component $-\Delta RP$ (the reduction in risk premium), and (iii) the certainty-equivalent effect $\Delta CE$.

The decomposition shows that improvements in storage efficiency operate through two channels: income growth and risk reduction. In most cases, $\Delta CE$ exceeds $\Delta E[\pi]$, implying $\Delta RP<0$: better storage reduces income variance and hence the risk premium. This follows because, for $\kappa<1$ and interior $s^*\in(0,1)$, the payoff $(1-s^*)p_1 + s^*\kappa p_2$ has lower variance than $p_1$, since  
$(1-s^*)^2 + (s^*\kappa)^2 < 1$. Optimal storage therefore acts as a hedge, shifting more weight to the future price in low-price states and trimming downside tails. Even modest expected-income gains can translate into sizable CE improvements through risk smoothing.

The magnitude of these gains varies with the initial efficiency level.  
When $\kappa$ is already high (e.g., $0.90\!\to\!0.95$), incremental welfare gains are larger than in low-efficiency steps (e.g., $0.75\!\to\!0.80$). This convexity reflects reinforcement on both margins: infra-marginal storers benefit directly from lower holding costs, while marginal farmers expand storage, amplifying the total response.

At low baseline $\kappa$, gains are smaller in absolute value but dominated by insurance effects, particularly for highly risk-averse deciles. In the $0.75\!\to\!0.80$ step, most benefits arise through risk-premium reductions, effectively acting as insurance subsidies for vulnerable farmers. At higher $\kappa$, risk reduction plays a smaller role and welfare improvements stem primarily from income growth.

Overall, Figure~\ref{fig:storage_subsidy_gain_decomposition} highlights a dual policy message:  efficiency improvements for low initial $\kappa$ mainly provide risk protection to highly risk-averse farmers, while improvements from high initial $\kappa$ generate stronger income gains by expanding storage participation and access to more competitive second-period markets. 


\begin{figure}[ht]
    \centering
    \includegraphics[width=\linewidth]{model_figures/storage_subsidy_gain_heatmap.png}
    \caption{Welfare Gains from Storage Efficiency Improvement: Line and Heatmap Decomposition across Buyer-Power Means and Risk Aversion}
    \label{fig: storage_subsidy_gain_heatmap}
    \begin{tablenotes}[flushleft]
    \footnotesize
    \item \textit{Notes:} Each row shows one storage-efficiency improvement contrast. Line charts (first column) plot changes in expected income ($\Delta E[\pi]$), certainty equivalent ($\Delta CE$), and insurance gain ($-\Delta RP$) against belief mean $\mu$. Heatmaps (columns 2--4) decompose these effects across $\gamma$-deciles, highlighting that welfare improvements are strongest for more risk-averse farmers and grow with the size of the improvement.
    \end{tablenotes}
\end{figure}

To dig deeper, Figure~\ref{fig: storage_subsidy_gain_heatmap} examines how welfare effects vary across the full range of buyer-power means ($\mu$, x-axis), holding variance fixed at $0.10$. Each row corresponds to one storage-efficiency increment, consistent with Figure~\ref{fig:storage_subsidy_gain_decomposition}. The first column presents line plots for the three welfare components (expected income, certainty equivalent, and the insurance gain), while the remaining columns show heatmaps that further decompose outcomes by risk-aversion deciles ($\gamma$, y-axis). Lighter shades indicate stronger welfare improvements from each per-unit efficiency increment, highlighting how both mean buyer power across periods and heterogeneity in risk preferences shape distributional impacts.

The line plots show a clear pattern. When buyer power is weak (low $\mu$), income and CE gains are similar across all increments, largely independent of initial efficiency. As buyer power strengthens, however, improvements in storage efficiency applied at higher baseline efficiency deliver disproportionately larger gains. This is visible in the flattening slopes of the income and CE curves as one moves down the rows of column~1. In effect, the aggregate welfare differences highlighted in Figure~\ref{fig:storage_subsidy_gain_decomposition} arise primarily from the superior performance of high-storage-efficiency villages when buyer power is strong in both periods, rather than from differences at the low end of the buyer-power distribution.


%----------------------------------------------------%
\subsection{Welfare Effects of Lower Farmer Risk Aversion}
\noindent 
Beyond storage subsidies, the aggregate welfare value of intertemporal storage depends critically on farmer risk preferences. Although risk attitudes are partly innate, they are also shaped by financial frictions, so improved access to credit can also reduce risk aversion.

We evaluate how reduced risk aversion in a village enhances the value of storage when expected buyer power is the same in both periods. Figure~\ref{fig: Income gains with RN case} reports village-average income gains from storage relative to the no-storage benchmark under heterogeneous risk preferences. The figure mirrors the structure of Figure~\ref{fig: Income gains}, with rows varying storage efficiency $\kappa$ and columns varying buyer-power variance $\sigma^2$. Within each panel, the horizontal axis plots the common mean buyer power, $\mu=\mu_1=\mu_2$. Three curves compare alternative risk environments in the 100-farmer village: (i) the risk-neutral envelope ($\gamma=0$); (ii) moderately risk-averse farmers ($\gamma_i\sim \mathrm{Unif}[0,5]$); and (iii) more risk-averse farmers ($\gamma_i\sim \mathrm{Unif}[0,10]$). The vertical distance between the risk-neutral and heterogeneous-risk curves measures the income lost to risk-averse behavior.

\begin{figure}[ht!]
    \centering
    \includegraphics[width=\linewidth]{model_figures/gainpct_grid_4x4_zero_gap_three_curves.png}
    \caption{Percent Mean Income Gains under High, Median, and No Risk Aversion}
    \label{fig: Income gains with RN case}
    \begin{tablenotes}[flushleft]
    \footnotesize
    \item \textit{Notes:} Panels show mean income gains from storage relative to no-storage under the zero-gap case ($\mu_2=\mu_1$). Rows vary storage efficiency and columns vary buyer-power variance. Curves compare heterogeneous farmers with $\gamma_i \sim \text{Unif}[0,10]$ (black solid) and $\gamma_i \sim \text{Unif}[0,5]$ (blue dashed) against the risk-neutral benchmark ($\gamma=0$, light-blue dash-dotted). Results average $R=20{,}000$ simulated worlds with $N=100$ farmers each.
    \end{tablenotes}
\end{figure}

Lower risk aversion shifts the heterogeneous--$\gamma$ curves upward toward the risk-neutral envelope. Under higher buyer-power variance (right-hand columns), reducing village risk aversion delivers additional income gains of roughly 1--5 percentage points, with the largest improvements occurring when $\kappa$ is high and $\mu$ lies in the interior of the range. By contrast, in low-return environments (bottom-left panels with low $\kappa$ and low $\sigma^2$), all curves cluster near zero at low $\mu$, and differences across risk profiles are economically minor.

Two comparative statics emerge clearly. First, holding $\sigma^2$ fixed, the gap between risk-neutral and heterogeneous $\gamma$ outcomes grows with $\kappa$. Higher storage efficiency increases the payoff to reallocating revenue inter-temporally; risk-neutral farmers take full advantage, while more risk-averse farmers optimally choose lower $s^*$ and leave potential income gains unrealized. Second, holding $\kappa$ fixed, the income gap between risk-neutral and risk-averse behavior rises with $\sigma^2$. Greater volatility in buyer competition raises the value of having multiple draws of competitive outcomes but also exacerbates the downside risk and deters accessing storage for the most risk averse. Thus, higher $\kappa$ and higher $\sigma^2$ jointly amplify the stakes of the storage decision and widen the gap between risk-averse and risk-neutral behavior.




\begin{table}[ht!]\centering
\caption{Welfare Gains of Storage when $\mu_1=\mu_2\in\{0.2,0.5,0.8\}$}
\label{tab:mu_cases_storage_outcomes}
\begin{threeparttable}
\vspace{0.35em}
\noindent\textbf{Panel A. Expected Income Gain vs No Storage (\%)}
\vspace{0.25em}
\begin{tabular}{l|ccc|ccc|ccc}
\toprule
 & \multicolumn{3}{c|}{$\mu=0.2$} & \multicolumn{3}{c|}{$\mu=0.5$} & \multicolumn{3}{c}{$\mu=0.8$} \\
Var($\theta$) & $\kappa=0.80$ & $\kappa=0.90$ & $\kappa=0.95$ & $\kappa=0.80$ & $\kappa=0.90$ & $\kappa=0.95$ & $\kappa=0.80$ & $\kappa=0.90$ & $\kappa=0.95$ \\
\midrule
0.02 & 0.16 & 1.22 & 2.42 & 0.00 & 0.47 & 1.61 & 0.00 & 0.00 & 0.79 \\
0.05 & 0.99 & 2.89 & 4.32 & 0.09 & 1.71 & 3.53 & 0.00 & 0.05 & 1.88 \\
0.10 & 2.36 & 4.62 & 6.01 & 0.61 & 3.61 & 6.01 & 0.00 & 0.72 & 3.66 \\
0.15 & 3.60 & 5.87 & 7.26 & 1.45 & 5.34 & 8.35 & 0.00 & 1.59 & 5.15 \\
\bottomrule
\end{tabular}

\vspace{0.35em}
\noindent\textbf{Panel B. Certainty Equivalent Gain vs No Storage (\%)}
\vspace{0.25em}
\begin{tabular}{l|ccc|ccc|ccc}
\toprule
 & \multicolumn{3}{c|}{$\mu=0.2$} & \multicolumn{3}{c|}{$\mu=0.5$} & \multicolumn{3}{c}{$\mu=0.8$} \\
Var($\theta$) & $\kappa=0.80$ & $\kappa=0.90$ & $\kappa=0.95$ & $\kappa=0.80$ & $\kappa=0.90$ & $\kappa=0.95$ & $\kappa=0.80$ & $\kappa=0.90$ & $\kappa=0.95$ \\
\midrule
0.02 & 0.39 & 2.10 & 3.62 & 0.00 & 0.60 & 1.90 & 0.00 & 0.00 & 0.71 \\
0.05 & 2.45 & 5.80 & 7.90 & 0.12 & 2.09 & 4.20 & 0.00 & 0.01 & 1.40 \\
0.10 & 5.72 & 10.19 & 12.67 & 0.70 & 4.17 & 6.89 & 0.00 & 0.49 & 2.50 \\
0.15 & 8.59 & 13.43 & 15.99 & 1.65 & 6.04 & 9.10 & 0.00 & 1.18 & 3.42 \\
\bottomrule
\end{tabular}

\vspace{0.35em}
\noindent\textbf{Panel C. Mean Storage Share}
\vspace{0.25em}
\begin{tabular}{l|ccc|ccc|ccc}
\toprule
 & \multicolumn{3}{c|}{$\mu=0.2$} & \multicolumn{3}{c|}{$\mu=0.5$} & \multicolumn{3}{c}{$\mu=0.8$} \\
Var($\theta$) & $\kappa=0.80$ & $\kappa=0.90$ & $\kappa=0.95$ & $\kappa=0.80$ & $\kappa=0.90$ & $\kappa=0.95$ & $\kappa=0.80$ & $\kappa=0.90$ & $\kappa=0.95$ \\
\midrule
0.02 & 0.03 & 0.15 & 0.25 & 0.00 & 0.11 & 0.26 & 0.00 & 0.01 & 0.26 \\
0.05 & 0.10 & 0.19 & 0.25 & 0.03 & 0.21 & 0.33 & 0.00 & 0.09 & 0.39 \\
0.10 & 0.14 & 0.20 & 0.22 & 0.11 & 0.28 & 0.37 & 0.00 & 0.20 & 0.44 \\
0.15 & 0.16 & 0.18 & 0.19 & 0.17 & 0.32 & 0.39 & 0.00 & 0.25 & 0.46 \\
\bottomrule
\end{tabular}

\vspace{0.35em}
\noindent\textbf{Panel D. Share of Storage Adopters}
\vspace{0.25em}
\begin{tabular}{l|ccc|ccc|ccc}
\toprule
 & \multicolumn{3}{c|}{$\mu=0.2$} & \multicolumn{3}{c|}{$\mu=0.5$} & \multicolumn{3}{c}{$\mu=0.8$} \\
Var($\theta$) & $\kappa=0.80$ & $\kappa=0.90$ & $\kappa=0.95$ & $\kappa=0.80$ & $\kappa=0.90$ & $\kappa=0.95$ & $\kappa=0.80$ & $\kappa=0.90$ & $\kappa=0.95$ \\
\midrule
0.02 & 0.05 & 0.23 & 0.34 & 0.00 & 0.16 & 0.35 & 0.00 & 0.06 & 0.35 \\
0.05 & 0.16 & 0.27 & 0.36 & 0.08 & 0.32 & 0.45 & 0.00 & 0.35 & 0.55 \\
0.10 & 0.21 & 0.26 & 0.30 & 0.26 & 0.42 & 0.50 & 0.00 & 0.64 & 0.71 \\
0.15 & 0.21 & 0.21 & 0.22 & 0.38 & 0.48 & 0.51 & 0.00 & 0.78 & 0.79 \\
\bottomrule
\end{tabular}
\
\begin{tablenotes}[flushleft]
\footnotesize
\item Notes: $N=100$, $R=20000$, $\gamma\sim U[0,10]$. Prices $p=1/(1+\theta)$. $\theta_1,\theta_2\sim \text{Beta}(\alpha,\beta)$ calibrated to $(\mu,\sigma^2)$. Policy $s^*(\theta_1,\gamma)$ solved by golden-section; expectation over $\theta_2$ by 12-point Gauss–Legendre quadrature. Adopters: $s^*>0.01$.
\end{tablenotes}
\end{threeparttable}
\end{table}



\subsection{Numerical Outcomes under Identical Buyer Power Distributions}
\noindent 
Table~\ref{tab:mu_cases_storage_outcomes} complements the preceding results by examining a relatively unfavorable environment for producer storage in which buyer power in both periods follows the same distribution, $\mu_1=\mu_2\in\{0.2,0.5,0.8\}$. Intertemporal arbitrage opportunities here arise solely from volatility in competition rather than shifts in its mean. Even under these restrictive conditions, storage delivers meaningful welfare gains once either volatility or efficiency moves beyond low levels, consistent with our option-value logic.

Panel~A shows that expected-income gains are largest when buyer power is low to moderate ($\mu=0.2,0.5$). Gains rise with both efficiency and volatility, reaching 7--8\% at high $\kappa$ and Var($\theta$). When buyer power is uniformly strong ($\mu=0.8$), timing gains are minimal unless efficiency is very high, and many cells are essentially zero.

Risk preferences shift this picture. Panel~B shows that certainty-equivalent (CE) gains substantially exceed mean-income gains when mean buyer power is low across periods ($\mu=0.2$): CE improvements reach 16\% versus roughly 7\% for mean income, highlighting the insurance value of storage. At high buyer power ($\mu=0.8$), this insurance premium disappears. The intermediate case ($\mu=0.5$) lies between these extremes, with only modest CE–income wedges.

Panels~C and D separate extensive and intensive storage responses. The adoption share responds strongly to both volatility and efficiency: at high $\mu$, participation rises from near zero at low Var($\theta$)=0.02 to nearly 80\% at high Var($\theta$)=0.15. By contrast, mean storage shares increase more modestly. More farmers adopt storage but store only small amounts, using storage mainly as a contingent hedge. In favorable competitive market environments, greater volatility even lowers average storage shares, as farmers reallocate more selectively across states.





\section{Further Extensions}

\subsection{Other Models of Risk Preference}
\noindent According to \cite{o2018modeling}, ``Real-world risk aversion is clearly not as straightforward as expected utility suggests. Additional sources of risk aversion (or risk-seeking) need to be used instead of, or in conjunction with, diminishing marginal utility of wealth.'' Individuals assess options involving both gains and losses, the ``kink'' in the value function between losses and gains induces risk aversion. Therefore, future researchers may extend my model by exploring the Loss Aversion models \citep{kahneman1979prospect}, in addition to the expected utility theorem. The approach developed by \cite{kHoszegi2006model, kHoszegi2007reference, kHoszegi2009reference} in addressing loss aversion with an endogenous reference point aims to mitigate this degree of freedom by asserting that the reference point is entirely determined by one's expectations about outcomes. Empirical work also highlights the relevance of these alternative frameworks: for instance, \cite{liu2013risk} show that Chinese Bt cotton farmers' pesticide use is systematically related to their risk and loss aversion, with more risk-averse farmers applying greater quantities of pesticides and more loss-averse farmers applying less. Despite this evidence, there has been limited progress in adapting alternative models to dynamic settings, with a notable exception of \cite{kHoszegi2009reference}, who define loss aversion concerning changes in beliefs regarding both current and future consumption.



\subsection{Collective-Action Dilemma: The Storage ``Treadmill''}
\noindent Field observations suggest a collective-action dilemma surrounding storage adoption. As more farmers in a village adopt storage, middlemen may redirect procurement efforts to neighboring villages with less storage penetration, where farmers face weaker bargaining positions. This strategic shift reduces buyer competition at harvest in the original village, particularly disadvantaging farmers without storage. Consequently, farmers who have not adopted storage face mounting pressure to invest, not necessarily for higher profits, but to avoid worsening terms of trade. 

This dynamic mirrors the classic ``technology treadmill'' described by Cochrane \citep{cochrane1958farm, levins1996treadmill}. It describes how technological progress in agriculture often fails to generate lasting gains for farmers. Early adopters of a new technology temporarily benefit from lower costs and higher profits, but as adoption spreads, aggregate supply expands and output prices fall due to the inelastic demand for food. These price declines erode the initial advantages, leaving non-adopters increasingly disadvantaged and eventually compelling them to adopt merely to maintain their previous income levels. In this way, technology adoption becomes a defensive necessity rather than a source of sustained profit. Future extensions of the model could incorporate this endogenous spillover, possibly via a parameter capturing buyer response to aggregate storage rates, highlighting how individually rational behavior may generate suboptimal collective outcomes.






\subsection{Non-linear Storage Cost}
\noindent Storage costs for farmers, defined as the sum of the actual cost of renting or operating storage space and the cost of deterioration, could be non-linear in many cases. For perishable goods like apples, the primary reason for convex storage costs is probably that spoilage and quality deterioration increase over time. Also, the operating costs of storage may not increase linearly with the volume of goods stored. Managing temperature and humidity conditions for larger quantities might require more sophisticated technology and energy consumption, leading to non-linear cost increases.

Supporting this notion, \cite{williams1989economic} demonstrates, through a quadratic form of total marketing costs, that farmers need to weigh the marginal revenue of later sales against the elevated costs incurred. This balance leads to a positive inventory, even without considering risk aversion. Instead of capturing the composite storage cost by a discounting factor, a well-designed mathematical model would allow other forms of storage costs, therefore, may give us other relevant real-world policy implications.


\section{Conclusion and Policy Implications}
\noindent This paper has developed a conceptual framework to study farmers' storage decisions when buyer market power varies within a marketing season. Storage gives farmers an option to shift sales from periods of relatively strong buyer power into potentially more competitive periods and, when used partially, provides an insurance mechanism by enabling sales to be allocated across time. By embedding a time-varying buyer conduct parameter into a two-period storage model, we show that temporal variation in buyer competition alone can make access to storage an income-enhancing tool for farmers.

Prior work has emphasized the implications of spatial immobility of farm products for buyer market power, especially in developing-country settings, and the manner in which reducing transport costs can enhance competitive outcomes (e.g., \citet{MSS-transport}). To our knowledge, this paper is the first to focus on the implications of temporal immobility of farm products for buyer power and the manner in which improved access to storage can enable farmers to achieve more competitive market outcomes.

The individual- and village-level simulations yield multiple policy-relevant insights. Stronger buyer competition at harvest raises farm-gate prices. Storage allows farmers to benefit whenever competition improves within the selling window. Even when competitive conditions are stable on average over time, welfare gains from storage are possible depending on storage efficiency, volatility in buyer power, and farmers' risk aversion. Improving storage efficiency directly augments income for infra-marginal storers and converts at least some non-storers into choosing partial or full storage. Greater volatility in competition increases the option value of waiting, and certainty-equivalent gains are largest when harvest-period buyer power is initially strong and insurance value is high. Measures that effectively lower farmer risk aversion matter most when storage is relatively efficient or volatility is substantial.

In addition to policies that enhance farmers' access to efficient storage, policies that relax farmers' liquidity constraints or expand financial access also hold potential to raise farmer incomes through expanded use of storage. For example, policies that expand access to credit and insurance can reduce risk aversion and raise effective storage efficiency by lowering farmer discount rates, enhancing their willingness to defer sales into periods of greater competition and higher incomes.




% %---------------------------%
% \newpage
% \section{Memo: A Difficult Tradeoff between Two Utility Aggregation Approaches}
% \noindent As I finalize revisions on the current model, I identify a major conceptual drawback: the optimal storage share becomes increasing in the farmer's degree of risk aversion when the expected second-period buyer power is higher than the first-period level. In other words, the model implies that farmers who expect a worse market condition in the future would store more if they are more risk-averse. This result is counterintuitive and lacks a clear behavioral or economic justification.

% This inconsistency has prompted me to revisit an alternative formulation we previously considered. Specifically, we face a fundamental modeling choice between:
% \begin{itemize}
%     \item Maximizing the expected sum of discounted utilities of income, versus
%     \item Maximizing the utility of the sum of discounted incomes.
% \end{itemize}
% Both approaches are theoretically sound but carry distinct implications. Below, I outline the tradeoffs between the two so we can more clearly evaluate the direction forward.

% \subsection{Current Setting: Sum of Discounted Utilities of Income}

% \noindent This is the standard formulation in intertemporal utility theory, where utility is evaluated separately in each period and future utility is discounted by a factor $\delta$:
% \begin{equation}
% \max_{s \in [0,1]} ; U\left((1 - s) p_1\right) + \delta \cdot \mathbb{E} \left[ U\left(s \cdot p_{2,\text{net}} \right) \right].
% \end{equation}

% \noindent Key assumptions and implications:
% \begin{itemize}
% \item Utility is additive and separable across time.
% \item The agent is risk-averse within each period, but not across periods.
% \item This structure implies a preference for income smoothing over time.
% \end{itemize}

% The primary advantage of this formulation lies in its analytical tractability. The separability allows us to derive a closed-form solution for the optimal storage share $s^*$, even under nontrivial distributional assumptions about future buyer power. It fits neatly within the expected utility framework and supports comparative statics analysis with intuitive results, except for the key case of higher second-period buyer power involving risk aversion.

% Through simulation and sensitivity checks (see Figure), I observe a troubling result: when expected second-period buyer power is pessimistic ($\theta_2$ is larger), the optimal storage share becomes increasing in risk aversion. This is counterintuitive. Typically, greater risk aversion should lead agents to reduce exposure to future uncertainty--i.e., store less. This expected behavior holds in the left region of the figure (where expectations about the future are optimistic), but breaks down in the right region.

% Upon closer inspection, this anomaly stems from the fact that the model assumes risk aversion only within periods. The framework lacks intertemporal risk aversion--i.e., there is no mechanism to penalize uncertainty in total income across periods. Unless we fundamentally alter the utility function (e.g., by adopting a non-additive or piecewise utility structure), this counterintuitive outcome cannot be avoided within this setting.

% One might ask how prior literature addresses this. The answer is: they often avoid it by making restrictive assumptions or narrowing the scope of analysis. For instance, \citet{ruhinduka2020smallholder} (the work by Travis Lybbert and his co-authors studying the post-harvest storage decisions) assume $\mathbb{E}(p_2) > p_1$, ensuring that $\partial s^* / \partial \gamma < 0$, thus sidestepping the unintuitive regime altogether.


% \subsection{Alternative Setting: Utility of the Sum of Discounted Income}
% \noindent This alternative formulation, similar to the structure used in your work with Saitone and Malan \citep{saitone2018price}, applies the utility function to the entire stream of income, aggregated and discounted to the present:
% \begin{equation}
% \max_{s \in [0,1]} ;\mathbb{E} \left(U\left[ (1 - s) p_1 + \delta \cdot s \cdot p_{2,\text{net}} \right]\right).
% \end{equation}

% \noindent Key assumptions and implications:
% \begin{itemize}
% \item Utility is applied to the total discounted income as a single aggregated payoff.
% \item The agent exhibits intertemporal risk aversion--preferences depend on risk in the total income stream, not just within each period.
% \item This formulation is commonly used in investment and project evaluation models, especially under uncertainty.
% \end{itemize}

% The primary strength of this setting is its consistency with empirical observations. Simulations under a CRRA utility specification (see Figure show that both risk-neutral and most moderately risk-averse farmers tend to make corner solutions--either storing everything or nothing--which aligns well with real-world behavior. Interior solutions exist but are confined to a narrow region of the parameter space, unlike the smoother, wider range of partial storage outcomes generated by the standard model.

% The main limitation, however, is analytical intractability. Closed-form solutions for the optimal storage share $s^*$ are not available under arbitrary distributions of $\theta_2$ in our case. Solving this model typically requires numerical methods, which restrict our ability to conduct formal comparative statics. An exception arises in cases with a finite number of discrete scenarios--e.g., buyer competition modeled as Bertrand duopoly with known probabilities--in which case closed-form solutions may be derived. Otherwise, we must rely on simulation-based approximation.

% \subsection{3D Visualizations of Both Formulations}

% \noindent I conduct simulations for both utility formulations over a grid defined by the coefficient of relative risk aversion ($\gamma$) and the storage efficiency factor ($\kappa$). The first-period buyer power is fixed at $\theta_1 = 0.6$, and the discount factor is set at $\delta = 1.0$ throughout.

% Figure~\ref{fig:3D_formulation} presents a $4 \times 4$ panel summarizing the simulation results. I examine eight distinct Beta distributions for $\theta_2$, varying in both the mean and variance: two levels of variance--low ($\sigma^2 = 0.02$) and high ($\sigma^2 = 0.05$)--and four values of the mean: $\mu \in {0.2,,0.4,,0.6,,0.8}$.

% \begin{itemize}
%     \item \textbf{Top and Bottom Rows}: These panels display the Beta probability density functions used in the simulations. The top row corresponds to low-variance cases ($\sigma^2 = 0.02$); the bottom row to high-variance cases ($\sigma^2 = 0.05$). Each panel is annotated with the corresponding values of $\mu$, $\sigma^2$, $\alpha$, and $\beta$.
%     \item \textbf{Middle Rows (2 and 3)}: These panels depict the optimal storage share $s^*$ as a 3D surface over the $\gamma$-$\kappa$ grid, conditional on the corresponding $\theta_2$ distribution. The second row shows results under low variance; the third row under high variance.
% \end{itemize}

% Across both formulations, the columns from left to right represent increasing expectations of future buyer power relative to the current level ($\theta_1 = 0.6$):
% \textit{much lower}, \textit{moderately lower}, \textit{approximately equal}, and \textit{much higher}.

% In Figure (corresponding to the sum of discounted utilities formulation), the first two columns align with economic intuition: higher risk aversion leads to lower storage under optimistic expectations, consistent with a dislike of uncertainty. However, in the last two columns--where future buyer power is expected to match or exceed $\theta_1$--the model yields counterintuitive results: only risk-neutral or near risk-neutral farmers avoid storage, while higher risk aversion induces greater storage shares, contrary to standard risk-averse behavior.

% By contrast, in Figure (corresponding to the utility of total discounted income formulation), the patterns are more intuitive. In the third and fourth columns--where second-period buyer power is expected to be comparable to or stronger than in the first period--nearly all farmers choose not to store at all ($s^* = 0$), regardless of risk preferences. This outcome reflects a stronger aversion to future uncertainty at the level of total income, and the model's capacity to rationalize boundary decisions aligns more closely with observed behavior.




% \subsection{Decision Point: Selecting the Modeling Framework}

% \noindent In summary, we face a fundamental modeling choice going forward. Each option carries significant tradeoffs in terms of interpretability, tractability, and alignment with empirical behavior:

% \begin{enumerate}
%     \item \textbf{Maintain the current formulation}--maximize the expected sum of discounted utilities of income.
% This approach is analytically tractable and allows for closed-form solutions and comparative statics. However, it yields a counter-intuitive prediction: under pessimistic expectations for future buyer power, more risk-averse farmers are predicted to store more, not less. Proceeding with this model requires us to develop a compelling economic rationale or behavioral interpretation for this result.
%     \item \textbf{Switch to the alternative formulation}--maximize the utility of the sum of expected discounted income.  
% This setting avoids the problematic comparative statics and aligns more closely with observed farmer behavior (e.g., boundary solutions). However, it sacrifices analytical tractability. We would need to rely on simulation-based numerical solutions, or begin with a tractable case (e.g., Bertrand competition with discrete buyer power realizations) and then generalize through approximation.
% \end{enumerate}
