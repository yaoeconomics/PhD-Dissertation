Thank you for the clarification. Incorporating the correction to the expression for the interior optimal storage share under risk aversion, we revise the analysis accordingly. The correct expression is:

\begin{equation}
s^\* = \left(1 + \left\[ \delta \cdot \kappa^{1 - \gamma} \cdot
\frac{\mathbb{E}\left\[ \left(1 + \theta\_1 + \Delta\_\theta \right)^{\gamma -1} \right]}
{\left(1 + \theta\_1\right)^{\gamma -1}} \right]^{-1/\gamma} \right)^{-1}.
\label{Eq: interior solution under risk averse}
\end{equation}

Letting the bracketed term again be denoted $\Lambda$, the storage share is now:

$$
s^* = \left(1 + \Lambda^{-1/\gamma} \right)^{-1},
\quad \text{where} \quad
\Lambda = \delta \cdot \kappa^{1 - \gamma} \cdot 
\frac{\mathbb{E}\left[ \left(1 + \theta_1 + \Delta_\theta \right)^{\gamma -1} \right]}
{\left(1 + \theta_1\right)^{\gamma -1}}.
$$

We now reassess the comparative statics with this corrected specification.





The effect of initial buyer power $\theta_1$ is also mediated through both the numerator and denominator of the expectation ratio. Since both $(1 + \theta_1 + \Delta_\theta)^{\gamma - 1}$ and $(1 + \theta_1)^{\gamma - 1}$ are decreasing in $\theta_1$, but the former declines more slowly due to the convexity of the function, the ratio decreases. Therefore:

$$
\frac{\partial \Lambda}{\partial \theta_1} < 0 \Rightarrow \frac{ds^*}{d\theta_1} < 0,
$$

indicating that higher first-period buyer power, which reduces immediate price, still makes future sales relatively less attractive due to increased effective discounting from worse expected future prices.







In summary, under the corrected formulation of the optimal storage share, $s^*$ decreases in the mean and variance of buyer power shocks, in first-period buyer power, and in the degree of risk aversion, while increasing in storage efficiency and time patience. These results reinforce the core intuition: storage is attractive when future revenues are relatively stable, well-compensated, and efficiently realized.
